\chapter{Physics Objects}
\label{sec:objects}

The ATLAS detector records events as raw data, 
which correspond to bits of electric signal collected 
when particles interact with the detectors. 
The goal of object identification is to reconstruct the particle four-momenta
by combining the information from the different sub-detectors.
This task is performed by algorithms optimized for the reconstruction of
electrons (Sec.~\ref{sec:electrons}), muons (Sec.~\ref{sec:muons}), 
jets (Sec.~\ref{sec:jets}) and the energy imbalance left
by the passage of a neutrino (Sec.~\ref{sec:met}).

\section{Tracks}
\label{sec:tracks}

In the solenoidal magnetic field of the inner detectors,
a charged particle moves along a helicoidal trajectory with a 
curvature proportional to its momentum.
Tracks are the reconstruction of these trajectories from the electric 
signals induced in the silicon detectors by ionization.
Therefore, tracks are used to identify charged particles and measure
their momenta. In addition, the extrapolation of the trajectories
allows to identify the interaction vertices.
 
In order to reconstruct the track, the first step is to retrieve the information from
the ID hits, which are converted into three-dimensional space points. Then, 
the {\it inside-out} algorithm~\cite{insideoutalgo} iteratively builds
a track by combining space points one by one, starting from a
seed of three aligned hits in the pixel detector or in the SCT.
For each new point a Kalman filter algorithm~\cite{kalmanfilter}
checks the compatibility between the track and the new point. 
A cleaning procedure prunes the track collection, removing 
tracks sharing hits with other tracks, or reconstructed from noise hits. 
Finally, the track quality is improved by taking into account the signal
from the TRT and the effects from the interaction
of the charged particle with the detector material. 

For signals in the TRT which are not associated to any track candidate
by the inside-out reconstruction, a second algorithm,
called {\it outside-in}, is applied in order to
reconstruct tracks from secondary charged particles. 
The algorithm uses as seeds hits in the TRT
and extrapolates back to the SCT and pixel detector.


\section{Primary Vertices}
\label{sec:pv}


\section{Electrons}
\label{sec:electrons}

An electron candidate object~\cite{elereco} is selected by searching 
for a narrow, 
localized cluster of energy deposits in the EM calorimeter, 
with at least one ID track associated to it.

A {\it sliding-window} clustering algorithm is used to identify electron 
clusters. The algorithm performs a scan of the calorimeter searching 
for local maxima of energy
within a $\eta$-$\phi$ window of $0.075\times{}0.125$.
The scan is performed in the range $|\eta_{\mathrm{cluster}}|<2.47$, 
which corresponds to the ID coverage for reconstructing tracks. 
Tracks within a $\eta$-$\phi$ window of $0.05\times{}0.10$ are 
associated with the cluster.

The electron four-momentum is built from the cluster energy
and the direction of the associated ID track.
When multiple tracks are associated with the cluster, the closest
to the cluster direction\footnote{The position of a cluster 
is computed as a weighted average of the $\eta$-$\phi$ positions 
of the calorimeters cells in the cluster, based on the absolute 
value cell's energy.} is considered.
The track momentum is required to be compatible with the cluster energy,
which is calibrated to the electromagnetic scale. The calibration is
derived from simulation, test-beam studies and data $Z\to ee$ 
events~\cite{elecalib}.

In order to suppress the mis-identification of other particles  
as electrons, $tight$ selection criteria based on cluster shape, 
track-cluster matching, track quality, and isolation are applied.
The shower development is narrower for electrons than for hadrons, and
the hadronic leakage\footnote{The hadronic leakage is the fraction of 
energy reconstructed in the first layer of the hadronic calorimeter}
is smaller. Track quality requirements reduce the impact of accidental 
track association with photons and energetic $\pi^0$ and $\eta$ mesons
with electromagnetic decays reconstructed as a single energy cluster.
Finally, the isolation helps rejecting electrons from heavy hadron 
semi-leptonic decays.

Overall, electron candidates are identified with \mbox{$\approx{}85\%$} 
efficiency and a contamination rate of about \mbox{$2\cdot{}10^{-5}$}.
A {\it loose} selection, where some of the requirements are relaxed, 
is also used, with the purpose of studying and modeling the contamination
in the {\it tight} candidates. The procedure to estimate this background
contribution is detailed in Sec.~\ref{sec:bckg}[QCD background].

The $\et$ of the electron is required to be larger than \mbox{$25 \GeV{}$}
and electrons in the transition region $1.37<|\eta_{\mathrm{cluster}}|<1.52$
are not considered.

\section{Muons}
\label{sec:muons}

Muons are reconstructed in the region $|\eta|<2.5$ using tracks measured in the 
Muon Spectrometer (MS) and in the ID.
The information from the two systems is used by matching
the tracks with the {\it MuId} algorithm~\cite{muidalgo} to build 
{\it combined} muon candidates.

The algorithm searches for track segments in the RPC and TGC in
$\Delta\eta\times\Delta\phi=0.4\times0.4$ regions where the trigger
fired. A single MS track is built with a least-square fitting method and
the trajectory is extrapolated back to the interaction point, taking
into account the energy loss in the calorimeter material. 
The MS track is combined with the ID track that provides the best
match, based on a $\chi^2$ test. If no track is found, no combined
muon candidate is built.
The momentum of the muon candidate is computed as a weighted average
of ID and MS measurements and calibrated using $Z\to \mu\mu$ events.
Muons are reconstructed with \mbox{$\approx{}85\%$} efficiency

The $\pt$ of the muon is required to be larger than \mbox{$25 \GeV{}$}
and the muon is required to be isolated: muons overlapping with selected
jets (see Section \ref{sec:jets}) within a 0.4 cone in $\Delta R$ are
rejected. In addition, a {\it mini-isolation}~\cite{miniisolation}
requirement is applied to reduce sensitivity to the high pile-up
conditions of $\sqrt{s} = $8~TeV collision events. It rejects muon candidates
with a large total transverse momentum of overlapping ID tracks.

\section{Jets}
\label{sec:jets}

Jets are collimated showers of particles from the hadronization of
quarks or gluons produced at the primary vertex.
The resulting stable particles leave signals both as tracks in the ID
and as energy deposits in the calorimeters.

Neighboring calorimeter cells are grouped into topological clusters
({\it topoclusters}) based on the significance of the energy deposit
in the calorimeter cells with respect to their noise
level. Cells with $|E_{\rm cell}|>4\sigma$ are considered as seeds and
all the neighboring cells with $|E_{\rm cell}| > 2\sigma$ are included
in the topocluster. Cells adjacent to the selected ones are also
included without any energy requirement.
Topoclusters are calibrated at the EM scale. For $\sqrt{s} = $8~TeV
collision events an additional correction, called Local Cluster
Weighting (LCW)~\cite{lcwcalib}, is applied. The LCW calibration
scheme classifies the clusters as {\it mainly electromagnetic} or
{\it mainly hadronic}, based on the their shape, and applies
dedicated corrections derived from simulation.

Jets are then reconstructed using the {\it anti-$k_t$}
algorithm~\cite{antiktalgo}, which combines topoclusters iteratively,
based on a distance parameter criterium. The distance parameter is
defined as:
\begin{equation}
d_{ij}=min(p_{T_i}^{-2},p_{T_j}^{-2})\frac{\Delta R_{ij}^{2}}{R^{2}},
\end{equation}
where $p_{Ti}$ is the transverse momentum of topocluster $i$, 
$\Delta R_{ij}$=$\sqrt{(\Delta\eta_{ij})^{2}+(\Delta\phi_{ij})^{2}}$ the distance 
between constituents $i$ and
$j$, $R$ a parameter of the algorithm that approximately controls the size
of the jet and is chosen to be 0.4.
The algorithm computes $d_{ij}$, the distance between two topocluster
inputs $i$ and $j$, and $d_{iB}$,  the distance between the input $i$
and the beam axis, for the whole list of topoclusters found in the
event. If $d_{iB}$ is the smallest distance, the
topocluster $i$ is considered a jet, removed from the list and the
algorithm repeats the procedure with the remaining input objects.
Otherwise, the $i$ and $j$ topoclusters corresponding to the smallest
distance $d_{ij}$ are combined and the list is updated for a new
iteration.
The procedure is repeated until the list is empty.

%The anti-$k_t$ algorithm is chosen for its good performance 
%in harsh pile-up conditions, since it consider first topoclusters 
%with higher momentum, and produces jets with a conical
%structure. 
The jet energy is computed as the sum of the energy of the cells
forming the jet.
A $\eta$-dependent correction is applied to subtract the additional
energy from in-time and out-of-time pileup based on the number of
primary vertices in the event and the average number of $pp$
interactions.
Finally a correction factor to extrapolate the particle level energy
is derived from simulation of single $pp$ interactions.

Only jets with $\pt{}>25 \GeV{}$ and $|\eta|<2.5$ are considered. In
order to ensure that the selected jets originate from the hard
scattering process, the information of tracks associated with the jets
is exploited.
It is required that the total $\pt{}$ of tracks originating from the
primary vertex be at least half of the total $\pt$ of tracks with
$\pt{}>1\GeV{}$ associated with the jet.

\subsection{$b$-tagging}
\label{sec:btag}

When a bottom quark is produced in an event, it hadronizes into a $B$
hadron, which has a lifetime of the order of $10^{-12}$~s and hence
can travel about 3~mm before decaying.
This leads to displaced secondary vertices and large impact parameters
for the decay products. Therefore the measurement and identification
of these objects in the ID allow to identify jets corresponding to
bottom quarks.

$b$-tagging is performed by the combination of three algorithms,
called {\it JetFitter}, {\it IP3D}, and {\it SV1}, which exploit
different properties to determine a probability ($b$-tag weight) for
the jet to correspond to a $b$-quark.
The IP3D algorithm estimates the $b$-tag weight by defining a
likelihood based on the impact parameters, along the beam axis and in
the transverse plane, of the tracks associated with the jet.
The SV1 algorithm reconstructs the secondary vertex and computes a
likelihood ratio to discriminate between $b$-jets and light jets using
the number of track pairs in the secondary vertex, their total
invariant mass and the fraction of momentum corresponding to the
secondary vertex.
The JetFitter algorithm performs a reconstruction of the full decay
chain of $B$ and $C$ hadrons by using a Kalman filter to determine a
common path between the primary vertex and the vertices from $b$ and
$c$ hadrons inside the jet. In addition to the observables used by the
SV1 algorithm, the significance of the significance of the
flight-length is used as discriminant.
A neural network, called {\it MV1} algorithm, is used to combine the
output of the three algorithms into a single $b$-tag weight.
The working point used to tag $b$-jets corresponds a 70\% efficiency,
while the contamination from light jets is $\sim0.8\%$, and the one
from $c$-jets $\sim20\%$.

The $b$-tagging efficiencies are measured for 
$b$, $c$ and light flavors~\cite{btagging,ctagging,ltagging} and the
simulation is calibrated with the appropriate scale factors, $\pt$ and
$\eta$-dependent.
For $b$-jets, the efficiency is derived from di-jet samples with
muons in the final states exploiting the momentum of the
muon transverse to the axis of the system formed by the muon and the jet
to distinguish muons from $b$-jets from muons from $c$- and light jets. 
The contamination from $c$-jets is measured in samples of jets with
$D^*$ mesons, where the yields of $D^*$ mesons with and without
$b$-tagging requirements are compared.
The efficiency for $b$-tagging light jets is measured in an inclusive
jet sample, using the {\it negative tag method}, where requirements on
significance of tracks impact parameter significance or of decay 
length of secondary vertices is reversed.

\subsubsection{Tag Rate Function method}
\label{sec:trf}

When requiring $\geq 1$ $b$-tagged jet the available Monte Carlo
statistics is significantly reduced for some particular background
processes, leading to large fluctuations in the predicted distributions.
To overcome this problem the Tag Rate Function (TRF) method is introduced.
Events from simulation are not rejected based on the $b$-tagging count
requirement. Events are assigned a weight corresponding to the
probability for the required number of $b$-jets to be present.
Appendix~\ref{app:trf} describes the TRF method in more detail.

\section{Missing Transverse Energy}
\label{sec:met}

At the LHC the overall momentum of the $pp$ collision is zero, as the
colliding protons have equal energies and opposite direction. 
However, for inelastic scattering, spectator partons travel down the
beam pipe without interacting with the detector. Hence, while the $z$
momentum of the colliding partons is unknown, the momentum transverse
to the beam pipe is very close to zero. The requirement of transverse
momentum conservation is therefore used to estimate the transverse
momentum of neutrinos, which do not interact with the detector, as the
missing transverse energy (\met{}).  

The \met{} is computed by combining all the topoclusters found in the
calorimeters. The energy of each topocluster is calibrated based on
the association with reconstructed objects: electrons, photons, jets
and muons. The calorimeter clusters that cannot be associated with any
object are calibrated as energy losses from showers in dead material
regions. The \met{} and its components are therefore defined as:

\begin{equation}\label{eq:met}
\begin{array}{lcl}
E^{\rm miss}_{T} & = & \big|-\sum\vec{p}_T \big| = \sqrt{(E^{\rm miss}_{x})^2 + (E^{\rm miss}_{y})^2} ,\\
E^{\rm miss}_{x} & = & -\sum\vec{p}_x ,\\
E^{\rm miss}_{y} & = & -\sum\vec{p}_y ,\\
\end{array}	\end{equation}
where the sums run over all the topoclusters in the event.