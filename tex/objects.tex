\chapter{Physics Objects}
\label{sec:objects}

The ATLAS detector records events as raw data, 
which correspond to bits of electric signal collected 
when particles interact with the detectors. 
The goal of object identification is to reconstruct the particle four-momenta
by combining the information from the different sub-detectors.
This task is performed by algorithms optimized for the reconstruction of
electrons (Sec.~\ref{sec:electrons}), muons (Sec.~\ref{sec:muons}), 
jets (Sec.~\ref{sec:jets}) and the energy imbalance left
by the passage of a neutrino (Sec.~\ref{sec:met}).

\section{Tracks}
\label{sec:tracks}

In the solenoidal magnetic field of the inner detectors,
a charged particle moves along a helicoidal trajectory with a 
curvature proportional to its momentum.
Tracks are the reconstruction of these trajectories from the electric 
signals induced in the silicon detectors by ionization.
Therefore, tracks are used to identify charged particles and measure
their momenta. In addition, the extrapolation of the trajectories
allows to identify the interaction vertices.
 
In order to reconstruct the track, the first step is to retrieve the information from
the ID hits, which are converted into three-dimensional space points. Then, 
the {\it inside-out} algorithm~\cite{insideoutalgo} iteratively builds
a track by combining space points one by one, starting from a
seed of three aligned hits in the pixel detector or in the SCT.
For each new point a Kalman filter algorithm~\cite{kalmanfilter}
checks the compatibility between the track and the new point. 
A cleaning procedure prunes the track collection, removing 
tracks sharing hits with other tracks, or reconstructed from noise hits. 
Finally, the track quality is improved by taking into account the signal
from the TRT and the effects from the interaction
of the charged particle with the detector material. 

For signals in the TRT which are not associated to any track candidate
by the inside-out reconstruction, a second algorithm,
called {\it outside-in}, is applied in order to
reconstruct tracks from secondary charged particles. 
The algorithm uses as seeds hits in the TRT
and extrapolates back to the SCT and pixel detector.


\section{Primary Vertices}
\label{sec:pv}


\section{Electrons}
\label{sec:electrons}

An electron candidate object~\cite{elereco} is selected by searching 
for a narrow, 
localized cluster of energy deposits in the EM calorimeter, 
with at least one ID track associated to it.

A {\it sliding-window} clustering algorithm is used to identify electron 
clusters. The algorithm performs a scan of the calorimeter searching 
for local maxima of energy
within a $\eta$-$\phi$ window of $0.075\times{}0.125$.
The scan is performed in the range $|\eta_{\mathrm{cluster}}|<2.47$, 
which corresponds to the ID coverage for reconstructing tracks. 
Tracks within a $\eta$-$\phi$ window of $0.05\times{}0.10$ are 
associated with the cluster.

The electron four-momentum is built from the cluster energy
and the direction of the associated ID track.
When multiple tracks are associated with the cluster, the closest
to the cluster direction\footnote{The position of a cluster 
is computed as a weighted average of the $\eta$-$\phi$ positions 
of the calorimeters cells in the cluster, based on the absolute 
value cell's energy.} is considered.
The track momentum is required to be compatible with the cluster energy,
which is calibrated to the electromagnetic scale. The calibration is
derived from simulation, test-beam studies and data $Z\to ee$ 
events~\cite{elecalib}.

In order to suppress the mis-identification of other particles  
as electrons, $tight$ selection criteria based on cluster shape, 
track-cluster matching, track quality, and isolation are applied.
The shower development is narrower for electrons than for hadrons, and
the hadronic leakage\footnote{The hadronic leakage is the fraction of 
energy reconstructed in the first layer of the hadronic calorimeter}
is smaller. Track quality requirements reduce the impact of accidental 
track association with photons and energetic $\pi^0$ and $\eta$ mesons
with electromagnetic decays reconstructed as a single energy cluster.
Finally, the isolation helps rejecting electrons from heavy hadron 
semi-leptonic decays.

Overall, electron candidates are identified with \mbox{$\approx{}70\%$} 
efficiency and a contamination rate of about \mbox{$2\cdot{}10^{-5}$}.
A {\it loose} selection, where some of the requirements are relaxed, 
is also used, with the purpose of studying and modeling the contamination
in the {\it tight} candidates. The procedure to estimate this background
contribution is detailed in Sec.~\ref{sec:bckg}[QCD background].

The $\et$ of the electron is required to be larger than \mbox{$25 \GeV$}
and electrons in the transition region $1.37<|\eta_{\mathrm{cluster}}|<1.52$
are not considered.

\section{Muons}
\label{sec:muons}

Muons are reconstructed using tracks measured in the 
Muon Spectrometer (MS) and in the ID.
The information from the two systems is used by matching
the tracks with the {\it MuId} algorithm~\cite{muidalgo} to build 
{\it combined} muon candidates.

The algorithm searches for track segments in the RPC and TGC in
$\Delta\eta\times\Delta\phi=0.4\times0.4$ regions where the trigger
fired. A single MS track is built with a least-square fitting method and
the trajectory is extrapolated back to the interaction point, taking
into account the energy loss in the calorimeter material. 



Starting from regions where interesting activity has been triggered,
track segments are searched for in the RPC and TGC and combined into a single track by means of a
least-square fitting method. These track candidates are 
subsequently extrapolated back to the interaction
point and their momentum corrected for the mip energy loss in the calorimeter material.
At this point a \chisq\ test (checking the difference between the extrapolated track coordinates weighted with
combined covariance matrix) on the matching of the candidate MS track and the tracks reconstructed in 
the ID is performed to obtain the final muon candidate track. Only ID tracks that satisfy some quality 
requirements are considered for the matching: they need to have at least two pixel hits, of which at least
one in the first layer; at least two pixel hits plus number of crossed dead pixel sensors; at least six SCT hits
plus number of  crossed dead SCT sensors; a maximum of two pixel or SCT holes\footnote{A ``hole'' in the silicon
detectors is a region where the module did not perform as expected even though the surrounding ones did.};
defining the number of TRT outliers\footnote{``Outlier'' is a hit that is deviated from the track path.} 
and the number of TRT hits as $N_{\rm TRT_{o}}$ and  $N_{\rm TRT_{h}}$ respectively, 
$N_{\rm TRT_{h}}>5$ and $N_{\rm TRT_{o}}/N_{\rm TRT_{h}}<0.9$ for $|\eta|<1.9$, 
$N_{\rm TRT_{o}}/N_{\rm TRT_{h}}<0.9$ if $N_{\rm TRT_{h}}>5$ for  $|\eta|\geq1.9$.
In case no matching is found, no \texttt{combined} 
muons are reconstructed, while if more candidates arise, the one giving the best \chisq\ is chosen.
The momentum is computed as a weighted average of ID and MS measurements.

Performance studies on muon reconstruction and identification done using 2010 data 
from $\rts=7$~\tev\ collisions and Monte Carlo $Z\to \mu\mu$ events~\cite{ATLAS-CONF-2011-063} 
have been recently updated with the data from $\rts=8$~\tev\ pp collisions~\cite{ATLAS-CONF-2013-088}.
The measured reconstruction efficiency for \texttt{combined} muons
is of about 98\% uniformely in pseudorapidity and 
is shown in Figure~\ref{fig:mueff}. 








\section{Jets}
\label{sec:jets}

\subsection{$b$-tagging}
\label{sec:btag}

\section{Missing Transverse Energy}
\label{sec:met}
