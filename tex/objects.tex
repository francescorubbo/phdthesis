\chapter{Physics Objects}
\label{sec:objects}

The ATLAS detector records events as raw data, 
which correspond to bits of electric signal collected 
when particles interact with the detectors. 
The goal of object identification is to reconstruct the particle four-momenta
by combining the information from the different sub-detectors.
This task is performed by algorithms optimized for the reconstruction of
electrons (Sec.~\ref{sec:electrons}), muons (Sec.~\ref{sec:muons}), 
jets (Sec.~\ref{sec:jets}) and the energy imbalance left
by the passage of a neutrino (Sec.~\ref{sec:met}).

\section{Tracks}
\label{sec:tracks}

In the solenoidal magnetic field of the inner detectors,
a charged particle moves along a helicoidal trajectory with a 
curvature proportional to its momentum.
Tracks are the reconstruction of these trajectories from the electric 
signals induced in the silicon detectors by ionization.
Therefore, tracks are used to identify charged particles and measure
their momenta. In addition, the extrapolation of the trajectories
allows to identify the interaction vertices.
 
In order to reconstruct the track, the first step is to retrieve the information from
the ID hits, which are converted into three-dimensional space points. Then, 
the {\it inside-out} algorithm~\cite{insideoutalgo} iteratively builds
a track by combining space points one by one, starting from a
seed of three aligned hits in the pixel detector or in the SCT.
For each new point a Kalman filter algorithm~\cite{kalmanfilter}
checks the compatibility between the track and the new point. 
A cleaning procedure prunes the track collection, removing 
tracks sharing hits with other tracks, or reconstructed from noise hits. 
Finally, the track quality is improved by taking into account the signal
from the TRT and the effects from the interaction
of the charged particle with the detector material. 

For signals in the TRT which are not associated to any track candidate
by the inside-out reconstruction, a second algorithm,
called {\it outside-in}, is applied in order to
reconstruct tracks from secondary charged particles. 
The algorithm uses as seeds hits in the TRT
and extrapolates back to the SCT and pixel detector.


\section{Primary Vertices}
\label{sec:pv}


\section{Electrons}
\label{sec:electrons}

An electron candidate object~\cite{elereco} is selected by searching 
for a narrow, 
localized cluster of energy deposits in the EM calorimeter, 
with at least one ID track associated to it.

A {\it sliding-window} clustering algorithm is used to identify electron 
clusters. The algorithm performs a scan of the calorimeter searching 
for local maxima of energy
within a $\eta$-$\phi$ window of $0.075\times{}0.125$.
The scan is performed in the range $|\eta_{\mathrm{cluster}}|<2.47$, 
which corresponds to the ID coverage for reconstructing tracks. 
Tracks within a $\eta$-$\phi$ window of $0.05\times{}0.10$ are 
associated with the cluster.

The electron four-momentum is built from the cluster energy
and the direction of the associated ID track.
When multiple tracks are associated with the cluster, the closest
to the cluster direction\footnote{The position of a cluster 
is computed as a weighted average of the $\eta$-$\phi$ positions 
of the calorimeters cells in the cluster, based on the absolute 
value cell's energy.} is considered.
The track momentum is required to be compatible with the cluster energy,
which is calibrated to the electromagnetic scale. The calibration is
derived from simulation, test-beam studies and data $Z\to ee$ 
events~\cite{elecalib}.

In order to suppress the mis-identification of other particles  
as electrons, $tight$ selection criteria based on cluster shape, 
track-cluster matching, track quality, and isolation are applied.
The shower development is narrower for electrons than for hadrons, and
the hadronic leakage\footnote{The hadronic leakage is the fraction of 
energy reconstructed in the first layer of the hadronic calorimeter}
is smaller. Track quality requirements reduce the impact of accidental 
track association with photons and energetic $\pi^0$ and $\eta$ mesons
with electromagnetic decays reconstructed as a single energy cluster.
Finally, the isolation helps rejecting electrons from heavy hadron 
semi-leptonic decays.

Overall, electron candidates are identified with \mbox{$\approx{}85\%$} 
efficiency and a contamination rate of about \mbox{$2\cdot{}10^{-5}$}.
A {\it loose} selection, where some of the requirements are relaxed, 
is also used, with the purpose of studying and modeling the contamination
in the {\it tight} candidates. The procedure to estimate this background
contribution is detailed in Sec.~\ref{sec:bckg}[QCD background].

The $\et$ of the electron is required to be larger than \mbox{$25 \GeV{}$}
and electrons in the transition region $1.37<|\eta_{\mathrm{cluster}}|<1.52$
are not considered.

\section{Muons}
\label{sec:muons}

Muons are reconstructed in the region $|\eta|<2.5$ using tracks measured in the 
Muon Spectrometer (MS) and in the ID.
The information from the two systems is used by matching
the tracks with the {\it MuId} algorithm~\cite{muidalgo} to build 
{\it combined} muon candidates.

The algorithm searches for track segments in the RPC and TGC in
$\Delta\eta\times\Delta\phi=0.4\times0.4$ regions where the trigger
fired. A single MS track is built with a least-square fitting method and
the trajectory is extrapolated back to the interaction point, taking
into account the energy loss in the calorimeter material. 
The MS track is combined with the ID track that provides the best
match, based on a $\chi^2$ test. If no track is found, no combined
muon candidate is built.
The momentum of the muon candidate is computed as a weighted average
of ID and MS measurements and calibrated using $Z\to \mu\mu$ events.
Muons are reconstructed with \mbox{$\approx{}85\%$} efficiency

The $\pt$ of the muon is required to be larger than \mbox{$25 \GeV{}$}
and the muon is required to be isolated: muons overlapping with selected
jets (see Section \ref{sec:jets}) within a 0.4 cone in $\Delta R$ are
rejected. In addition, a {\it mini-isolation}~\cite{miniisolation}
requirement is applied to reduce sensitivity to the high pile-up
conditions of $\sqrt{s} = $8~TeV collision events. It rejects muon candidates
with a large total transverse momentum of overlapping ID tracks.

\section{Jets}
\label{sec:jets}

Jets are collimated showers of particles from the hadronization of
quarks or gluons produced at the primary vertex.
The resulting stable particles leave signals both as tracks in the ID
and as energy deposits in the calorimeters.

Neighboring calorimeter cells are grouped into topological clusters
({\it topoclusters}) based on the significance of the energy deposit
in the calorimeter cells with respect to their noise
level. Cells with $|E_{\rm cell}|>4\sigma$ are considered as seeds and
all the neighboring cells with $|E_{\rm cell}| > 2\sigma$ are included
in the topocluster. Cells adjacent to the selected ones are also
included without any energy requirement.
Topoclusters are calibrated at the EM scale. For $\sqrt{s} = $8~TeV
collision events an additional correction, called Local Cluster
Weigthing (LCW)~\cite{lcwcalib}, is applied. The LCW calibration
scheme classifies the clusters as {\it mainly electromagnetic} or
{\it mainly hadronic}, based on the their shape, and applies
dedicated corrections derived from simulation.

Jets are then reconstructed using the {\it anti-$k_t$}
algorithm~\cite{antiktalgo}, which combines topoclusters iteratively,
based on a distance parameter criterium. The distance parameter is
defined as:
\begin{equation}
d_{ij}=min(p_{T_i}^{-2},p_{T_j}^{-2})\frac{\Delta R_{ij}^{2}}{R^{2}},
\end{equation}
where $p_{Ti}$ is the transverse momentum of topocluster $i$, 
$\Delta R_{ij}$=$\sqrt{(\Delta\eta_{ij})^{2}+(\Delta\phi_{ij})^{2}}$ the distance 
between constituents $i$ and
$j$, $R$ a parameter of the algorithm that approximately controls the size
of the jet and is chosen to be 0.4.
The algorithm computes $d_{ij}$, the distance between two topocluster
inputs $i$ and $j$, and $d_{iB}$,  the distance between the input $i$
and the beam axis, for the whole list of topoclusters found in the
event. If $d_{iB}$ is the smallest distance, the
topocluster $i$ is considered a jet, removed from the list and the
algorithm repeats the procedure with the remaining input objects.
Otherwise, the $i$ and $j$ topoclusters corresponding to the smallest
distance $d_{ij}$ are combined and the list is updated for a new
iteration.
The procedure is repeated until the list is empty.

The anti-$k_t$ algorithm is chosen for its good performance 
in harsh pile-up conditions, since it consider first topoclusters 
with higher momentum, and produces jets with a conical
structure.

A pile-up correction is applied

Once the jet is reconstructed, a 
pile-up correction is applied in 
order to subtract the average extra 
energy in the jet originating from pp 
interactions in the same or in other bunch crossings. 
This pile-up correction depends on the number of primary vertices in an event ($NPV$),
on the number of average interactions in 
a luminosity block ($<\mu>$), and is derived in bins of jet $\eta$ in order to
account for the different geometry of the detector. The
correction is applied to bring all jets to a 
reference point corresponding to $\mu=0$ and $NPV = 1$

Finally the jet energy is corrected to the particle
level using a orrection factor (referred to as 
{\it jet energy response} obtained from a QCD 
dijet \texttt{PYTHIA} Monte Carlo sample
which does not include
multiple pp interactions, as these have been already
corrected for.

The jets used for the calibration are isolated jets\footnote{Here an
isolated jet is a jet which does not have any other
jet with $\pt>7\gev$
around a $\Delta R$ cone of 2.5$R$, where $R$ is
the jet algorithm paramenter, i.e., for the \texttt{AKT4}, 0.4}
reconstructed in the calorimeter and matched to
a (also isolated) truth-level jet within $\Delta R<0.3$
The simulated jet energy response 
is the ratio between the energy measured in the LC jets ($E_{\rm LC}^j$)
and the truth jet energy ($E_{\rm truth}^j$). It is measured in bins of
the truth jet energy and of the calorimeter jet pseudorapidity $\eta_{\rm det}$
and a bin-by-bin fit is performed to obtain a
calibration function in terms of $E_{\rm LC}^j$
binned in $\eta_{\rm det}$.
The jet response correction is
shown in Figure~\ref{fig:corr_jet}
as a function of $\eta_{\rm det}$
for different values of the jet energy.


\subsection{$b$-tagging}
\label{sec:btag}

\section{Missing Transverse Energy}
\label{sec:met}
