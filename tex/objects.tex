\chapter{Physics Objects}
\label{sec:objects}

The ATLAS detector records events as raw data, 
which correspond to bits of electric signal collected 
when particles interact with the detectors. 
The goal of object identification is to reconstruct the particle four-momenta
by combining the information from the different sub-detectors.
This task is performed by algorithms optimized for the reconstruction of
electrons (Sec.~\ref{sec:electrons}), muons (Sec.~\ref{sec:muons}), 
jets (Sec.~\ref{sec:jets}) and the energy imbalance left
by the passage of a neutrino (Sec.~\ref{sec:met}).

\section{Tracks}
\label{sec:tracks}

\section{Primary Vertices}
\label{sec:pv}

\section{Electrons}
\label{sec:electrons}

An electron candidate object~\cite{elereco} is selected by searching 
for a narrow, 
localized cluster of energy deposits in the EM calorimeter, 
with at least one ID track\footnote{A track is the reconstruction 
of the trajectory of a charged particle, which leaves $hits$ in the 
ID layers. The tracking algorithms~\cite{trackalgos} convert the 
$hits$ into three-dimensional space points and extrapolate the 
corresponding trajectory.}
associated to it.

A $sliding-window$ clustering algorithm is used to identify electron 
clusters. The algorithm performs a scan of the calorimeter searching 
for local maxima of energy
within a $\eta$-$\phi$ window of $0.075\times{}0.125$.
The scan is performed in the range $|\eta_{\mathrm{cluster}}|<2.47$, 
which corresponds to the ID coverage for reconstructing tracks. 
Tracks within a $\eta$-$\phi$ window of $0.05\times{}0.10$ are 
associated with the cluster.

The electron four-momentum is built from the cluster energy
and the direction of the associated ID track.
When multiple tracks are associated with the cluster, the closest
to the cluster direction\footnote{The position of a cluster 
is computed as a weighted average of the $\eta$-$\phi$ positions 
of the calorimeters cells in the cluster, based on the absolute 
value cell's energy.} is considered.
The track momentum is required to be compatible with the cluster energy,
which is calibrated to the electromagnetic scale. The calibration is
derived from simulation, test-beam studies and data $Z\to ee$ 
events~\cite{elecalib}.

In order to suppress the mis-identification of other particles  
as electrons, $tight$ selection criteria based on cluster shape, 
track-cluster matching, track quality, and isolation are applied.
The shower development is narrower for electrons than for hadrons, and
the hadronic leakage\footnote{The hadronic leakage is the fraction of 
energy reconstructed in the first layer of the hadronic calorimeter}
is smaller. Track quality requirements reduce the impact of accidental 
track association with photons and energetic $\pi^0$ and $\eta$ mesons
with electromagnetic decays reconstructed as a single energy cluster.
Finally, the isolation helps rejecting electrons from heavy hadron 
semi-leptonic decays.

Overall, electron candidates are identified with \mbox{$\approx{}70\%$} 
efficiency and a contamination rate of about \mbox{$2\cdot{}10^{-5}$}.
A $loose$ selection, where some of the requirements are relaxed, 
is also used, with the purpose of studying and modeling the contamination
in the $tight$ candidates. The procedure to estimate this background
contribution is detailed in Sec.~\ref{sec:bckg}[QCD background].

The $\et$ of the electron is required to be larger than \mbox{$25 \GeV$}
and electrons in the transition region $1.37<|\eta_{\mathrm{cluster}}|<1.52$
are not considered.

\section{Muons}
\label{sec:muons}

\section{Jets}
\label{sec:jets}

\subsection{$b$-tagging}
\label{sec:btag}

\section{Missing Transverse Energy}
\label{sec:met}
