\chapter{Physics Objects}
\label{sec:objects}

The ATLAS detector records events as raw data, 
which correspond to bits of electric signal collected 
when particles interact with the detectors. 
The goal of object identification is to reconstruct the particle 4-momenta
by combining the information from the different sub-detectors.
This task is performed by algorithms optimized for the reconstruction of
electrons (Sec.~\ref{sec:electrons}), muons (Sec.~\ref{sec:muons}), 
jets (Sec.~\ref{sec:jets}) and the energy imbalance left
by the passage of a neutrino (Sec.~\ref{sec:met}).

\section{Tracks}
\label{sec:tracks}

\section{Primary Vertices}
\label{sec:pv}

\section{Electrons}
\label{sec:electrons}

An electron candidate object~\cite{elereco} is selected by searching for a narrow, 
localized cluster of energy deposits in the EM calorimeter, 
with at least one ID track\footnote{A track is the reconstruction 
of the trajectory of a charged particle, which leaves $hits$ in the 
ID layers. The tracking algorithms~\cite{trackalgos} convert the 
$hits$ into three-dimensional space points and extrapolate the 
corresponding trajectory.}
associated to it.

A $sliding-window$ clustering algorithm is used to identify electron 
clusters. The algorithm performs a scan of the calorimeter searching 
for local maxima of energy
within a $\eta$-$\phi$ window of $0.075\times{}0.125$.
The scan is performed in the range $|\eta_{\mathrm{cluster}}|<2.47$, 
which corresponds to the ID coverage for reconstructing tracks. 
Tracks within a $\eta$-$\phi$ window of $0.05\times{}0.10$ are 
associated with the cluster.

The electron four-momentum is built from the cluster energy
and the direction of the associated ID track.
When multiple tracks are associated with the cluster, the closest
to the cluster direction\footnote{The position of a cluster 
is computed as a weighted average of the $\eta$-$\phi$ positions 
of the calorimeters cells in the cluster, based on the absolute 
value cell's energy.} is considered.                        

The $\et$ of the electron is required to be larger than \mbox{$25 \GeV$}.

In order to suppress backgrounds from other particles misidentified as electrons, cut-based sets of quality criteria (menus)
provided increasing background rejection: the so-called \texttt{loose}, \texttt{medium} and \texttt{tight} menus.
In the following, the \texttt{tight} menu is described, which yields an overall efficiency of roughly 75\%.

Electron clusters tend to be smaller in size than clusters from hadrons within jets.
Hence, several observables constructed from the geometrical shape of EM clusters (shower shapes), such as their lateral width or
the energy in the highest-energetic calorimeter cells, were used in the \texttt{tight} menu.
Also, the fraction of the energy deposited in the hadronic calorimeter, which is typically very small for electrons, was exploited to\
 suppress
backgrounds from jets.
The transition radiation in the TRT was used in addition to discriminate against charged hadrons.
In order to assure that tracks are not accidentally associated to clusters, quality criteria on the number of hits in the silicon tra\
ckers,
a good geometrical matching of the track direction and the cluster position as well as of the track momentum and the cluster energy w\
ere required.


\section{Muons}
\label{sec:muons}

\section{Jets}
\label{sec:jets}

\subsection{$b$-tagging}
\label{sec:btag}

\section{Missing Transverse Energy}
\label{sec:met}
