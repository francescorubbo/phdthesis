\chapter{Physics Objects}
\label{sec:objects}

The ATLAS detector records events as raw data, 
which correspond to bits of electric signal collected 
when particles interact with the detectors. 
The goal of object identification is to reconstruct the particle
four-momenta by combining the information from the different
sub-detectors. This task is performed by algorithms optimized for the
reconstruction of electrons, muons, jets and the energy imbalance left
by the passage of neutrinos. 

\section{Tracks}
\label{sec:tracks}

In the solenoidal magnetic field of the inner detectors,
a charged particle moves along a helicoidal trajectory with a 
curvature proportional to its momentum.
Tracks are the reconstruction of these trajectories from the electric 
signals induced in the detectors by ionization.
Therefore, tracks are used to identify charged particles and measure
their momenta. In addition, the extrapolation of the trajectories
allows the identification of the interaction vertices.
 
The parameters describing a track are shown in
Fig.~\ref{fig:trackpar}: $\theta$, the angle with respect to the Z axis in the
$R$Z plane measured from the perigee\footnote{The perigee is the point
  of the track closest to the origin.}; $\phi_0$, the angle with
respect to the X axis in the XY plane measured from the perigee;
$d_0$, the impact parameter, or perigee with respect to the Z axis in
the XY plane; $z_0$, Z component of the perigee. 

\begin{figure}[htb!]
  \centering
  \includegraphics[width=0.7\textwidth]{figures/objects/tracks}
  \caption{Track parameters in the XY (left) and $R$Z (right) planes
    where the origin is the beam spot, i.e. the region in the where
    the protons collide. 
    \label{fig:trackpar}}
\end{figure}

In order to reconstruct the track, the first step is to retrieve the
information from the ID hits, which are converted into
three-dimensional space points. Then, the {\it inside-out}
algorithm~\cite{insideoutalgo}  iteratively builds a track by
combining space points one by one, starting from a seed of three
aligned hits in the pixel detector or in the SCT. 
For each new point, a Kalman filter algorithm~\cite{kalmanfilter}
checks the compatibility between the track and the new point. 
A cleaning procedure prunes the track collection, removing 
tracks sharing hits with other tracks, and tracks reconstructed from
noise hits. 
Finally, the track quality is improved by taking into account the
signal from the TRT and the effects from the interaction of the
charged particle with the detector material. 

For signals in the TRT that are not associated to any track candidate
by the inside-out reconstruction, a second algorithm,
referred to as {\it outside-in}, is applied in order to
reconstruct tracks from secondary charged particles. 
The algorithm uses as seeds hits in the TRT
and extrapolates back to the SCT and pixel detector.


\section{Primary Vertices}
\label{sec:pv}

The reconstruction of the interaction points, referred to as {\it Primary
  Vertices} (PVs), is essential to identify which one corresponds to
the hard scattering process and reconstruct the physics
objects accordingly. The reconstruction of primary vertices is divided
in two steps: first vertex candidates are identified by association
of reconstructed tracks; then, the actual vertex position is precisely
determined. The two steps are repeated iteratively~\cite{vertexalgo}.

The initial position for a PV is the maximum in the distribution of
the $z_0$ parameter of reconstructed tracks. The PV is then
estimated with an iterative $\chi^2$ fitting algorithm that
down-weighs the contribution of incompatible tracks. Tracks
incompatible with the PV by more than 7$\sigma$ are used to
reconstruct a new vertex. The procedure is repeated until there are no
tracks left to create a new vertex. Among the PV candidates found with
this procedure, the ones with less than two tracks associated are ignored. 

The hard scatter PV is assumed to be the one with the highest sum of
squared transverse momenta of the tracks. The rest of the PVs are
considered pile-up interactions. 

Vertices incompatible with the beam collision region are considered
secondary vertices.
Also referred to as displaced vertices, they typically originate from
decays of long-lived particles.
The reconstruction of secondary vertices is useful to identify
$B$-hadrons, as it will be described in Section~\ref{sec:btag}.

\section{Electrons}
\label{sec:electrons}

An electron candidate object~\cite{elereco} is selected by searching 
for a narrow, 
localized cluster of energy deposits in the EM calorimeter, 
with at least one ID track associated to it.
A {\it sliding-window} clustering algorithm is used to identify electron 
clusters. The algorithm performs a scan of the calorimeter, searching 
for local maxima of energy within a window of dimensions
$\Delta\eta\times\Delta\phi=0.075\times{}0.125$. 
The scan is performed in the range $|\eta_{\mathrm{cluster}}|<2.47$, 
which corresponds to the ID coverage for reconstructing tracks. 
Tracks within a window $\Delta\eta\times\Delta\phi=0.05\times{}0.10$
are associated with the cluster.

The electron four-momentum is built from the cluster energy
and the direction of the associated ID track.
When multiple tracks are associated with the cluster, the closest
to the cluster\footnote{The position of a cluster 
is computed as a weighted average of the $\eta$--$\phi$ positions 
of the calorimeters cells in the cluster, based on the absolute 
value cell's energy.} is considered.
The track momentum is required to be compatible with the cluster energy,
which is calibrated to the electromagnetic scale. The calibration is
derived from simulation, test-beam studies, and $Z\to ee$ data 
events~\cite{elecalib}.

In order to suppress the mis-identification of other particles  
as electrons, selection criteria based on cluster shape, 
track-cluster matching, track quality, and isolation are applied.
The shower development is narrower for electrons than for hadrons, and
the hadronic leakage\footnote{The hadronic leakage is the fraction of 
energy reconstructed in the first layer of the hadronic calorimeter}
is smaller. Track quality requirements reduce the impact of accidental 
track association with photons, energetic $\pi^0$ or $\eta$ mesons
with electromagnetic decays reconstructed as a single energy cluster.
Finally, the isolation helps rejecting electrons from semi-leptonic
decays of  heavy hadrons.
Six electron definitions are used in ATLAS in order to discriminate
real electrons from misidentified ones. Each definition applies
additional requirements with respect to the one preceding.

\texttt{Loose} electrons are defined by applying requirements of low
hadronic leakage and on the shower shape.
The identification efficiency is above $95\%$, with a jet
contamination rate of about $1/500$.

\texttt{Loose++} electrons are \texttt{loose} electrons whose track
has at least one hit in  the pixel detector and at least 7 hits in the
combined silicon detectors. The $|\Delta\eta_{\rm first EM}|$ between
the track extrapolated to the first EM layer and the matched cluster
must be lower than 0.015. The identification efficiency is similar as
for \texttt{loose} one with a ten times higher jet rejection. 

\texttt{Medium} electrons are \texttt{loose++} electrons with
additional requirements on shower shape track quality: $|d_0|<$5~mm
and $\Delta|\eta_{\rm first EM}|<0.01$. The efficiency is about 88\%
and the rejection is better than for \texttt{loose++} electrons.

\texttt{Medium++} electrons are \texttt{medium} electrons whose track
has at least one hit in the first pixel detector layer and a high
fraction of high-threshold TRT hits. In addition, $\Delta|\eta_{\rm
  first EM}|$ is required to be smaller than $0.005$, and more
stringent requirements are applied to the shower shape of clusters at
$|\eta|<2.01$. The efficiency is about 85\% and rejection is a bit
less $50\times 10^3$.

\texttt{Tight} electrons are \texttt{medium++} electrons with
additional requirements on the distance between the track and the
matched cluster ($|\Delta\phi|<0.02$, $|\Delta\eta|<0.005$). 
 An higher fraction of high-threshold TRT hits is required, as well as
 an impact parameter $|d_0|<$1~mm. 
The efficiency is about $75\%$ and the rejection is slightly better
than for \texttt{medium++}.

\texttt{Tight++} electrons are \texttt{tight} electrons with
asymmetric $\Delta\phi$ requirements, yielding better efficiency
(about $80\%$) and rejection (about $50\times 10^3$) with respect to
\texttt{tight}. This definition is used to identify electrons in this
work.

The identification efficiencies depend on the electron \pt{} and
pseudo--rapidity, while they are not strongly affected by pileup. As
shown in Fig.~\ref{fig:eleeff}, the modeling in simulation differs
slightly from what observed in data, therefore a calibration is
applied in MC samples.

\begin{figure}[htb!]\centering
  \includegraphics[width=0.45\textwidth]{figures/objects/eleid}
  \includegraphics[width=0.45\textwidth]{figures/objects/eleidvset}
  \caption{Electron identification efficiencies measured from
    $Z\to ee$ events in data and simulation as a function of the
    number of PVs (left) and ~\et{} (right).}
  \label{fig:eleeff}
\end{figure}

The \texttt{medium++} selection, where isolation and track-cluster matching
requirements are relaxed, is also used, with the purpose of studying
and modeling the contamination from QCD multijet production among the
tight candidates.
The procedure to estimate this background
contribution is detailed in Section~\ref{sec:qcdbckg}.

In order to ensure high purity, in addition to the identification
requirements, the $\et$ of the electron used in the analysis is required
to be larger than \mbox{$25 \GeV{}$}, and electrons in the transition
region $1.37<|\eta_{\mathrm{cluster}}|<1.52$ are not
considered. Electrons are also required to be isolated both at the
calorimeter level, considering the energy within $\deltaR{}{}<0.2$,
and at the track level, considering the scalar sum of \pt{} of tracks
within $\deltaR{}{}<0.3$

\section{Muons}
\label{sec:muons}

Muons are reconstructed in the region $|\eta|<2.5$ using tracks
measured in the Muon Spectrometer (MS) and in the ID.
The information from the two systems is used by matching
the tracks with the {\it MuId} algorithm~\cite{muidalgo} to build 
{\it combined} muon candidates.
The algorithm searches for track segments in the RPC and TGC in
$\Delta\eta\times\Delta\phi=0.4\times0.4$ regions where the trigger
fired. A single MS track is built with a least-square fitting method, and
the trajectory is extrapolated back to the interaction point, taking
into account the energy losses in the calorimeter material. 
The MS track is combined with the ID track that provides the best
match, based on a $\chi^2$ test. If no track is found, no combined
muon candidate is built.
The momentum of the muon candidate is computed as a weighted average
of ID and MS measurements and calibrated using $Z\to \mu\mu$ events.
Muons are reconstructed with \mbox{$\approx{}98\%$} efficiency, as shown in
Fig.~\ref{fig:mueff}.

\begin{figure}[htb!]\centering
  \includegraphics[width=0.6\textwidth]{figures/objects/muonid}
  \caption{\texttt{Combined} muon reconstruction efficiencies using
    the \texttt{Muid} algorithm measured from $Z\to \mu\mu$ events in
    data and simulation as a function of
    $\eta$~\cite{miniisolation}.} 
  \label{fig:mueff}
\end{figure}

The $\pt$ of the muon used in the analysis is required to be larger
than \mbox{$25 \GeV{}$} and the muon is required to be isolated: muons
overlapping with reconstructed jets (see Section \ref{sec:jets})
within a 0.4 cone in $\Delta R$ are rejected. In addition, a {\it
  mini--isolation}~\cite{miniisolation} requirement is applied to
reject non--isolated muons and reduce sensitivity to the high pile-up
conditions of \eighttev{} collision events. The mini--isolation is
defined as
\begin{equation}
I^{\mu}_{mini}=\sum_{tracks}p_T^{tracks}/p_T^{\mu}
\end{equation}
where $p_T^{\mu}$ is the transverse momentum of the reconstructed muon
and the sum runs over all tracks found within a \pt{}--dependent cone
radius:
\begin{equation}
\deltaR{\mu}{track}<\frac{10\GeV}{p_T^{\mu}}
\end{equation}
The mini--isolation requirement is $I^{\mu}_{mini}<0.05$, yielding
a $97\%$ efficiency for identifying hard scatter muons.

\section{Jets}
\label{sec:jets}

Jets are collimated showers of particles from the hadronization of
quarks or gluons produced in the collision.
The resulting stable particles leave tracks in the ID, if charged, and
clusters of energy deposits in the calorimeters.
Reconstructing the total energy and direction of a cluster allows an
estimation of the four momentum of the jet originating the shower.

Neighboring calorimeter cells are grouped into topological clusters
({\it topoclusters}) based on the significance of the energy deposit
in the calorimeter cells $E_{\rm cell}$ with respect to their noise
level $\sigma$. The noise level $\sigma$ is defined as the RMS of the
energy distribution measured in events triggered at random bunch
crossings. Cells with $|E_{\rm cell}|/\sigma>4$ are considered as
seeds, and all the neighboring cells with $|E_{\rm cell}|/\sigma > 2$
are included in the topocluster. Cells adjacent to the selected ones
are also included without any energy requirement.
Topoclusters are calibrated at the electromagnetic (EM) scale as the
calorimeter signals originates from the electromagnetic interaction of
particles with the detector material. For \eighttev{} collision events
an additional correction, referred to as Local Cluster Weighting
(LCW)~\cite{lcwcalib}, is applied. The LCW calibration
scheme classifies the clusters as {\it mainly electromagnetic} or
{\it mainly hadronic}, based on their shape, and applies
dedicated corrections, derived from simulation, accounting for
non--compensation, out--of--cluster energy and dead material effects. 

Jets are then reconstructed using the {\it anti-$k_t$}
algorithm~\cite{antiktalgo}, which combines topoclusters iteratively,
based on a distance parameter criterium. The distance parameter is
defined as:
\begin{equation}
d_{ij}=min(\frac{1}{p_{T_i}^2},\frac{1}{p_{T_j}^2})\frac{\Delta R_{ij}^{2}}{R^{2}},
\end{equation}
where $p_{Ti}$ is the transverse momentum of topocluster $i$, 
$\Delta R_{ij}$=$\sqrt{(\Delta\eta_{ij})^{2}+(\Delta\phi_{ij})^{2}}$ the distance 
between topoclusters $i$ and
$j$, $R$ a parameter of the algorithm that approximately controls the size
of the jet and is chosen to be 0.4.
The algorithm computes $d_{ij}$, the distance between two topocluster
inputs $i$ and $j$, and $d_{iB}=1/p_{T_i}^2$,  the distance between the input $i$
and the beam axis, for the whole list of topoclusters found in the
event. If $d_{iB}$ is the smallest distance, the
topocluster $i$ is considered a jet and removed from the list. Then the
algorithm repeats the procedure with the remaining input objects.
Otherwise, the $i$ and $j$ topoclusters corresponding to the smallest
distance $d_{ij}$ are combined, and the list is updated for a new
iteration.
The procedure is repeated until the list is empty.

The jet energy is computed as the sum of the energy of the cells
forming the jet.
A \mbox{$\eta$--dependent} correction is applied to subtract the additional
energy from in-time and out-of-time pileup, based on the number of
primary vertices in the event and the average number of $pp$
interactions. The corrected energy corresponds to the energy of the
jet as reconstructed in a scenario without pileup collisions and,
therefore, a single primary vertex in the event.
Finally, a correction factor to extrapolate the particle level energy,
referred to as {\it jet energy response}, is applied.
The jet energy response is derived from the simulation of di--jet
events from single $pp$ interactions.
The jets used for the calibration are isolated jets\footnote{A jet is
  considered isolated when no other jet with $\pt>7\GeV$
  is found within a $\Delta R$ cone of 2.5$R$, where $R=0.4$ is
  the jet radius.}
reconstructed in the calorimeter and matched to a truth-level jet
within $\Delta R<0.3$.
The jet energy response is the ratio between the energy measured in
the reconstructed jets ($E_{\rm LC}^j$) and the truth jet energy
($E_{\rm truth}^j$). The calibration is derived as a function of the
jet energy $E_{\rm LC}^j$ and pseudo--rapidity $\eta_{\rm det}$, as
shown in Fig.~\ref{fig:jetresponse}.

\begin{figure}[htb!]\centering
  \includegraphics[width=0.45\textwidth]{figures/objects/jetresponse}
  \includegraphics[width=0.45\textwidth]{figures/objects/jetresponselcw}
  \caption{Average jet response for topoclusters at EM scale (left)
    and at LCW scale (right). The response is shown separately for
    various truth-jet energies as function of the jet pseudorapidity
    $η_{\rm det}$. Also indicated are the different calorimeter regions.} 
  \label{fig:jetresponse}
\end{figure}

Only jets with $\pt{}>25 \GeV{}$ and $|\eta|<2.5$ are considered in
the analysis. In addition, jets found within $\Delta R=0.3$ of a
reconstructed electron are not considered, in order to avoid double
counting the energy deposit of the electron shower.
In order to ensure that the selected jets originate from
the hard scattering process, the information of tracks associated with
the jets is exploited.
It is required that the total $\pt{}$ of tracks originating from the
primary vertex for the signal process be at least $75\%$ of the total
$\pt$ of tracks with $\pt{}>1\GeV{}$ associated with the jet. This
requirement, referred to as {\it Jet Vertex Fraction} (JVF), is
slightly different in the analysis of the dataset at \eighttev{},
where only jets with $\pt{}<50 \GeV{}$ are required to have a
$JVF>50\%$. 

\subsection{$b$--tagging}
\label{sec:btag}

When a bottom quark is produced in an event, it hadronizes into a $B$
hadron, which has a lifetime of the order of $10^{-12}$~s and hence
can travel about 3~mm before decaying.
This long decay length leads to displaced secondary vertices and large
impact parameters for the decay products. Therefore the measurement
and identification of these objects in the ID allows to identify jets
corresponding to bottom quarks.

The identification of $b$--jets ($b$-tagging) is performed by the
combination of three algorithms, referred to as {\it JetFitter}, {\it
  IP3D}, and {\it SV1}, which exploit different properties to
determine a probability ($b$--tag weight) for the jet to originate
from the fragmentation of a $b$--quark. 
The IP3D algorithm estimates the $b$--tag weight by defining a
likelihood based on the impact parameters, along the beam axis and in
the transverse plane, of the tracks associated with the jet.
The SV1 algorithm reconstructs the secondary vertex and computes a
likelihood ratio to discriminate between $b$--jets and light
jets\footnote{Jets originating from the fragmentation of light quarks
  or gluons.} using the number of track pairs in the secondary vertex,
their total invariant mass and the fraction of momentum corresponding
to the secondary vertex.
The JetFitter algorithm performs a reconstruction of the full decay
chain of $B$ and $C$ hadrons by using a Kalman filter to determine a
common path between the primary vertex and the vertices from $b$ and
$c$ hadrons inside the jet. The significance of the flight length is
used as discriminant, in addition to the observables used by the
SV1 algorithm.
A neural network, referred to as {\it MV1} algorithm, is used to combine the
output of the three algorithms into a single $b$--tag weight.
The working point used to tag $b$--jets corresponds to a $70\%$
efficiency, as measured in \ttbar{} simulated events and shown in
Fig.~\ref{fig:btag}, while the mis--identification probability is
$\sim0.8\%$ for light jets, and the one from $c$--jets $\sim20\%$. 

\begin{figure}[htb!]\centering
  \includegraphics[width=0.45\textwidth]{figures/objects/btageff}
  \includegraphics[width=0.45\textwidth]{figures/objects/btagsf}
  \caption{Efficiency of the MV1 tagger to select $b$--, $c$--, and
    light--flavor jets, as a function of jet $|\eta|$ (right) and the
    \pt{}--dependent data--to--simulation scale factor for data at
    \eighttev{} (left).}
  \label{fig:btag}
\end{figure}

The $b$--tagging efficiencies are measured for 
$b$--, $c$--, and light--flavor jets~\cite{btagging,ctagging,ltagging} and the
simulation is calibrated with the appropriate scale factors, $\pt$ and
$\eta$--dependent (Fig.~\ref{fig:btag}).
For $b$--jets, the efficiency collision data at \seventev{} is
derived from di--jet samples with muons in the final states. 
The muon momentum component transverse to
the axis of the system formed by the muon and the jet is exploited
to distinguish muons from $b$--jets from muons from $c$-- and
light--jets. 
The large dataset at \eighttev{} allows the use of a pure
\ttbar{} sample where both $W$ bosons decay leptonically. Events with exactly
2 jets are selected, where one of the two is $b$--tagged. By
construction, the other jet originates from a bottom quark as well, and
can be used to probe the $b$--tagging efficiency.

The efficiency for $c$--jets is measured in samples of jets with
$D^*$ mesons, where the yields of $D^*$ mesons with and without
$b$--tagging requirements are compared.
The efficiency for $b$--tagging light jets is measured in an inclusive
jet sample, using the {\it negative tag method}, where a small
significance of tracks impact parameter and of decay 
length of secondary vertices is required.

\subsubsection{Tag Rate Function method}
\label{sec:trf}

When requiring $\geq 1$ $b$-tagged jet, the amount of simulated events
is significantly reduced for some particular background
processes, leading to large fluctuations in the predicted distributions.
To overcome this problem, the Tag Rate Function (TRF) method is introduced.
Events from simulation are not rejected based on the $b$-tagging count
requirement. Instead, they are assigned a weight corresponding to the
probability for the required number of $b$-jets to be present.
Appendix~\ref{app:trf} describes the TRF method in greater detail.

\section{Missing Transverse Energy}
\label{sec:met}

At the LHC the overall momentum of the $pp$ collision is zero, as the
colliding protons have equal energies and opposite direction. 
However, for inelastic scattering, spectator partons travel down the
beam pipe without interacting with the detector. Hence, while the $z$
momentum of the colliding partons is unknown, the momentum transverse
to the beam pipe is very close to zero. Since the detector provides a
near $4\pi$ coverage of the solid angle, the requirement of transverse
momentum conservation can be used to estimate the transverse
momentum of neutrinos, which do not interact with the detector, as the
missing transverse energy (\met{}).

The \met{} is computed by combining all the topoclusters found in the
calorimeters. Each topocluster is associated with reconstructed
objects: electrons, photons, jets and muons. The topoclusters are then
calibrated according to the reconstructed physics objects to which
they are associated. Topoclusters which are not associated with any
physics object are also considered and a dedicated calibration is
applied. 
The \met{} and its components are therefore defined as:

\begin{equation}\label{eq:met}
\begin{array}{lcl}
E^{\rm miss}_{T} & = & \big|-\sum\vec{p}_T \big| = \sqrt{(E^{\rm miss}_{x})^2 + (E^{\rm miss}_{y})^2} ,\\
E^{\rm miss}_{x} & = & -\sum\vec{p}_x ,\\
E^{\rm miss}_{y} & = & -\sum\vec{p}_y ,\\
\end{array}	\end{equation}
where the sums run over all the topoclusters in the event.
