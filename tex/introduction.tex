\chapter{Introduction}
\label{sec:introduction}

Asymmetry is the ``lack of equality or equivalence between parts or
aspects of something''~\cite{oxforddict}.
Symmetries are powerful tools in particle physics, because it has
become evident that practically all laws of nature originates in
symmetry. Therefore, violations of symmetry present theoretical
puzzles leading to a deeper understanding of nature. From the
experimental point of view, the measurement of asymmetries is
virtually free from systematic uncertainties, and allow to distinguish
small asymmetric signals from large symmetric backgrounds.
In the recent history of particle physics, asymmetries have been
exploited to investigate the nature of the weak interaction. 
Due to the parity violation typical of the weak
interaction, processes show differences between particles and
antiparticles, or between left-handed or right-handed particles.
Various asymmetries have been studied at collider experiments in the
past thirty years:

\begin{itemize}
\item A {\it left-right asymmetry} is defined as $A_{LR}=(N_L -
  N_R)/(N_L +N_R)$, where $N_L$ is the number of events with a
  left-polarized particle, and $N_R$ the number of right-polarized
  events. The left-right asymmetry in $Z$ boson production was
  measured in 1992 at the Stanford Linear Collider of polarized and
  unpolarized electron beams\cite{Elia:1993ka}.
\item A {\it charge asymmetry}  is defined as $A_{+-}=(N_+ -
  N_-)/(N_+ +N_-)$, where $N_+$ is the number of events with a
  positively charged particle, and $N_-$ the number of events with a
  negative particle. Charge asymmetries are used to measure CP
  violation in $B$ meson decays.
\item A {\it forward-backward asymmetry} is defined as $A_{FB}=(N_F -
  N_B)/(N_F +N_B)$, where $N_F$ is the number of events with a
  particle moving {\it forward} with respect to a chosen reference
  direction, and $N_B$ the number of {\it backward} events.
\end{itemize}

The latter is the type of asymmetry more relevant for this
dissertation. A famous example of forward-backward asymmetry is the
measurement in muon pair production in $e^+e^-$ annihilation by the
TASSO collaboration~\cite{Braunschweig:1988fy}. The measurement,
performed at center-of-mass energies between $35$ and $46.8\GeV{}$,
showed how the asymmetry provides sensitivity to probe for physics at
much higher energy, beyond reach for direct observation. In the
specific case, Fig.~\ref{fig:tasso} shows that, while the inclusive
cross section measurements agree with the symmetric $QED$ description,
a forward--backward asymmetry is induced by the interference with the
$Z$ boson production at $\sqrt{s}\approx 91\GeV{}$.

\begin{figure}[!htb]\centering
  \includegraphics[width=0.495\textwidth]{figures/introduction/tasso_xs}
  \includegraphics[width=0.495\textwidth]{figures/introduction/tasso_diffxs}
  \caption{Inclusive cross section as a function of center-of-mass
    energy (left) and angular distribution of the muons (right) for
    muon pair production at $e^+e^-$ collisions~\cite{Braunschweig:1988fy}.}
  \label{fig:tasso}
\end{figure}

An analogous asymmetry in top quark pair (\ttbar{}) production might provide a
probe for new physics processes at scales beyond the current reach of direct
searches. The CDF and \dzero{} collaborations reported an unexpectedly
large forward--backward asymmetry in \ttbar{} production at \ppbar{}
collisions at Tevatron, Fermilab~\cite{Abazov:2014cca,}. As shown in
Fig.~\ref{fig:afbtev}, the asymmetry has a dependence on the \ttbar{}
invariant mass (i.e. the center-of-mass energy) with a interesting
hint of discrepancy with respect to the Standard Model prediction.
\begin{figure}[!htb]\centering
  \includegraphics[width=0.495\textwidth]{figures/introduction/afb_tevatron}
  \caption{Dependence of the forward--backward asymmetry on \ttbar{}
    invariant mass for Tevatron experiments \dzero{} and CDF, compared
    with the Standard Model prediction. The horizontal error bars
    indicate the binning used in each experiment.~\cite{Aguilar-Saavedra:2014kpa}}
  \label{fig:afbtev}
\end{figure}
With the end of Tevatron operations in 2011, no additional data will
be collected, therefore little room is left for improving the
precision of the forward--backward asymmetry measurement at \ppbar{}
collisions, with the goal of clarifying the experimental status.
However, complementary probes of the asymmetry in \ttbar{} production
can be studied at $pp$ collisions, provided by the Large Hadron
Collider at CERN.

This thesis work focus on the study of a central--forward asymmetry in
\ttbar{} production at $pp$ collisions, also referred to as {\it
  charge} asymmetry, and defined in Sec.~\ref{sec:topca}. This
document is organized as follows: Chapter~\ref{sec:theory} gives an
introduction to the Standard Model of particle physics, with a focus
on the top quark phenomenology and the production
asymmetry. Chapter~\ref{sec:experiment} describes the experimental
setup of the ATLAS detector within the Large Hadron Collider
accelerator facility. The simulation of the physics processed used in
the analysis is described in Chapter~\ref{sec:simulation}, while the
definition of the physics objects follows in
Chapter~\ref{sec:objects}. Chapter~\ref{sec:strategy} illustrates the
analysis strategy followed by the description of the event selection
and the background determination. The reconstruction of the \ttbar
system kinematics is described in Chapter~\ref{sec:reconstruction},
while Chapter~\ref{sec:unfolding} describes the unfolding procedure
used to account for acceptance and resolution effects.
The results are then summarized in Chapter~\ref{sec:results}, and a
summary of this thesis is given in Chapter~\ref{sec:conclusion}.