\chapter{Introduction}
\label{sec:introduction}

Asymmetry is the ``lack of equality or equivalence between parts or
aspects of something''~\cite{oxforddict}.
Symmetries are powerful tools in particle physics because it has
become evident that practically all laws of nature originate in
symmetry. Therefore, violations of symmetry present theoretical
puzzles whose resolution leads to a deeper understanding of Nature. 
From the experimental point of view, the measurement of asymmetries is
virtually free from systematic uncertainties and allow to distinguish
small asymmetric signals from large symmetric backgrounds.
In the recent history of particle physics, asymmetries have been
exploited to investigate the nature of the weak interaction. 
Due to the parity violation typical of the weak
interaction, processes show differences between particles and
antiparticles, or between left--handed and right--handed particles.
Various asymmetries have been studied at collider experiments in the
last thirty years:

\begin{itemize}
\item A {\it charge asymmetry}  is defined as $A_{+-}=(N_+ -
  N_-)/(N_+ +N_-)$, where $N_+$ is the number of events with a
  positively--charged particle, and $N_-$ the number of events with a
  negatively--charged particle. Charge asymmetries are used to measure CP
  violation in meson decays~\cite{LHCb:2012fb}.
\item A {\it left-right asymmetry} is defined as $A_{LR}=(N_L -
  N_R)/(N_L +N_R)$, where $N_L$ and $N_R$ are the number of events with a
  left--polarized and a right--polarized particle, respectively. 
  The left-right asymmetry in $Z$ boson production was
  measured in 1992 at the Stanford Linear Collider using polarized and
  unpolarized electron beams~\cite{Elia:1993ka}.
\item A {\it forward-backward asymmetry} is defined as $A_{FB}=(N_F -
  N_B)/(N_F +N_B)$, where $N_F$ is the number of events with a
  particle moving {\it forward} with respect to a chosen reference
  direction, and $N_B$ the number of {\it backward} events. A notable
  example is the forward--backward asymmetry in $\bbbar$ production
  measured at LEP~\cite{ALEPH:2005ab}.
\end{itemize}

The latter is the type of asymmetry more relevant for this
dissertation. A famous example of forward-backward asymmetry is the
measurement in $\mu^+\mu^-$ production in $e^+e^-$ annihilation by the
TASSO Collaboration~\cite{Braunschweig:1988fy}. The measurement,
performed at center-of-mass energies between $35$ and $46.8\GeV{}$,
showed how the asymmetry allows probing for new physics contributing at
energy scales beyond direct reach. In this case, Fig.~\ref{fig:tasso}
shows that, while the inclusive cross section measurements of
$e^+e^-\to\mu^+\mu^-$ agree with the symmetric QED description, a
forward--backward asymmetry is induced by the interference with a
virtual $Z$ boson discovered at a mass of $\sim 91\GeV{}$ in
proton--antiproton collisions by the UA1 and UA2
Collaborations~\cite{Arnison:1983rp,Banner:1983jy}. 

\begin{figure}[!htb]\centering
  \includegraphics[width=0.495\textwidth]{figures/introduction/tasso_xs}
  \includegraphics[width=0.495\textwidth]{figures/introduction/tasso_diffxs}
  \caption{Inclusive cross section as a function of center-of-mass
    energy (left) and angular distribution of the muons (right) for
    $e^+e^-\to\mu^+\mu^-$ production, as measured by the TASSO
    Collaboration~\cite{Braunschweig:1988fy}. The measured inclusive cross
    section agrees with the QED prediction, shown with a solid line,
    while the best fit of the angular distribution deviates from the
    QED prediction, shown with a dashed line.} 
  \label{fig:tasso}
\end{figure}

An analogous asymmetry in top quark pair (\ttbar{}) production might also provide a
probe for new physics processes at scales beyond the current reach of direct
searches. The CDF and \dzero{} Collaborations reported an unexpectedly
large forward--backward asymmetry in \ttbar{} production using
proton--antiproton collisions at $\sqrt{s}=1.96\TeV{}$ at Fermilab's
Tevatron collider~\cite{Abazov:2014cca,Aaltonen:2012it}. As shown in
Fig.~\ref{fig:afbtev}, the asymmetry has a dependence on the \ttbar{}
invariant mass (a proxy for the partonic center--of--mass energy) with
an interesting hint of discrepancy with respect to the Standard Model (SM)
prediction.
\begin{figure}[!htb]\centering
  \includegraphics[width=0.495\textwidth]{figures/introduction/afb_tevatron}
  \caption{Dependence of the forward--backward asymmetry measurements
    by the CDF and D0 Collaborations on the invariant mass of the
    \ttbar{} system. The horizontal error bars indicate the binning
    used in each experiment. The horizontal lines show the SM
    prediction from simulation (solid line) and the theoretical
    calculation (dashed line)~\cite{Bernreuther:2012sx}.}
  \label{fig:afbtev}
\end{figure}
With the end of Tevatron operations in 2011, no additional data will
be collected. Therefore little room is left for improving the
precision of the forward--backward asymmetry measurement in \ppbar{}
collisions, with the goal of clarifying the experimental status.
However, complementary probes of the asymmetry in \ttbar{} production
can be studied in $pp$ collisions, provided by the Large Hadron
Collider (LHC) at CERN.

This thesis' work focus on the study of a central--forward asymmetry in
\ttbar{} production, also referred to as {\it
  charge} asymmetry, using $pp$ collisions data collected with the
ATLAS detector at the LHC. This document is organized as follows:
Chapter~\ref{sec:theory} gives an introduction to the Standard Model
of particle physics, with a focus on the top quark phenomenology and
the production asymmetry. Chapter~\ref{sec:experiment} describes the
experimental setup of the ATLAS detector within the LHC accelerator
facility. 
The simulation of the physics processed
used in the analysis is described in Chapter~\ref{sec:simulation},
while the definition of the physics objects follows in
Chapter~\ref{sec:objects}. Chapter~\ref{sec:strategy} illustrates the
analysis strategy followed by the description of the event selection
and the background determination. The reconstruction of the \ttbar{}
system kinematics is described in Chapter~\ref{sec:reconstruction},
while Chapter~\ref{sec:unfolding} describes the unfolding procedure
used to account for acceptance and resolution effects.
The results of the inclusive and differential measurements of the
\ttbar{} charge asymmetry using the full LHC Run 1 dataset collected
with the ATLAS detector are summarized in Chapter~\ref{sec:results},
along with a comparison to other existing measurements at the LHC.
Finally the summary of this thesis and an outlook is given in
Chapter~\ref{sec:conclusion}.

\subsubsection*{Personal contributions and acknowledgement}

The results presented in this dissertation represent a small subset of
the research carried out by the ATLAS Collaboration, comprised of over
3000 scientists from 38 countries. These researchers are engaged in a
variety of tasks, ensuring the successful operation of the experiment and
the subsequent data analysis. The author actively contributed to the
data--taking during 2011 and 2012 by developing and maintaining the
jet trigger menu within the ATLAS Jet Trigger group. 
The author's main focus, however, was the analyses illustrated in this
dissertation, performed within the ATLAS Top group. Specifically, he
designed and implemented the strategy and performed the necessary
validation and optimization studies.
A special acknowledgement goes to the former and present members of
the IFAE--Top group for their fundamental contributions to the common
analysis framework used for these analyses.
The results based on data collected during 2011 have been published in
the Journal of High Energy Physics~\cite{Aad:2013cea}, while the
publication of the results based on the 2012 dataset is expected by
the end of 2014.
The author also studied the phenomenology of the \ttbar{} charge
asymmetry in collaboration with theorists J.A. Aguilar--Saavedra and
E. \'Alvarez, resulting in two publications in Physics Letters
B~\cite{AguilarSaavedra:2011cp} and the Journal of High Energy
Physics~\cite{Aguilar-Saavedra:2014vta}.

