To overcome the loss of simulated events due to the $b$--tag
requirement, the tag rate function (TRF) method is introduced. By
using the TRF method, no event is discarded based on its b--tagging
count, but instead all the events are weighted.  This weight can be
interpreted as the probability of the given event to contain the
desired number of $b$--jets.  The tagging efficiency, parametrized as a
function of $\eta$, $\pt$ and true jet flavor, is used to compute
the event weight.

Given a jet with $\eta$, $\pt$ and flavor $f$, its tagging
probability can be expressed as:
\begin{equation*}
	\varepsilon \left(f,|\eta|,\pt\right)
\end{equation*}

For a given event with $N$ jets, its probability of containing exactly
one b-tag jet is computed as:
\begin{equation*}
	P_{=1} = \sum\limits_{i=1}^N \left( \varepsilon_{i}
    \prod\limits_{i \neq j} \left( 1 - \varepsilon_{j} \right) \right)
\end{equation*}

In the same way, it can be used to compute the probability for
inclusive b-tag selections:
\begin{align*}
	P_{=0} &= \prod\limits_{i=1}^N \left( 1 - \varepsilon_{j} \right) \\
	P_{\geq 1} &= 1 - P_{=0}
\end{align*}

\section{Validation}
This method relies on the correct calibration of the tagging
efficiency in MC samples.  However, closure tests performed with the
official calibration files have shown that the efficiency
parametrization is not as accurate as expected.  Assuming a correct
calibration, the average of the histogram of $1/\varepsilon$ vs
$\eta$, $\pt$ and true jet flavour should be flat and with mean equal
to one.  Figure \ref{fig:app:trfClosure} shows the result of this test.

As it can be observed, there is a departure from the expected behavior
in the edges of the distribution that amounts to up to 20\% in some
regions of the light and c flavour.  In the analysis at \seventev{}, an
efficiency map has been derived, using the \mcatnlo{} sample, in order
to overcome this problem. In the analysis at \eighttev{}, the
distributions are normalized to the number of events predicted by
applying the actual tagging requirement.

Figure \ref{fig:app:trfClosure} shows the result of the closure test
on the derived efficiency map.  The derived efficiency map will
therefore be used for the probability computations in the TRF method.
%
\begin{figure}\centering
    \includegraphics[width=0.45\textwidth]{figures/appendix/trf/closure_test_official_b} 
    \includegraphics[width=0.45\textwidth]{figures/appendix/trf/closure_test_myeff_b}
    \includegraphics[width=0.45\textwidth]{figures/appendix/trf/closure_test_official_c}
    \includegraphics[width=0.45\textwidth]{figures/appendix/trf/closure_test_myeff_c}
    \includegraphics[width=0.45\textwidth]{figures/appendix/trf/closure_test_official_light} 
    \includegraphics[width=0.45\textwidth]{figures/appendix/trf/closure_test_myeff_light}  
  \caption{Results of the closure test using efficiency from the
    official calibration file (left column) and the private efficiency
    map (right column). The test is split in the different jet
    flavours: $b$--jets (top), $c$--jets (middle) and light jets (bottom)
  }
  \label{fig:app:trfClosure}
\end{figure}

The validation of the method is further done by comparing in \alpgen{} the
normalisation and shape of the relevant distributions shown in
Fig.~\ref{fig:app:trfSpectra}.  As it is seen in the plots, the
prediction of the TRF method is accurate up to the statistical
error.
%
\begin{figure}\centering
\includegraphics[width=0.45\textwidth]{figures/appendix/trf/ttbar_muhpresel_jet_pT_14jetin252525251btagin}
\includegraphics[width=0.45\textwidth]{figures/appendix/trf/ttbar_muhpresel_jet_eta_14jetin252525251btagin}
\includegraphics[width=0.45\textwidth]{figures/appendix/trf/ttbar_muhpresel_missingET_missET4jetin252525251btagin}
\includegraphics[width=0.45\textwidth]{figures/appendix/trf/ttbar_muhmuon_Wlep_MassT4jetin252525251btagin}
  \caption{
    Comparison between the TRF prediction and the cut--based $b$--tag
    prediction in \alpgen{}.  The plots show, for the \ttbar{} \mujets{} sample,
    the \pt{} (top left) and $\eta$ (top right) distributions for the
    highest--\pt{} jet, the \met{} (bottom left) and the \mtw{}
    (bottom right) distributions.}
    \label{fig:app:trfSpectra}
\end{figure}

