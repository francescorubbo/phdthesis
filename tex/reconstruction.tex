\chapter{Kinematic Reconstruction}
\label{sec:reconstruction}

In order to access the kinematic quantities of the \ttbar{} system, an
event--by--event estimation of the four momenta of \ttbar{}
pair is required. This section describes the kinematic fit performed
to associate the reconstructed objects to the final state particles of
the semi--leptonic \ttbar{} decay and obtain the four momenta of the
top quark and antiquark.

\section{Kinematic fit}

A maximum likelihood approach~\cite{klfitter} is used to determine the
four-momenta of the top and of the antitop quarks from the four-momenta of
the reconstructed decay products: at least four reconstructed jets, one lepton, and the
missing transverse momentum. The kinematic likelihood fit finds the
most likely association between reconstructed objects and \ttbar{} final state particles:
two $b$ quarks, two light quarks, one lepton and one neutrino.
The free parameters of the likelihood are the parton energies, the
lepton transverse momentum and the three neutrino momentum components.
The likelihood is defined as
\begin{eqnarray}
L & = & \mathcal{B}(\widetilde{E}_{\rm{p,}1}, \widetilde{E}_{\rm{p,}2} | m_W,  \Gamma_W)
      \cdot \mathcal{B}(\widetilde{E}_{\rm l}, \widetilde{E}_{\nu} | m_W, \Gamma_W) \cdot \nonumber \\
           &   & \mathcal{B}(\widetilde{E}_{\rm{p,}1}, \widetilde{E}_{\rm{p,}2},     \widetilde{E}_{\rm{p,}3} | m_t, \Gamma_t)
                 \cdot \mathcal{B}(\widetilde{E}_{\rm l}, \widetilde{E}_{\nu},    \widetilde{E}_{\rm{p,}4} | m_t, \Gamma_t) \cdot \nonumber \\
           &   & \mathcal{W}( \hat{E}_{x}^{miss}| \widetilde{p}_{x, \nu})
                 \cdot \mathcal{W}(\hat{E}_{y}^{miss} |    \widetilde{p}_{y, \nu})
                 \cdot \mathcal{W}(\hat{E}_{\rm{lep}} | \widetilde{E}_{\rm{lep}}) \cdot \nonumber \\
           &   & \prod_{\rm{i=1}}^4 \mathcal{W}(\hat{E}_{\rm{jet,}i} | \widetilde{E}_{\rm{p,}i})
                 \cdot P(\textrm{$b$ tag} ~| ~\textrm{quark flavor}),
\label{eq:klflikelihood}
\end{eqnarray}
where:
\begin{itemize}
\item the $\widetilde{\square{}}$ are the partonic quantities to be estimated
       and $\hat{\square{}}$ are the measured values;
\item $\mathcal{B}(\widetilde{E}|m,\Gamma)\propto((E^2-m^2)^2+m^2\Gamma^2)^{-1}$ is
  the Breit-Wigner parametrization of the parton energies
  $\widetilde{E}_{\rm{p,}i}$, lepton energy $\widetilde{E}_{\rm{lep}}$
  and neutrino energy $\widetilde{E}_{\nu}$, in the $W$ boson and top quark decays.
\item $\mathcal{W}$ is the transfer function mapping the measured energy
      of a reconstructed object to the energy of the corresponding final state particle.
\item The mass and width of the $W$ boson and of the top quark are set
  to their measured values $m_W = 80.4~\GeV{}$, $\Gamma_W = 2.1~\GeV{}$, $m_t =
       172.5~\GeV{}$, and $\Gamma_t = 1.5~\GeV{}$. 
\item The term $P(b-\textrm{tag} ~| ~\textrm{quark flavor})$ is used
  to weight the jet--parton association based on $b$--tagging
  information, and it is described in the following.
\end{itemize}

The likelihood is evaluated for all permutations of four reconstructed
jets assigned to the four partons.
In events where five or more jets were reconstructed, only
the five highest-\pt{} jets are considered. The most probable
combination, out of all the possible four-jets permutations, is
chosen. The role of the probability term $P(b-\textrm{tag} ~|
~\textrm{quark flavor})$ is to favor the association of $b$--jets with
the $b$ quarks from the top quark decays. The probability is defined,
for a given permutation of four jets, as the product of $b$-tag
efficiencies $\epsilon_b$ or mis--tag rates $\epsilon_{light}$ of each
jet $j$, based on the $b$--tagging decision and the flavor of the
associated quark:
\begin{equation}
P(b-\textrm{tag}~|~\textrm{quark flavor}) = \prod_{j}
\begin{cases}
\epsilon_b~\text{if tagged jet associated with $b$ quark}\\
(1-\epsilon_b)~\text{if non-tagged jet associated with $b$ quark} \\
\epsilon_{light}~\text{if non-tagged jet associated with quark from
  $W$ boson} \\
(1-\epsilon_{light})~\text{if tagged jet associated with quark from
  $W$ boson} \\
\end{cases}
\end{equation} 
The average values for $\epsilon_b$ and $\epsilon_{light}$ are $70\%$
and $0.8\%$ respectively, even though they mildly depend on
jet \pt{} and pseudo--rapidity. 

The transfer functions $\mathcal{W}(E_{truth}|E_{reco})$ describe the probability that the
reconstructed energy $E_{reco}$ corresponds to a true value
$E_{truth}$.
The transfer functions are parametrized as double gaussians
\begin{equation}
  \label{eq:doublegaus}
  \mathcal{W}(E_\mathrm{truth}|E_\mathrm{reco}) = \frac{1}{\sqrt{2\pi}(p_2+p_3p_5)}\Bigg[ \exp\Bigg(-\frac{(\Delta E - p_1)^2}{2p_2^2}\Bigg) + p_3\exp\Bigg(-\frac{(\Delta E - p_4)^2}\
  {2p_5^2}\Bigg) \Bigg]\phantom{,}
\end{equation}
where $\Delta E = (E_{truth}-E_{reco})/E_{truth}$, and the $p_i$
free parameters are estimated in different regions of $E_{truth}$ and
$\eta$ for each particle. Figure~\ref{fig:transferfunctions} show, as
example, the transfer functions for $b$ quarks and for electrons.  

\begin{figure}[!htb]\centering
  \includegraphics[width=0.495\textwidth]{figures/reconstruction/btf}
  \includegraphics[width=0.495\textwidth]{figures/reconstruction/eletf}
  \caption[Energy evolution of transfer functions]{
    Energy evolution of the transfer functions for electrons in the
    range $0.8<\eta<1.37$ (left) and for $b$ quarks in the range
    $1.37<\eta<1.52$ (right). 
    \label{fig:transferfunctions}
    }
\end{figure}

\section{Performance}

Due to selection, resolution and combinatorics effects, the efficiency
of reconstructing a \ttbar{} pair matching\footnote{A reconstructed
  object $i$ is considered matched with a parton $j$ if the condition
  $\deltaR{i}{j}<0.2$ is satisfied.} the original partons is
$25\%$. The dominant cause of mis--reconstruction is the wrong jet
assignment to the corresponding \ttbar{} final state parton. This
typically occurs in three cases:
\begin{itemize}
\item {\it jets outside acceptance}: the jet originating from a given
  parton does not satisfy the reconstruction requirements, such as
  $\pt{}>25 \GeV$ or pseudo--rapidity within the $|\eta|<2.5$ range.
  The fraction of events where all \ttbar{} final state jets are
  reconstructed is $25-40\%$, depending on the $b$--jets requirement.
\item {\it jets outside selection}: not all four jets corresponding to
  the \ttbar{} decay are used in the reconstruction algorithm.
  This occurs approximately $50\%$ of the times for events with more
  than five jets and all matched jets within acceptance.
\item {\it jets mis-assigned}: one or more jets are not assigned to
  the correct parton. It occurs about $15-25\%$ of the times for
  events where the correct jets where used in the reconstruction
  algorithm, depending on the $b$--jet requirement.
\end{itemize}

The impact of mis--reconstruction is reflected on the resolution of
the kinematic quantities of the reconstructed \ttbar. The plots
in Fig.~\ref{fig:reso} show the two dimensional distributions of the
reconstructed quantities in relation to their true values. The
resolution on \mtt{} and \pttt{} improves at higher values, while the
top rapidity shows a small spread over the whole range.

\begin{figure}[!htb]\centering
  \includegraphics[width=0.495\textwidth]{figures/reconstruction/mtt}
  \includegraphics[width=0.495\textwidth]{figures/reconstruction/pttt}
  \includegraphics[width=0.495\textwidth]{figures/reconstruction/betatt}
  \includegraphics[width=0.495\textwidth]{figures/reconstruction/topy}
  \caption{
    \label{fig:reso}
    Distributions of kinematic quantities of the
    reconstructed \ttbar{} pair in relation to their true values.
  }
\end{figure} 

For what concerns the asymmetry measurement, the relevant benchmark of
the reconstruction performance is the efficiency of reconstructing
correctly the \dy{} sign, defined as
\begin{equation}
  \label{eq:dysign}
  \epsilon = \frac{N(\dy{}_{truth}\cdot\dy{}_{reco}>0)}{N_{tot}}\phantom{.}
\end{equation}
Events where the \dy{} sign is measured incorrectly can occur when the top and
antitop quarks are not reconstructed from the correct final
state particles (mis-reconstruction). Even when the reconstruction
algorithm associates the correct particles, small \dy{} values can be
reconstructed with the wrong sign due to resolution effects.
As a result the asymmetry of the distribution of reconstructed \dy{}
is diluted by a factor 
\begin{equation}
  \label{eq:dilution}
D = 2\times{}\epsilon - 1\phantom{.}
\end{equation}  

The efficiency $\epsilon$, measured in the \ttbar{} simulated sample,
is $\approx75\%$ corresponding to a dilution factor
$D=2\times0.75-1=0.5$.
The efficiency $\epsilon$ depends on the kinematic region
considered. As shown in Fig.~\ref{fig:dysign}, the \dy{} reconstruction performs
better at high \mtt{}. 

\begin{figure}[!htb]\centering
  \includegraphics[width=0.495\textwidth]{figures/reconstruction/signdy_eff}
  \caption{
    \label{fig:dysign}
    Probability to reconstruct the correct \dy{} sign as a funcion of
    the \ttbar{} invariant mass \mtt{} for the \mujets{} sample.
  }
\end{figure}

%Table~\ref{tab:actruthreco} shows the evolution of the asymmetry
%values due to selection and reconstruction effects. 
%
%\begin{table}[!htb]\centering
%  \begin{tabular}{ l c c c}
%    \toprule
%    sample      & true \ac{} & true \ac{} after selection & reconstructed \ac{}\\
%    \midrule
%    \powheg{} &                &     &    \\
%    \mcatnlo{} &                &     &    \\
%    \bottomrule
%  \end{tabular}
%  \caption{\ac{} values in simulated \ttbar{} samples at truth and
%    reconstructed level}
%  \label{tab:actruthreco}
%\end{table}

\section{Comparison between data and prediction}
\label{sec:datamcreco}

The modeling of the reconstructed quantities is validated by comparing
the distributions as observed in data with the prediction. 
Figure~\ref{fig:datamcreco2011} 
%and ~\ref{fig:datamcreco2012}
shows the comparison of the distributions of
the top quark pair invariant mass \mtt{}, transverse momentum \pttt{},
rapidity \ytt{}, and velocity along the beam axis \betatt{} for the
2011 dataset at $\sqrt{s} = 7 \TeV{}$.
                                   
\begin{figure}[!htb]\centering
  \includegraphics[width=0.495\textwidth]{figures/reconstruction/7TeV/dy}
  \includegraphics[width=0.495\textwidth]{figures/reconstruction/7TeV/mtt}
  \includegraphics[width=0.495\textwidth]{figures/reconstruction/7TeV/pttt}
  \includegraphics[width=0.495\textwidth]{figures/reconstruction/7TeV/ytt}
  \includegraphics[width=0.495\textwidth]{figures/reconstruction/7TeV/betatt}
  \caption[Control plots for reconstructed quantities at $\sqrt{s} = 7
  \TeV{}$]{Control plots for \dy{} (top left), invariant mass \mtt{} 
    (top right), transverse momentum \pttt{} (centre left), rapidity
    \ytt{} (centre right) and velocity \betatt{} (bottom)
    distributions for the \ejets{} and \mujets{} channels combined in the
    tagged sample at $\sqrt{s} = 7 \TeV{}$. The uncertainty on the
    total prediction includes both the statistical and the systematic
    components.} 
  \label{fig:datamcreco2011}
\end{figure}

%\begin{figure}[!htb]\centering
%  \includegraphics[width=0.495\textwidth]{figures/reconstruction/7TeV/dy}
%  \includegraphics[width=0.495\textwidth]{figures/reconstruction/7TeV/mtt}
%  \includegraphics[width=0.495\textwidth]{figures/reconstruction/7TeV/pttt}
%  \includegraphics[width=0.495\textwidth]{figures/reconstruction/7TeV/ytt}
%  \includegraphics[width=0.495\textwidth]{figures/reconstruction/7TeV/betatt}
%  \caption[Control plots for reconstructed quantities at $\sqrt{s} = 8
%  \TeV{}$]{Control plots for \dy{} (top left), invariant mass \mtt{} 
%    (top right), transverse momentum \pttt{} (centre left), rapidity
%    \ytt{} (centre right) and velocity \betatt{} (bottom)
%    distributions for the \ejets{} and \mujets{} channels combined in the
%    tagged sample at $\sqrt{s} = 8 \TeV{}$. The uncertainty on the
%    total prediction includes both the statistical and the systematic
%    components. [TO BE UPDATED]} 
%  \label{fig:datamcreco2012}
%\end{figure}

\subsection{Likelihood requirement}
\label{sec:Reconstruction:lhood}

For the analysis of the 2011 dataset collected at $\sqrt{s} = 8
\TeV{}$, a requirement on reconstruction quality is applied to improve
the resolution on reconstructed quantities.
A 20\% reduction on the \ac{} uncertainty in the highest mass bin ($\mtt{}>750\GeV{}$)
is achieved by requiring the logarithm of the likelihood ({\it log-likelihood}) to be $>-55$,
as shown in Figure~\ref{fig:errvsllcut}.

\begin{figure}[!htb]\centering
  \includegraphics[width=0.495\textwidth]{figures/reconstruction/asymm5errVsLLcut}
  \caption{
    \label{fig:errvsllcut}
    Scan of the statistical uncertainty on \ac{} for $\mtt{}>750\GeV{}$
    with different requirement on log-likelihood.
  }
\end{figure}

The modeling of the log-likelihood is validated by comparing its
distribution in data with the prediction, as shown in Figure~\ref{fig:logl}.

\begin{figure}[!htb]\centering
  \includegraphics[width=0.495\textwidth]{figures/reconstruction/7TeV/ejets/hll_datamc_1taginTRF}
  \includegraphics[width=0.495\textwidth]{figures/reconstruction/7TeV/mujets/hll_datamc_1taginTRF}                                                                            
  \caption{
    \label{fig:logl}
    log-likelihood distributions for \ejets{} (left) and
    for \mujets{} (right).
    }
\end{figure}

