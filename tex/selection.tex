\chapter{Selection and Backgrounds}
\label{sec:selection}

This section describes the event selection criteria used for this
analysis. The event samples used for this work are selected based on
requirements applied to the reconstructed objects in order to maximize
the signal (\ttbar{} events) to background ratio. In addition, this
section illustrates the techniques used to estimate the background
composition of the sample. 

\section{Event Selection}
\label{sec:evtsel}

As discussed in section \ref{sec:topdecay}, top and antitop quarks
decay mainly into a $W$ boson and a $b$ quark. The $W$ boson decays
into two quarks or into a lepton and a neutrino. Therefore, the
semi-leptonic decay channel corresponds to an event topology with at
least four reconstructed jets, exactly one isolated lepton (muon or
electron) and missing transverse energy.

A preliminary selection of the events is performed during data
acquisition by the single lepton trigger. Events are recorded when at
least one high-\pt{} muon or electron is reconstructed by the trigger
algorithms. 
The combination of \pt{} and isolation requirements is optimized to ensure
a good selection efficiency in different luminosity conditions. The
minimum \pt{} requirements applied for data collected in 2011 and 2012
are detailed in table~\ref{tab:trigthresholds}.

\begin{table}
  \caption{Minimum lepton \pt{} requirements for the single lepton trigger during
  2011 and 2012 data-taking, and the corresponding offline requirements.}
  \label{tab:trigthresholds}
  \centering
  \begin{tabular}{lcccc}
    \toprule
    \multirow{2}*{channel} & \multicolumn{2}{c}{$2011$ -- $\sqrt{s} =
      7 \TeV{}$} & \multicolumn{2}{c}{$2012$ -- $\sqrt{s} = 8\TeV{}$}
    \\
    & trigger & offline & trigger & offline \\
    \midrule
    \mujets{}  & $\pt{}>18 \GeV{}$ & $\pt{}>20 \GeV{}$ & $\pt{}>24
    \GeV{}$ & $\pt{}>25 \GeV{}$ \\
    \ejets{}     & $\pt{}>20 \GeV{}$ & $\pt{}>25 \GeV{}$ & $\pt{}>24
    \GeV{}$  & $\pt{}>25 \GeV{}$ \\
    \bottomrule
  \end{tabular}
\end{table}

The events used in the analysis are required to contain exactly one
lepton reconstructed offline as described in
section~\ref{sec:objects}. The lepton selected offline is required to
match within $\Delta R<0.15$ the corresponding lepton at the trigger
level.
Minimum \pt{} requirements ensure that the trigger efficiencies
are maximized and constant as a function of the lepton \pt{} for the
selected events.
The thresholds for the various scenarios are detailed in table~\ref{tab:trigthresholds}.

In addition to the lepton, a minimum of four jets with \pt{} larger
than $25 \GeV$ are required to be reconstructed in the event. Only
jets reconstructed in the range $|\eta|<2.5$ are
considered. The number of jets tagged as $b$-jets is used to define
\ttbar{} (with at least one $b$-jet) or background (without $b$-jets)
enriched regions.

In order to suppress the QCD multi-jets and Z+jets backgrounds, requirements on
the \met{} and the transverse mass \mt{} of the leptonically-decaying
W boson are applied.\footnote{$\mt = \sqrt{2 p^\ell_{\rm T} \met
    (1-\cos\Delta\phi)}$, with $p^\ell_{\rm T}$  being the transverse
  momentum (energy) of the muon (electron) and $\Delta\phi$ the
  azimuthal angle separation between the lepton and the direction of
  the missing transverse momentum.} Since the sample composition
varies with the lepton flavor and $b$-jet multiplicity, different
requirements are applied for each scenario, as detailed in table~\ref{tab:metmtwcuts}

\begin{table}
  \caption{Minimum \met{} and \mt{}(W) requirements.}
  \label{tab:metmtwcuts}
  \centering
  \begin{tabular}{lcccc}
    \toprule
     &
    $2011$ -- $\sqrt{s} = 7 \TeV$ &
    \multicolumn{3}{c}{$2012$ -- $\sqrt{s} = 8 \TeV$} \\
    channel   & $\geq{}1$ $b$-jet & 0 $b$-jets & 1
    $b$-jet & $\geq{}2$ $b$-jets \\
    \midrule
    \multirow{2}*{\mujets{}} & $\met{}>20 \GeV{}$ & $\met{}>40
    \GeV{}$ & $\met{}>20 \GeV{}$ & -- \\
                                           &
                                           $\met{}+\mtw{}>60 \GeV{}$ &
                                           $\met{}+\mtw{}>60 \GeV{}$ &
                                           $\met{}+\mtw{}>60 \GeV{}$ &
                                           -- \\

     \multirow{2}*{\ejets{}} & $\met{}>30 \GeV{}$ & $\met{}>40 \GeV{}$
     & $\met{}>20 \GeV{}$ & -- \\ 
                                         & $\mtw{}>30 \GeV{}$ &
                                       $\met{}+\mtw{}>60 \GeV{}$ &
                                       $\met{}+\mtw{}>60 \GeV{}$ & -- \\
    \bottomrule
  \end{tabular}
\end{table}

\section{Signal and background modeling}
\label{sec:bckgmodel}

The requirements described in section~\ref{sec:evtsel} are designed to
select events with semi-leptonic decays of the \ttbar{} pair. However, fully
leptonic decays can also be selected when one of the two lepton is not
reconstructed. These events are considered as part of the signal
sample, notwithstanding that the top quark pair kinematics cannot be
properly reconstructed under the semi-leptonic assumption.

Even though the requirements are designed to select the \ttbar{}
topology, other processes contaminate the samples: $W$ and $Z$ boson
production in association with jets (\wjets{},\zjets{}), single top,
QCD multi-jets, and boson pair productions constitute a non-negligible
fraction of the selected events.
In the case of \wjets{} events, the largest background, the leptonic W
decay produce a high-\pt{} isolated lepton, while the additional jets
production, including $b$-jets, mimics the hadronic top quark decay.
Analogously the leptonic decays of a boson in \zjets{} and boson pairs
productions feature high-\pt{} leptons reconstructed with high
efficiency.
On the other hand, the reconstruction of fake leptons happens seldom;
however, a very small selection efficiency for QCD multi-jets
production still results in a significant amount of background
events, due to the large cross section.

The background composition of the data samples is estimated with Monte
Carlo simulation and data-driven techniques. In particular the
expected yield of \wjets{} events and the flavor composition of the
associated jet is calibrated in-situ, while the QCD
multi-jet background is extrapolated from data control regions.

\subsection{Simulated samples}
\label{sec:mcsamples}

Different MC generators, PDF sets and parton shower and fragmentation
algorithms are used to model signal and backgrounds in the data
samples collected at $\sqrt{s} = 7 \TeV{}$ (2011) and at $\sqrt{s} = 8
\TeV{}$ (2012), with the goal of providing the most reliable
prediction in each scenario. In the following the different choices
for each process are listed.

The \ttbar{}+jets process is simulated using the \alpgen{} generator
with the {\tt CTEQ6L1} PDF set~\cite{cteq6}, interfaced to the 
\herwig parton shower (2011) and using \powheg{} with the {\tt CT10}
PDF set~\cite{ct10}, interfaced to the \pythia{} parton shower
(2012).

Simulated samples of $W/Z$ boson production are generated with up to
five additional partons using the {\tt ALPGEN} generator and the
{\tt CTEQ6L1} PDF set, interfaced to \herwig{} (2011) or
\pythia{} (2012) for parton showering and fragmentation. 
Dedicated samples are produced to simulate $W$ production in
association with bottom quark pairs (\wbb{}), charm quark
pairs (\wcc{}), single charm quarks (\wc{}), and light
flavor partons, including gluons (\wlight{}).
The $Z$+jets samples are generated separately for $Z$+light jets,
$Zb\bar{b}$+jets, and $Zc\bar{c}$+jets and normalized to the inclusive
NNLO theoretical cross section~\cite{vjetsxs}.
Overlap between $W/ZQ\bar{Q}$+jets ($Q=b,c$) events generated from the
matrix element calculation and those generated from parton-shower
evolution in the $W/Z$+light jets samples is avoided by using the
matrix element prediction only for $\Delta R(Q,\bar{Q})>0.4$.

Simulated samples of single top quark backgrounds corresponding to the
$s$-channel and $Wt$ production mechanisms are generated with \mcatnlo{}
interfaced with the \herwig{} parton shower (2011) and \powheg{}
interfaced with \pythia{} (2012), using the {\tt CT10} PDF set.
The simulation of $t$-channel single top quark production is
generated with the {\tt ACERMC} LO generator with the {\tt MRST LO**}
PDF set~\cite{mrst}  (2011) and \powheg{} interfaced with \pythia{}
with the {\tt CT10} PDF set (2012). Single top samples are normalized
to the corresponding NLO cross sections~\cite{stschan,sttchan,stwt}.
Finally, the boson pair production is modelled using \herwig{} with
the {\tt MRST LO**} PDF set, and is normalized to the NLO
theoretical cross sections~\cite{dibosonxs}.

\subsection{\wjets{} background normalization}
\label{sec:wjets}

The normalization of the \wjets{} background is measured in data in
order to constrain its uncertainty.
The procedure exploits the difference in production cross section at LHC
between \wpp{} and \wm{} to estimate the \wjets{} yield.
Due to the higher density in protons of $u$ quarks with respect to $d$ quarks,
the cross sections $\sigma(u\bar{d}\to \wpp{})$ and $\sigma(d\bar{u}\to
\wm{})$ are different, with a larger production rate for \wpp{}. The
prediction for the $W$ boson charge asymmetry in \wjets{} production is less
affected by theoretical uncertainties~\cite{wasym} and can be measured
in data to derive the correct overall normalization for the MC prediction.
The total number of \wjets{} events in the selected data sample
$N_W=N_{\wpp{}}+N_{\wm{}}$ is estimated as
\begin{equation}
N_W = \left(\frac{N_{\wpp{}}+N_{\wm{}}}{N_{\wpp{}}-N_{\wm{}}}\right )_{\rm MC}(N_{\wpp{}}-N_{\wm{}})_{\rm meas}
\label{eq:nw}
\end{equation}
, where positive and negative $W$ bosons are identified from the
charge of the reconstructed lepton. 

The $W$ boson charge asymmetry observed in the MC simulation depends
on the flavor composition of the sample, as \wbb{}, \wcc{}, \wc{}, \wlight{}
processes have different charge conjugations.
Therefore a calibration of the flavor composition is derived
simultaneously with the estimation of the total normalization in
equation~\ref{eq:nw}.
The relative fractions are estimated in a \wjets{} enriched control
region where exactly two reconstructed jets are required and no
$b$--tagging requirement is applied ({\it pretag} region). 
The additional requirement of at least one $b$--tagged jet is applied
to define a \wbb{} enriched region ({\it tag} region) and the \wjets{}
event yield is given by: 
\begin{equation}
\label{eq:nwt}
N^{W,{\rm tag}} = 
N^{W,{\rm pretag}}
\sum_{x=\bbbar{}, \ccbar{}, c, light} F_xP_x,
\end{equation}
where $F_x$ are the flavor fractions $N^{\rm pretag}_x/N^{\rm pretag}$
and $P_x$ is the selection efficiency of the $b$--tagging requirement for each
flavor type $x = \bbbar{}, \ccbar{}, c, light$. 
 With the assumption that \wbb{} and \wcc{} simulations require a
 fully correlated calibration, given the similarity of the processes,
 three calibration factors $K_{\bbbar{}/\ccbar{}}=F^{\rm
   data}_{\bbbar{}}/F^{\rm MC}_{\bbbar{}}=F^{\rm
   data}_{\ccbar{}}/F^{\rm MC}_{\ccbar{}}$, $K_c=F^{\rm data}_{c}/F^{\rm MC}_{c}$ and $K_{light}=F^{\rm
  data}_{light}/F^{\rm MC}_{light}$ are estimated to fit data. The
calibration factors are then extrapolated to the signal region, where
at least four jets are required, using the MC prediction
to renormalize to unity the sum of the flavor fractions.
Table~\ref{tab:wsf} summarizes the flavor fraction and normalization
calibration factors derived in the \ejets{} and \mujets{} channel for
the 2011 dataset at $\sqrt{s} = 7 \TeV{}$.
For the analysis of the 2012 dataset a similar approach is used by
implementing the {\it in-situ} calibration described in section~\ref{sec:fit}.

\begin{table}
  \caption{Calibration factors for flavor composition and overall
    normalization of the \wjets{} background as measured in the 2011 dataset.}
  \label{tab:wsf}
  \centering
  \begin{tabular}{lccc}
    \toprule
    channel   & $K_{\bbbar{}/\ccbar{}}$   & $K_{c}$       &
    $K_{light}$ \\
    \midrule
    \mujets{}  & $1.2\pm0.4$            & $1.0\pm0.4$ &
    $0.97\pm0.09$\\ 
    \ejets{}    & $1.4\pm0.4$            & $0.7\pm0.4$ &
    $1.00\pm0.10$\\ 
    \bottomrule
  \end{tabular}
\end{table}

\subsection{Multi-jets background}
\label{sec:qcdbckg}

Multijet events can pass the selection criteria when a lepton is
reconstructed in the event. The dominant sources of spurious leptons
are:
\begin{itemize}
\item semi-leptonic $b$-jet decays;
\item long lived weakly decaying states such as $\pi^{\pm}$ or $K$
  mesons;
\item reconstruction of $\pi^0$ showers as electrons;
\item reconstruction of electrons from conversions or direct photons.
\end{itemize}

While the probability of a QCD multi-jets event being selected is very
low, the production cross section for multi-jets events is orders of
magnitude larger above that of \ttbar{} production.
Because of this and the fact that these probabilities depend on the
detector configuration and geometry, it is more efficient to
extrapolate the multi-jet background in the selected sample from a
control region with a technique referred to as {\it matrix method}.

The matrix method is based on the selection of two categories of
events: the ones that satisfy loose lepton selection requirements, and
the ones that satisfy tight lepton selection requirements.
The number of events with one loose lepton and
the number of events with one tight lepton can be written as:
%                                                                                             
\begin{eqnarray}
  N^\mathrm{loose}
  & = & N^\mathrm{loose}_\mathrm{real}
  + N^{\mathrm{loose}}_\mathrm{fake} \nonumber \\
  N^\mathrm{tight}
  & = & N^\mathrm{tight}_\mathrm{real}
  + N^{\mathrm{tight}}_\mathrm{fake}
  \label{eqn:intro-mm-Nloose}
\end{eqnarray}
%                                                                                             
where $N^\mathrm{loose}_\mathrm{real}$ is the number of events with a
real lepton satisfying loose lepton-requirements,
$N^\mathrm{loose}_\mathrm{fake}$ is the number of multijet events
without a ``real'' lepton but still satisfying loose
lepton-requirements, and the same are $N^\mathrm{tight}_\mathrm{real}$
and $N^\mathrm{tight}_\mathrm{fake}$ for the tight
lepton-requirements. The efficiency
$\epsilon=N^\mathrm{tight}/N^\mathrm{loose}$ is different for real
leptons and for fake leptons. The signal efficiency
%                                                                                             
\begin{equation}
  \effReal{}
  = \frac{N^\mathrm{tight}_\mathrm{real}}{N^\mathrm{loose}_\mathrm{real}}
\end{equation}
can be measured with the tag-and-probe method in data dilepton events.
The fake efficiency
\begin{equation}
  \effFake{}
  = \frac{N^\mathrm{tight}_\mathrm{fake}}{N^\mathrm{loose}_\mathrm{fake}}.
  \label{eqn:intro-mm-real-fake}
\end{equation}
can be measured in data control regions where the contribution from
fake leptons is larger. $\epsilon$ is parametrized as a
function of kinematic observables such as lepton $\eta$ and \deltaR{\ell{}}{jet}. With these
two efficiencies, one can solve the two linear
equations~(\ref{eqn:intro-mm-Nloose}) for
$N^{\mathrm{tight}}_\mathrm{fake}$ as a function of $N^\mathrm{loose}$
and $N^\mathrm{tight}$, obtaining:
%                                                                                             
\begin{equation}
  N^\mathrm{tight}_\mathrm{fake}
  = \frac{\effFake{}}
  {\effReal{} - \effFake{}}
  (N^\mathrm{loose} \effReal{} - N^\mathrm{tight}).
  \label{eqn:intro-mm-tight_fake}
\end{equation}
%                                                                                             
The sample of multijet events is therefore estimated by the weighted
data events, where the weight for tight events is
$\effFake{}(\effReal{}-1)/(\effReal{}-\effFake{})$
and the weight for loose-not-tight events is
$(\effFake{}\cdot{}\effReal{})/(\effReal{}-\effFake{})$.

\section{Comparison between data and prediction}
\label{sec:datamc}

The expected and observed number of selected events in both the \mujets{} and
the \ejets{} channels are reported in tables~\ref{tab:yields2011}
(2011), \ref{tab:yields2012mu} and \ref{tab:yields2012ele} (2012).

\begin{table}
  \centering
  \begin{tabular}{ l  rrr  rrr  rrr  rrr }
    \toprule
    Channel & \multicolumn{3}{c}{$\mu$ + jets pretag} & \multicolumn{3}{c}{$\mu$ + jets tag} & \multicolumn{3}{c}{$e$ + jets pretag} & \multicolumn{3}{c}{$e$ + jets tag} \\ 
    \midrule
    \ttbar{}         & 34900 &$\!\!\!\pm\!\!\!$& 2200  & 30100 &$\!\!\!\pm\!\!\!$& 1900  & 21400 &$\!\!\!\pm\!\!\!$& 1300  & 18500 &$\!\!\!\pm\!\!\!$& 1100  \\ %NEW tt x-sec
    Single top       & 2460  &$\!\!\!\pm\!\!\!$& 120   & 1970  &$\!\!\!\pm\!\!\!$& 100   & 1530  &$\!\!\!\pm\!\!\!$& 80    & 1220  &$\!\!\!\pm\!\!\!$& 60   \\ 
    \wjets{}         & 28200 &$\!\!\!\pm\!\!\!$& 3100  & 4800  &$\!\!\!\pm\!\!\!$& 900   & 13200 &$\!\!\!\pm\!\!\!$& 1600  & 2300  &$\!\!\!\pm\!\!\!$& 900  \\
    \zjets{}         & 3000  &$\!\!\!\pm\!\!\!$& 1900  & 480   &$\!\!\!\pm\!\!\!$& 230   & 3000  &$\!\!\!\pm\!\!\!$& 1400  & 460   &$\!\!\!\pm\!\!\!$& 220   \\
    Diboson          & 380   &$\!\!\!\pm\!\!\!$& 180   & 80    &$\!\!\!\pm\!\!\!$& 40    & 230   &$\!\!\!\pm\!\!\!$& 110   & 47    &$\!\!\!\pm\!\!\!$& 22   \\ 
    Multijet      & 5500  &$\!\!\!\pm\!\!\!$& 1100  & 1800
    &$\!\!\!\pm\!\!\!$& 400   & 3800  &$\!\!\!\pm\!\!\!$& 1900  & 800
    &$\!\!\!\pm\!\!\!$& 400  \\ 
    \midrule
    Total background & 40000 &$\!\!\!\pm\!\!\!$& 4000  & 9200
    &$\!\!\!\pm\!\!\!$& 1000  & 21700 &$\!\!\!\pm\!\!\!$& 2900  & 4800
    &$\!\!\!\pm\!\!\!$& 1000  \\ 
    \midrule
    Total expected     & 74000 &$\!\!\!\pm\!\!\!$ & 4000 & 39300
    &$\!\!\!\pm\!\!\!$& 2100  & 43100 &$\!\!\!\pm\!\!\!$& 3100 & 23300
    &$\!\!\!\pm\!\!\!$& 1600  \\
    \midrule
    Observed                & \multicolumn{3}{c}{70845}  & \multicolumn{3}{c}{37568} & \multicolumn{3}{c}{40972}      & \multicolumn{3}{c}{21929} \\
    \bottomrule
  \end{tabular}
  \caption[Event yields for the samples at $\sqrt{s} = 7 \TeV{}$]{
    Numbers of expected events for the \ttbar{} signal and the various background 
    processes and observed events in data for the dataset 
    at $\sqrt{s} = 7 \TeV{}$.
    The uncertainties include statistical and systematic components.}
  \label{tab:yields2011}
\end{table}


\begin{table}
  \begin{tabular}{ l  rrr  rrr  rrr  rrr }
    \toprule
    Channel & \multicolumn{3}{c}{\mujets{} pretag} & \multicolumn{3}{c}{\mujets{} 0-tag} & \multicolumn{3}{c}{\mujets{} 1-tag} & \multicolumn{3}{c}{\mujets{} 2-tag} \\
    \midrule      
    $t\bar{t}$                  & 203000 &$\!\!\!\pm\!\!\!$& 3000   & 21100 &$\!\!\!\pm\!\!\!$& 800   & 86100 &$\!\!\!\pm\!\!\!$& 1700 & 96100 &$\!\!\!\pm\!\!\!$& 900  \\
    Single top                  & 13300  &$\!\!\!\pm\!\!\!$& 1300   & 1950 &$\!\!\!\pm\!\!\!$& 220    & 6700 &$\!\!\!\pm\!\!\!$& 700   & 4600 &$\!\!\!\pm\!\!\!$& 400   \\
    \wbb{}                      & 48000  &$\!\!\!\pm\!\!\!$& 4000   & 28500 &$\!\!\!\pm\!\!\!$& 2800  & 15200 &$\!\!\!\pm\!\!\!$& 1400 & 4000 &$\!\!\!\pm\!\!\!$& 400   \\
    \wc{}                       & 19000  &$\!\!\!\pm\!\!\!$& 5000   & 13200 &$\!\!\!\pm\!\!\!$& 3600  & 5100 &$\!\!\!\pm\!\!\!$& 1400  & 320 &$\!\!\!\pm\!\!\!$& 90\\
    \wlight{}                   & 62400  &$\!\!\!\pm\!\!\!$& 3300   & 58600 &$\!\!\!\pm\!\!\!$& 3100  & 3580 &$\!\!\!\pm\!\!\!$& 200   & 197 &$\!\!\!\pm\!\!\!$& 11  \\   
    \zjets{}                    & 10000  &$\!\!\!\pm\!\!\!$& 4000   & 7000 &$\!\!\!\pm\!\!\!$& 3300   & 1900 &$\!\!\!\pm\!\!\!$& 1000  & 800 &$\!\!\!\pm\!\!\!$& 400\\
    Diboson                     & 5900   &$\!\!\!\pm\!\!\!$& 2700   & 4600 &$\!\!\!\pm\!\!\!$& 2100   & 1100 &$\!\!\!\pm\!\!\!$& 500   & 210 &$\!\!\!\pm\!\!\!$& 90   \\  
    Multijet                    & 7500   &$\!\!\!\pm\!\!\!$& 2300   & 3600 &$\!\!\!\pm\!\!\!$& 1600   & 2500 &$\!\!\!\pm\!\!\!$& 1400  & 1400 &$\!\!\!\pm\!\!\!$& 600   \\
    \midrule                                                                                                                                                            
    Total background            & 165000 &$\!\!\!\pm\!\!\!$& 3000   & 117400 &$\!\!\!\pm\!\!\!$& 900  & 36100 &$\!\!\!\pm\!\!\!$& 1700 & 11500 &$\!\!\!\pm\!\!\!$& 800\\
    \midrule                                                                                                                                                            
    Total expected              & 368300 &$\!\!\!\pm\!\!\!$& 700    & 138500 &$\!\!\!\pm\!\!\!$& 400  & 122200 &$\!\!\!\pm\!\!\!$& 400 & 107600 &$\!\!\!\pm\!\!\!$& 330 \\
    \midrule                                                                                                                                                            
    Observed                    & \multicolumn{3}{c}{368324}        & \multicolumn{3}{c}{138490}      & \multicolumn{3}{c}{122230}     & \multicolumn{3}{c}{107604}     \\
    \bottomrule
  \end{tabular}
  \caption[Event yields for the \mujets{} samples at $\sqrt{s} = 8 \TeV{}$]{
    Numbers of expected events for the \ttbar{} signal and the various background 
    processes and observed events in data for the \mujets{} dataset
    at $\sqrt{s} = 8 \TeV{}$.
    The calibrated yields and uncertainties are shown.}
  \label{tab:yields2012mu}
\end{table}

\begin{table}
  \begin{tabular}{ l  rrr  rrr  rrr  rrr }
    \toprule
    Channel & \multicolumn{3}{c}{\ejets{} pretag} & \multicolumn{3}{c}{\ejets{} 0-tag} & \multicolumn{3}{c}{\ejets{} 1-tag} & \multicolumn{3}{c}{\ejets{} 2-tag} \\
    \midrule
    $t\bar{t}$            & 153100 &$\!\!\!\pm\!\!\!$& 2700   & 13900 &$\!\!\!\pm\!\!\!$& 700   & 63000 &$\!\!\!\pm\!\!\!$& 1600   & 76200 &$\!\!\!\pm\!\!\!$& 800  \\ 
    Single top            & 11300  &$\!\!\!\pm\!\!\!$& 1200   & 1540 &$\!\!\!\pm\!\!\!$& 200    & 5600 &$\!\!\!\pm\!\!\!$& 600     & 4200 &$\!\!\!\pm\!\!\!$& 400  \\
    \wbb{}                & 36000  &$\!\!\!\pm\!\!\!$& 5000   & 21200 &$\!\!\!\pm\!\!\!$& 2700  & 12000 &$\!\!\!\pm\!\!\!$& 1500   & 3200 &$\!\!\!\pm\!\!\!$& 400  \\
    \wc{}                 & 16000  &$\!\!\!\pm\!\!\!$& 6000   & 11000 &$\!\!\!\pm\!\!\!$& 4000  & 4700 &$\!\!\!\pm\!\!\!$& 1700    & 310 &$\!\!\!\pm\!\!\!$& 120\\
    \wlight{}             & 39600  &$\!\!\!\pm\!\!\!$& 3200   & 37200 &$\!\!\!\pm\!\!\!$& 3000  & 2400 &$\!\!\!\pm\!\!\!$& 200     & 86 &$\!\!\!\pm\!\!\!$& 7     \\ 
    \zjets{}              & 10000  &$\!\!\!\pm\!\!\!$& 4000   & 6800 &$\!\!\!\pm\!\!\!$& 3000   & 2500 &$\!\!\!\pm\!\!\!$& 1200    & 1000 &$\!\!\!\pm\!\!\!$& 500\\
    Diboson               & 5100   &$\!\!\!\pm\!\!\!$& 2300   & 3900 &$\!\!\!\pm\!\!\!$& 1700   & 1000 &$\!\!\!\pm\!\!\!$& 400     & 220 &$\!\!\!\pm\!\!\!$& 100   \\ 
    Multijet              & 12000  &$\!\!\!\pm\!\!\!$& 4000   & 8400 &$\!\!\!\pm\!\!\!$& 3100   & 3100 &$\!\!\!\pm\!\!\!$& 1300    & 640 &$\!\!\!\pm\!\!\!$& 340 \\
    \midrule                                                                                                                                            
    Total background      & 130800 &$\!\!\!\pm\!\!\!$& 2800   & 90000 &$\!\!\!\pm\!\!\!$& 900   & 31300 &$\!\!\!\pm\!\!\!$& 1600   & 9600 &$\!\!\!\pm\!\!\!$& 700  \\
    \midrule                                                                                                                                                
    Total expected        & 284000 &$\!\!\!\pm\!\!\!$& 600    & 103900 &$\!\!\!\pm\!\!\!$& 400  & 94270 &$\!\!\!\pm\!\!\!$& 320    & 85780 &$\!\!\!\pm\!\!\!$& 290 \\
    \midrule                                                                                                                                                      
    Observed              & \multicolumn{3}{c}{283979}        & \multicolumn{3}{c}{103930}      & \multicolumn{3}{c}{94235}        & \multicolumn{3}{c}{85814}       \\
    \bottomrule
  \end{tabular}
  \caption[Event yields for the \ejets{} samples at $\sqrt{s} = 8 \TeV{}$]{
    Numbers of expected events for the \ttbar{} signal and the various background 
    processes and observed events in data for the \ejets{} dataset
    at $\sqrt{s} = 8 \TeV{}$.
    The calibrated yields and uncertainties are shown.}
  \label{tab:yields2012ele}
\end{table}


The modeling of the main background process is validated by comparing
distributions as observed in the data samples with the corresponding
predictions from simulation. Kinematic quantities of the reconstructed
objects used for selection are
considered. Figures~\ref{fig:2011mu_pretag}
and~\ref{fig:2011mu_tagged} (\mujets{}), and~\ref{fig:2011ele_pretag}
and~\ref{fig:2011ele_tagged} show the comparison of the distributions
of the lepton \pt{} and pseudo-rapidity, the leading\footnote{Jets are
conventionally ordered in decreasing \pt{}; hence the ``leading jet''
is the reconstructed jet with the highest transverse momentum} jet
\pt{} and pseudo-rapidity, the W transverse mass and the missing
transverse energy for background and signal enriched samples of the
2011 dataset at $\sqrt{s} = 7 \TeV{}$. The analogous distributions for
samples of the 2012 dataset at $\sqrt{s} = 8 \TeV{}$ are shown in
figures~\ref{fig:2012mu_pretag},~\ref{fig:2012mu_0tag},~\ref{fig:2012mu_1tag},~\ref{fig:2012mu_2tag}
(\mujets{}) and
~\ref{fig:2012ele_pretag},~\ref{fig:2012ele_0tag},~\ref{fig:2012ele_1tag},~\ref{fig:2012ele_2tag} (\ejets{}).

\begin{figure}
  \centering
  \includegraphics[width=0.495\textwidth]{figures/selection/7TeV/mujets/pretag/Muon_Pt}
  \includegraphics[width=0.495\textwidth]{figures/selection/7TeV/mujets/pretag/Muon_Eta}
  \includegraphics[width=0.495\textwidth]{figures/selection/7TeV/mujets/pretag/Jet1_Pt}
  \includegraphics[width=0.495\textwidth]{figures/selection/7TeV/mujets/pretag/Jet1_Eta}
  \includegraphics[width=0.495\textwidth]{figures/selection/7TeV/mujets/pretag/WMt}
  \includegraphics[width=0.495\textwidth]{figures/selection/7TeV/mujets/pretag/MET}
  \caption[Control plots for the \mujets{} channel in the pretag
    sample at $\sqrt{s} = 7 \TeV{}$.]{Control plots for the \mujets{} channel in the pretag
    sample at $\sqrt{s} = 7 \TeV{}$. From the top left to the bottom right, the transverse
    momentum \pt{}, and the pseudorapidity $\eta{}$ are shown for the
    muon (top) and for the leading jet (middle).  The distributions
    for the W transverse mass \mtw{} and the missing transverse
    momentum \met{} are shown in the bottom row. 
    Uncertainty on background normalization is shown.}
  \label{fig:2011mu_pretag}
\end{figure}
%
\begin{figure}
\centering
\includegraphics[width=0.495\textwidth]{figures/selection/7TeV/mujets/tagged/Muon_Pt}
\includegraphics[width=0.495\textwidth]{figures/selection/7TeV/mujets/tagged/Muon_Eta}
\includegraphics[width=0.495\textwidth]{figures/selection/7TeV/mujets/tagged/Jet1_Pt}
\includegraphics[width=0.495\textwidth]{figures/selection/7TeV/mujets/tagged/Jet1_Eta}
\includegraphics[width=0.495\textwidth]{figures/selection/7TeV/mujets/tagged/WMt}
\includegraphics[width=0.495\textwidth]{figures/selection/7TeV/mujets/tagged/MET}
\caption[Control plots for the \mujets{} channel in the tagged
  sample at $\sqrt{s} = 7 \TeV{}$]{Control plots for the \mujets{} channel in the tagged
  sample at $\sqrt{s} = 7 \TeV{}$.  From the top left to the bottom right, the transverse
  momentum \pt{}, and the pseudorapidity $\eta$ are shown for the
  muon (top) and for the leading jet (middle). The bottom plots show
  the W transverse mass \mtw{} and the missing transverse
  momentum \met{}. Uncertainty on background normalization is shown.}
\label{fig:2011mu_tagged}
\end{figure}
%
\begin{figure}
  \centering
  \includegraphics[width=0.495\textwidth]{figures/selection/7TeV/ejets/pretag/Electron_Pt}
  \includegraphics[width=0.495\textwidth]{figures/selection/7TeV/ejets/pretag/Electron_Eta}
  \includegraphics[width=0.495\textwidth]{figures/selection/7TeV/ejets/pretag/Jet1_Pt}
  \includegraphics[width=0.495\textwidth]{figures/selection/7TeV/ejets/pretag/Jet1_Eta}
  \includegraphics[width=0.495\textwidth]{figures/selection/7TeV/ejets/pretag/WMt}
  \includegraphics[width=0.495\textwidth]{figures/selection/7TeV/ejets/pretag/MET}
  \caption[Control plots for the \ejets{} channel in the pretag
    sample at $\sqrt{s} = 7 \TeV{}$]{Control plots for the \ejets{} channel in the pretag
    sample at $\sqrt{s} = 7 \TeV{}$.  From the top left to the bottom right, the transverse
    momentum \pt{}, and the pseudorapidity $\eta$ are shown for the
    electron (top) and for the leading jet (middle). The bottom plots
    show the W transverse mass \mtw{} and the missing transverse
    momentum \met{}. Uncertainty on background normalization is shown.}
  \label{fig:2011ele_pretag}
\end{figure}
%                                      
\begin{figure}
\centering
\includegraphics[width=0.495\textwidth]{figures/selection/7TeV/ejets/tagged/Electron_Pt}
\includegraphics[width=0.495\textwidth]{figures/selection/7TeV/ejets/tagged/Electron_Eta}
\includegraphics[width=0.495\textwidth]{figures/selection/7TeV/ejets/tagged/Jet1_Pt}
\includegraphics[width=0.495\textwidth]{figures/selection/7TeV/ejets/tagged/Jet1_Eta}
\includegraphics[width=0.495\textwidth]{figures/selection/7TeV/ejets/tagged/WMt}
\includegraphics[width=0.495\textwidth]{figures/selection/7TeV/ejets/tagged/MET}
\caption[Control plots for the \ejets{} channel in the tagged
  sample at $\sqrt{s} = 7 \TeV{}$]{Control plots for the \ejets{} channel in the tagged
  sample at $\sqrt{s} = 7 \TeV{}$.  From the top left to the bottom right, the transverse
  momentum \pt{}, and the pseudorapidity $\eta$ are shown for the
  electron (top) and for the leading jet (middle). The bottom plots
  show the W transverse mass \mtw{} and the missing transverse
  momentum \met{}. Uncertainty on background normalization is shown.}
\label{fig:2011ele_tagged}
\end{figure}

% 8 TeV

\begin{figure}
  \centering
  \includegraphics[width=0.495\textwidth]{figures/selection/8TeV/mujets/pretag/lep_pt_1}
  \includegraphics[width=0.495\textwidth]{figures/selection/8TeV/mujets/pretag/lep_eta_1}
  \includegraphics[width=0.495\textwidth]{figures/selection/8TeV/mujets/pretag/jet_pt_1}
  \includegraphics[width=0.495\textwidth]{figures/selection/8TeV/mujets/pretag/jet_eta_1}
  \includegraphics[width=0.495\textwidth]{figures/selection/8TeV/mujets/pretag/mtw}
  \includegraphics[width=0.495\textwidth]{figures/selection/8TeV/mujets/pretag/met}
  \caption[Control plots for the \mujets{} channel in the pretag
    sample at $\sqrt{s} = 8 \TeV{}$]{Control plots for the \mujets{} channel in the pretag
    sample at $\sqrt{s} = 8 \TeV{}$.  From the top left to the bottom right, the transverse
    momentum \pt{}, and the pseudorapidity $\eta$ are shown for the
    muon (top) and for the leading jet (middle).  The distributions
    for the W transverse mass \mtw{} and the missing transverse
    momentum \met{} are shown in the bottom row. 
    %    Uncertainty on background normalization is shown.
  }
  \label{fig:2012mu_pretag}
\end{figure}
%
\begin{figure}
  \centering
  \includegraphics[width=0.495\textwidth]{figures/selection/8TeV/mujets/zeroTag/lep_pt_1}
  \includegraphics[width=0.495\textwidth]{figures/selection/8TeV/mujets/zeroTag/lep_eta_1}
  \includegraphics[width=0.495\textwidth]{figures/selection/8TeV/mujets/zeroTag/jet_pt_1}
  \includegraphics[width=0.495\textwidth]{figures/selection/8TeV/mujets/zeroTag/jet_eta_1}
  \includegraphics[width=0.495\textwidth]{figures/selection/8TeV/mujets/zeroTag/mtw}
  \includegraphics[width=0.495\textwidth]{figures/selection/8TeV/mujets/zeroTag/met}
  \caption[Control plots for the \mujets{} channel in the 0-tag
    sample at $\sqrt{s} = 8 \TeV{}$]{Control plots for the \mujets{} channel in the 0-tag
    sample at $\sqrt{s} = 8 \TeV{}$.  From the top left to the bottom right, the transverse
    momentum \pt{}, and the pseudorapidity $\eta$ are shown for the
    muon (top) and for the leading jet (middle). The bottom plots show
    the W transverse mass \mtw{} and the missing transverse
    momentum \met{}. 
    %    Uncertainty on background normalization is shown.
  }
  \label{fig:2012mu_0tag}
\end{figure}
%
\begin{figure}
  \centering
  \includegraphics[width=0.495\textwidth]{figures/selection/8TeV/mujets/oneTag/lep_pt_1}
  \includegraphics[width=0.495\textwidth]{figures/selection/8TeV/mujets/oneTag/lep_eta_1}
  \includegraphics[width=0.495\textwidth]{figures/selection/8TeV/mujets/oneTag/jet_pt_1}
  \includegraphics[width=0.495\textwidth]{figures/selection/8TeV/mujets/oneTag/jet_eta_1}
  \includegraphics[width=0.495\textwidth]{figures/selection/8TeV/mujets/oneTag/mtw}
  \includegraphics[width=0.495\textwidth]{figures/selection/8TeV/mujets/oneTag/met}
  \caption[Control plots for the \mujets{} channel in the 1-tag
    sample at $\sqrt{s} = 8 \TeV{}$]{Control plots for the \mujets{} channel in the 1-tag
    sample at $\sqrt{s} = 8 \TeV{}$.  From the top left to the bottom right, the transverse
    momentum \pt{}, and the pseudorapidity $\eta$ are shown for the
    muon (top) and for the leading jet (middle). The bottom plots show
    the W transverse mass \mtw{} and the missing transverse
    momentum \met{}. 
    %   Uncertainty on background normalization is shown.
  }
  \label{fig:2012mu_1tag}
\end{figure}
%
\begin{figure}
  \centering
  \includegraphics[width=0.495\textwidth]{figures/selection/8TeV/mujets/twoTag/lep_pt_1}
  \includegraphics[width=0.495\textwidth]{figures/selection/8TeV/mujets/twoTag/lep_eta_1}
  \includegraphics[width=0.495\textwidth]{figures/selection/8TeV/mujets/twoTag/jet_pt_1}
  \includegraphics[width=0.495\textwidth]{figures/selection/8TeV/mujets/twoTag/jet_eta_1}
  \includegraphics[width=0.495\textwidth]{figures/selection/8TeV/mujets/twoTag/mtw}
  \includegraphics[width=0.495\textwidth]{figures/selection/8TeV/mujets/twoTag/met}
  \caption[Control plots for the \mujets{} channel in the 2-tag
    sample at $\sqrt{s} = 8 \TeV{}$]{Control plots for the \mujets{} channel in the 2-tag
    sample at $\sqrt{s} = 8 \TeV{}$.  From the top left to the bottom right, the transverse
    momentum \pt{}, and the pseudorapidity $\eta$ are shown for the
    muon (top) and for the leading jet (middle). The bottom plots show
    the W transverse mass \mtw{} and the missing transverse
    momentum \met{}. 
    %    Uncertainty on background normalization is shown.
  }
  \label{fig:2012mu_2tag}
\end{figure}

\begin{figure}
  \centering
  \includegraphics[width=0.495\textwidth]{figures/selection/8TeV/ejets/pretag/lep_pt_1}
  \includegraphics[width=0.495\textwidth]{figures/selection/8TeV/ejets/pretag/lep_eta_1}
  \includegraphics[width=0.495\textwidth]{figures/selection/8TeV/ejets/pretag/jet_pt_1}
  \includegraphics[width=0.495\textwidth]{figures/selection/8TeV/ejets/pretag/jet_eta_1}
  \includegraphics[width=0.495\textwidth]{figures/selection/8TeV/ejets/pretag/mtw}
  \includegraphics[width=0.495\textwidth]{figures/selection/8TeV/ejets/pretag/met}
  \caption[Control plots for the \ejets{} channel in the pretag
    sample at $\sqrt{s} = 8 \TeV{}$]{Control plots for the \ejets{} channel in the pretag
    sample at $\sqrt{s} = 8 \TeV{}$.  From the top left to the bottom right, the transverse
    momentum \pt{}, and the pseudorapidity $\eta$ are shown for the
    electron (top) and for the leading jet (middle). The bottom plots
    show the W transverse mass \mtw{} and the missing transverse
    momentum \met{}. 
    %    Uncertainty on background normalization is shown.
  }
  \label{fig:2012ele_pretag}
\end{figure}
%
\begin{figure}
  \centering
  \includegraphics[width=0.495\textwidth]{figures/selection/8TeV/ejets/zeroTag/lep_pt_1}
  \includegraphics[width=0.495\textwidth]{figures/selection/8TeV/ejets/zeroTag/lep_eta_1}
  \includegraphics[width=0.495\textwidth]{figures/selection/8TeV/ejets/zeroTag/jet_pt_1}
  \includegraphics[width=0.495\textwidth]{figures/selection/8TeV/ejets/zeroTag/jet_eta_1}
  \includegraphics[width=0.495\textwidth]{figures/selection/8TeV/ejets/zeroTag/mtw}
  \includegraphics[width=0.495\textwidth]{figures/selection/8TeV/ejets/zeroTag/met}
  \caption[Control plots for the \ejets{} channel in the 0-tag
    sample at $\sqrt{s} = 8 \TeV{}$]{Control plots for the \ejets{} channel in the 0-tag
    sample at $\sqrt{s} = 8 \TeV{}$.  From the top left to the bottom right, the transverse
    momentum \pt{}, and the pseudorapidity $\eta$ are shown for the
    electron (top) and for the leading jet (middle). The bottom plots
    show the W transverse mass \mtw{} and the missing transverse
    momentum \met{}.
    %    Uncertainty on background normalization is shown.
  }
  \label{fig:2012ele_0tag}
\end{figure}
%
\begin{figure}
  \centering
  \includegraphics[width=0.495\textwidth]{figures/selection/8TeV/ejets/oneTag/lep_pt_1}
  \includegraphics[width=0.495\textwidth]{figures/selection/8TeV/ejets/oneTag/lep_eta_1}
  \includegraphics[width=0.495\textwidth]{figures/selection/8TeV/ejets/oneTag/jet_pt_1}
  \includegraphics[width=0.495\textwidth]{figures/selection/8TeV/ejets/oneTag/jet_eta_1}
  \includegraphics[width=0.495\textwidth]{figures/selection/8TeV/ejets/oneTag/mtw}
  \includegraphics[width=0.495\textwidth]{figures/selection/8TeV/ejets/oneTag/met}
  \caption[Control plots for the \ejets{} channel in the 1-tag
    sample at $\sqrt{s} = 8 \TeV{}$]{Control plots for the \ejets{} channel in the 1-tag
    sample at $\sqrt{s} = 8 \TeV{}$.  From the top left to the bottom right, the transverse
    momentum \pt{}, and the pseudorapidity $\eta$ are shown for the
    electron (top) and for the leading jet (middle). The bottom plots
    show the W transverse mass \mtw{} and the missing transverse
    momentum \met{}.
    %    Uncertainty on background normalization is shown.
  }
  \label{fig:2012ele_1tag}
\end{figure}
%
\begin{figure}
  \centering
  \includegraphics[width=0.495\textwidth]{figures/selection/8TeV/ejets/twoTag/lep_pt_1}
  \includegraphics[width=0.495\textwidth]{figures/selection/8TeV/ejets/twoTag/lep_eta_1}
  \includegraphics[width=0.495\textwidth]{figures/selection/8TeV/ejets/twoTag/jet_pt_1}
  \includegraphics[width=0.495\textwidth]{figures/selection/8TeV/ejets/twoTag/jet_eta_1}
  \includegraphics[width=0.495\textwidth]{figures/selection/8TeV/ejets/twoTag/mtw}
  \includegraphics[width=0.495\textwidth]{figures/selection/8TeV/ejets/twoTag/met}
  \caption[Control plots for the \ejets{} channel in the 2-tag
    sample at $\sqrt{s} = 8 \TeV{}$]{Control plots for the \ejets{} channel in the 2-tag
    sample at $\sqrt{s} = 8 \TeV{}$.  From the top left to the bottom right, the transverse
    momentum \pt{}, and the pseudorapidity $\eta$ are shown for the
    electron (top) and for the leading jet (middle). The bottom plots
    show the W transverse mass \mtw{} and the missing transverse
    momentum \met{}. 
    %   Uncertainty on background normalization is shown.
  }
  \label{fig:2012ele_2tag}
\end{figure}



