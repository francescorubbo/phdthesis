\chapter{Results}
\label{sec:results}


\section{\ac{} measurements at \seventev{}}
\label{sec:res7tev}

The measurements are performed in the $\ell{}+\ge4$ jets sample with at least 1
$b$--jet. The observed \dy{} distribution in data and the background
predictions used as input for the inclusive measurement are shown in
Fig.~\ref{fig:dy7tev}.

\begin{figure}[!htb]\centering
  \includegraphics[width=0.6\textwidth]{figures/results/dy7tev} 
  \caption{}
  \label{fig:dy7tev}
\end{figure}

The asymmetry \ac{} in \ttbar{} production at \seventev{} is measured to be
$\ac{}= 0.006\pm0.010$, compatible with the SM prediction
$\ac{}=0.0123\pm0.0005$.
Table~\ref{tab:results} summarizes the
measurements and predictions for the inclusive asymmetry, at high
invariant mass of the top quark pair ($\mtt{} >600 \GeV{})$, and at
high $z$--component of the velocity of the \ttbar{} system ($\betatt{}>0.6$).
The quoted uncertainty corresponds to the RMS of the marginal posterior
probability density $\conditionalProb{\ac{}}{\Data{}}$ described in
Sec.~\ref{sec:marginalization}; therefore it accounts for both
statistical and systematic components.

\begin{table}[!htb]\centering
\begin{tabular}{c c c}
\toprule
  & Data & Theory\\
\midrule
 \ac{}                                     & $0.006\pm0.010$ & $0.0123\pm0.0005$ \\
 $\ac{}(\mtt{} > 600 \GeV{})$ & $0.018\pm0.022$ & $0.0175^{+0.0005}_{-0.0004}$\\
 $\ac{}(\beta_{z,\ttbar}>0.6)$   & $0.011\pm0.018$ & $0.020^{+0.006}_{-0.007}$\\
\bottomrule
\end{tabular}
\caption{Measured charge asymmetry \ac{} compared with SM
  predictions. The measurements with the \qqbar{}--enhancing
  $\beta_{z,\ttbar} > 0.6$ requirement and at high \ttbar{} invariant
  mass $\mtt{} > 600 \GeV{}$ are also included.}
\label{tab:results}
\end{table}

In order to study the impact on \ac{} of each systematic uncertainty,
the unfolding procedure is performed for each variation without using
marginalization. 
For each source of uncertainty, the background templates and
response matrix corresponding to $\pm1\sigma$ variation are
considered. 
Table~\ref{tab:systematics} shows the average variations in the
asymmetry computed as $|\ac{}(+1\sigma)-\ac{}(-1\sigma)|/2$.
While the precision is dominated by the statistical uncertainty, the
sources of largest systematic uncertainties are the ones with a large
impact on the size of the \wjets{} background, such as the
uncertainty on the energy scale and resolution of lepton and jets. 

\begin{table}[!htb]\centering
\begin{tabular}{l c c c}
\toprule
Source of systematic uncertainty  & \multicolumn{2}{c}{$\delta{\ac{}}$} \\
\midrule
  & Inclusive & $\mtt{} >600\GeV{}$ & $\betatt{}>0.6$ \\
\midrule
Lepton reconstruction/identification    & $<0.001$ & $<0.002$ & $<0.002$\\
Lepton energy scale and resolution      & $0.003$    & $0.003$   &  $0.003$\\
Jet energy scale and resolution             & $0.003$    & $0.003$   &
$0.005$ \\
Missing transverse momentum and pile--up modeling & $0.002$ & $0.002$
& $0.004$\\
Multi--jets background normalisation    & $<0.001$ & $<0.002$ & $<0.002$\\
$b$--tagging/mis--tag efficiency         & $<0.001$  & $<0.002$ & $<0.002$\\
Signal modeling                                       & $<0.001$ &
$<0.002$ & $<0.002$\\
Parton shower/hadronization             & $<0.001$  & $<0.002$ & $<0.002$\\
Monte Carlo statistics                         & $0.002$  & $<0.002$ & $<0.002$\\
PDF                                                      & $0.001$  & $<0.002$  & $<0.002$\\
$W$+jets normalisation and shape     & $0.002$  & $<0.002$ & $<0.002$\\
\midrule
Statistical uncertainty                          & $0.010$ & $0.021$ & $0.017$\\
\bottomrule
\end{tabular}
\caption{Impact of individual sources of uncertainty on the measured
  \ac{} and $\ac{}(\mtt{}>600 \GeV{})$. Variations below $10\%$ of the
  statistical uncertainty are considered negligible.}
\label{tab:systematics}
\end{table}

The asymmetry spectra, measured as functions of
\mtt{}, \pttt{} and \ytt{} are compared in Fig.~\ref{fig:unfac_diff}
with the theoretical SM predictions and found to be compatible. 
In addition, the results are compared with new physics predictions for
color octets ({\it axigluons}) with masses $M_G=300 \GeV{}$ and $M_G=7
\TeV{}$~\cite{AguilarSaavedra:2011ci}. Both models would not
be observable in direct searches as \ttbar{} resonances: the light
axigluon mass is below the threshold for production of top quark
pairs, while a $7 \TeV$ mass is beyond the kinematic reach of the
current searches at \seventev{} and \eighttev{}. In both scenarios
the new physics couplings with top and light quarks are tuned to yield a
forward--backward asymmetry compatible with the results reported by
Tevatron experiments.

\begin{figure}[!htb]\centering
  \subfloat[][]
  {  \includegraphics[width=0.45\textwidth]{figures/results/dy_mtt}
  } \quad
  \subfloat[][]
  {  \includegraphics[width=0.45\textwidth]{figures/results/dy_pttt} } \\
  \subfloat[][]
  {  \includegraphics[width=0.45\textwidth]{figures/results/dy_ytt}
  } \quad
  \subfloat[][]
  {  \includegraphics[width=0.45\textwidth]{figures/results/dy_mtt_beta} }
  \caption{Measured \ac{} spectra compared with SM and bSM predictions
    as functions of \mtt{} (a), \pttt{} (b) and \ytt{} (c). The
    asymmetry as a function of \mtt{} with $\betatt{}>0.6$ is also
    shown (d).}
  \label{fig:unfac_diff}
\end{figure}

The asymmetry values corresponding to the spectra in
Fig.~\ref{fig:unfac_diff} are detailed in App.~\ref{app:differentialresults}.
The correlation of the asymmetry values $\ac{}^i$ across differential bins is
fully described by the posterior probability density
$\conditionalProb{\ac{}^1,...,\ac{}^N}{\Data{}}$, where
$N$ is the number of differential bins, and it is summarized in
App.~\ref{app:correlations}.
The \ac{} spectrum as a function of the \ttbar{} invariant mass \mtt{}
with the additional requirement on the $z$--component of high velocity
of the \ttbar{} system, $\betatt{}>0.6$, is also shown in
Fig.~\ref{fig:unfac_diff}. The enhancement of the $\qqbar{}\to\ttbar{}$
process results in larger asymmetries for both SM and new physics
predictions.

A comparison of the inclusive \ac{} measurements, together with the Tevatron
\afb{} ones, with predictions from a broad range of new physics model
(see Sec.~\ref{sec:bsmca}) is shown in Fig.~\ref{fig:summarybsm}. The
\ac{} and \afb{} measurements, inclusive and at high \mtt{} mass,
define a region where the new physics prediction are compatible with
the current measurements from both Tevatron and LHC experiments.
Some models show only a limited range of parameter values which yield acceptable
asymmetries.

\begin{figure}[!htb]
  \centering
  \includegraphics[width=0.495\textwidth]{figures/results/bsm}
  \includegraphics[width=0.495\textwidth]{figures/results/bsm_highmtt}
  \caption{Measured forward--backward asymmetries \afb{} at Tevatron
    and charge asymmetries \ac{} at LHC, compared with the SM
    predictions (black box) and values in allowed new physics
    scenarios. The horizontal bands and lines correspond to the ATLAS
    (light green) and CMS (dark green) measurements, while the
    vertical ones correspond to the CDF (orange) and D0 (yellow)
    measurements. The inclusive asymmetry measurements are shown in
    the left plot. The right plot shows a comparison with the \afb{}
    measurement by CDF for $\mtt{} > 450 \GeV{}$ and the ATLAS \ac{}
    measurement for $\mtt{} > 600 \GeV{}$.}
  \label{fig:summarybsm}
\end{figure}


\section{\ac{} measurements at \eighttev{}}
\label{sec:res8tev}

The measurements at \eighttev{} are performed using six sub-samples of
$\ell{}+\ge4$ jets events, where, simultaneously, the background normalizations are
fitted and the parton--level \dy{} distribution is
estimated. Fig.~\ref{fig:dy7tev} shows the fitted distribution used
for the inclusive \ac{} measurement compared with the observed one. 

\begin{figure}[!htb]\centering
  \includegraphics[width=0.6\textwidth]{figures/results/datamcinclu8TeV_postfit} 
  \caption{}
  \label{fig:dy7tev}
\end{figure}

The asymmetry \ac{} in \ttbar{} production at \eighttev{} is measured to be
$\ac{}= 0.011\pm0.005$, compatible with the SM prediction
$\ac{}=0.0111\pm0.0004$.

Since the background estimation is part of the bayesian inference
procedure described in Sec.~\ref{sec:unfolding}, it's not possible to
study the impact of systematic uncertainties by repeating unfolding on data with
varied templates, without using marginalization. Instead, the expected
impact of systematic uncertainties is studied with pseudo-data distributions
corresponding to the sum of the background and signal predictions.
For each source of uncertainty, the $\pm{}1\sigma$ variations of the
predictions are used to build the pseudo--data, and the unfolding
procedure is repeated. The baseline background templates and response
matrices, as in the actual measurements, are used.
Table~\ref{tab:8tevsystematics} shows the average asymmetry variation
$\delta{\ac{}}$ computed, for each source of uncertainty, as
$|\ac{}(+1\sigma)-\ac{}(-1\sigma)|/2$.
As in the measurements at \seventev{}, the precision is dominated by
the statistical uncertainty, and the sources of largest systematic
uncertainties are the ones with a large impact on the size of the
\wjets{} background, such as the uncertainty on the energy scale and
resolution of lepton and jets. 

\begin{table}[!htb]\centering
\begin{tabular}{l c}
\toprule
Source of systematic uncertainty  & $\delta{\ac{}}$ \\
\midrule
Lepton reconstruction/identification    & $<0.0005$\\
Lepton energy scale and resolution      &  $0.0007$\\
Jet reconstruction efficiency                 &  $<0.0005$ \\
Jet energy scale and resolution             &  $0.0017$ \\
Missing transverse momentum and pile--up modeling& $<0.0005$\\
Multi--jets background normalization & $<0.0005$\\
Other backgrounds normalization        & $<0.0005$\\
$b$--tagging/mis--tag efficiency       & $<0.0005$\\
Signal modeling                                    & $??$\\
Parton shower/hadronization                & $??$\\
Monte Carlo statistics                            & $0.0008$\\
PDF                                                        &$0.0006$\\
Unfolding response                               &$<0.0005$\\
\midrule
Statistical uncertainty                           & $0.005$ \\
\midrule
Total uncertainty                                   & $0.005$ \\
\bottomrule
\end{tabular}
\caption{Expected impact of individual sources of uncertainty on the
  measured \ac{}. Variations below $10\%$ of the
  statistical uncertainty are considered negligible.}
\label{tab:8tevsystematics}
\end{table}

The asymmetry spectrum as a function of
\mtt{} is compared in Fig.~\ref{fig:8tevacvsmtt}
with the theoretical SM predictions and found to be compatible. 

\begin{figure}[!htb]\centering
  \includegraphics[width=0.6\textwidth]{figures/results/acvsmtt8tev}
  \caption{Measured \ac{} spectrum as a function of \mtt{} compared with SM prediction.}
  \label{fig:8tevacvsmtt}
\end{figure}

The asymmetry values corresponding to the spectra in
Fig.~\ref{fig:8tevacvsmtt} are detailed in
App.~\ref{app:differentialresults}, together with the breakdown of the
systematic uncertainties (App.~\ref{app:systematics}) and the
correlation matrix (App.~\ref{app:correlations}).

