\chapter{Results}
\label{sec:results}

This section presents the unfolded \ac{} measurements at \seventev{}
and \eighttev{}, both inclusive and differential in bins of \mtt{},
\pttt{} and \ytt{}. The measurements at \seventev{} have been
published~\cite{Aad:2013cea}, while the measurements at \eighttev{}
are internal results reported here for the purpose of this
dissertation. The corresponding publication results are being
finalized and are expected before the end of 2014. The results are
compared with the corresponding measurements from CMS, and the
results of the combination between ATLAS and CMS inclusive
measurements at \seventev{} are summarized.

\section{\ac{} measurements at \seventev{}}
\label{sec:res7tev}

The measurements are performed in the $\ell{}+\ge4$ jets sample with
at least 1 $b$--jet. The observed \dy{} distribution in data and the
background predictions used as input for the inclusive measurement is
shown in Fig.~\ref{fig:dy7tev}.

\begin{figure}[!htb]\centering
  \includegraphics[width=0.6\textwidth]{figures/results/dy7tev} 
  \caption{Observed \dy{} distributions for the $\ell{}+\ge4$ jets
    sample with at least 1 $b$--jet. Data (dots) and SM expectations
    (solid lines) are shown. The uncertainty on the total prediction
    includes both the statistical and the systematic components.}
  \label{fig:dy7tev}
\end{figure}

The inclusive \ac{} in \ttbar{} production at \seventev{} is measured to be
$\ac{}= 0.006\pm0.010(stat.+syst.)$, compatible with the SM prediction,
$\ac{}=0.0123\pm0.0005$ (see Sec.~\ref{sec:topca}).
Table~\ref{tab:results} summarizes the
measurements and predictions for the inclusive asymmetry, at high
invariant mass of the \ttbar{} pair ($\mtt{} >600 \GeV{})$, and at
high $z$--component of the velocity of the \ttbar{} system ($\betatt{}>0.6$).
The quoted uncertainties correspond to the RMS of the marginal posterior
probability densities $\conditionalProb{\ac{}}{\Data{}}$ described in
Sec.~\ref{sec:marginalization}; therefore it accounts for both
statistical and systematic components.

\begin{table}[!htb]\centering
\begin{tabular}{c c c}
\toprule
  & Data & Theory\\
\midrule
 \ac{}                                     & $0.006\pm0.010$ & $0.0123\pm0.0005$ \\
 $\ac{}(\beta_{z,\ttbar}>0.6)$   & $0.011\pm0.018$ & $0.020^{+0.006}_{-0.007}$\\
 $\ac{}(\mtt{} > 600 \GeV{})$ & $0.018\pm0.022$ & $0.0175^{+0.0005}_{-0.0004}$\\
\bottomrule
\end{tabular}
\caption{Measured inclusive \ac{} compared with SM
  predictions. The measurements with the \qqbar{}--enhancing
  $\beta_{z,\ttbar} > 0.6$ requirement and at high \ttbar{} invariant
  mass $\mtt{} > 600 \GeV{}$ are also included. The uncertainties
  include statistical and systematic components.}
\label{tab:results}
\end{table}

In order to study the impact on \ac{} of each systematic uncertainty,
the unfolding procedure is performed for each variation without using
marginalization. 
For each source of uncertainty, the background templates and
response matrix corresponding to $\pm1\sigma$ variation are
considered. 
Table~\ref{tab:systematics} shows the average asymmetry variations
$\delta\ac{}$ computed, for each source of uncertainty, as
$|\ac{}(+1\sigma)-\ac{}(-1\sigma)|/2$. 
While the precision is dominated by the statistical uncertainty, the main
sources of systematic uncertainty are the ones with a large
impact on the size of the \wjets{} background, such as the
uncertainty on the jet energy scale and resolution. 

\begin{table}[!htb]\centering
\begin{tabular}{l c c c}
\toprule
Source of systematic uncertainty  & \multicolumn{2}{c}{$\delta{\ac{}}$} \\
\midrule
  & Inclusive & $\mtt{} >600\GeV{}$ & $\betatt{}>0.6$ \\
\midrule
Lepton reconstruction/identification    & $<0.001$ & $<0.002$ & $<0.002$\\
Lepton energy scale and resolution      & $0.003$    & $0.003$   &  $0.003$\\
Jet energy scale and resolution             & $0.003$    & $0.003$   &
$0.005$ \\
Missing transverse momentum and pile--up modeling & $0.002$ & $0.002$
& $0.004$\\
Multi--jets background normalisation    & $<0.001$ & $<0.002$ & $<0.002$\\
$b$--tagging/mis--tag efficiency         & $<0.001$  & $<0.002$ & $<0.002$\\
Signal modeling                                       & $<0.001$ &
$<0.002$ & $<0.002$\\
Parton shower/hadronization             & $<0.001$  & $<0.002$ & $<0.002$\\
Monte Carlo statistics                         & $0.002$  & $<0.002$ & $<0.002$\\
PDF                                                      & $0.001$  & $<0.002$  & $<0.002$\\
$W$+jets normalisation and shape     & $0.002$  & $<0.002$ & $<0.002$\\
\midrule
Statistical uncertainty                          & $0.010$ & $0.021$ & $0.017$\\
\bottomrule
\end{tabular}
\caption{Impact of individual sources of uncertainty on the inclusive
  \ac{}, \mbox{$\ac{}(\betatt{}>0.6)$} and \mbox{$\ac{}(\mtt{}>600
    \GeV{})$} at \seventev{}. Variations below $10\%$ of the
  statistical uncertainty, quoted as ``$<0.001$'' and ``$<0.002$'',
  are considered negligible.}
\label{tab:systematics}
\end{table}

The \ac{} differential measurements as functions of
\mtt{}, \pttt{} and \ytt{} are compared in Fig.~\ref{fig:unfac_diff}
with the SM predictions and found to be consistent. 
In addition, the results are compared with new physics predictions for
color octets ({\it axigluons}) with masses of $m=300 \GeV{}$ and $7
\TeV{}$~\cite{AguilarSaavedra:2011ci}. Both models would not
be observable in direct searches as \ttbar{} resonances: the light
axigluon mass is below the threshold for \ttbar{} production, while
the one of $m=7 \TeV$ is beyond the kinematic reach of the current LHC
searches at \seventev{} and \eighttev{}. In both scenarios the new
physics couplings with top and light quarks are tuned to yield a
forward--backward asymmetry compatible with the results reported by
Tevatron experiments.
The \ac{} spectrum as a function of the \ttbar{} invariant mass \mtt{}
with the additional requirement on the $z$--component of high velocity
of the \ttbar{} system, $\betatt{}>0.6$, is also shown in
Fig.~\ref{fig:unfac_diff}. The enhancement of the $\qqbar{}\to\ttbar{}$
process results in larger asymmetries for both SM and new physics
predictions.
\begin{figure}[!htb]\centering
  \subfloat[][]
  {  \includegraphics[width=0.45\textwidth]{figures/results/dy_mtt}
  } \quad
  \subfloat[][]
  {  \includegraphics[width=0.45\textwidth]{figures/results/dy_pttt} } \\
  \subfloat[][]
  {  \includegraphics[width=0.45\textwidth]{figures/results/dy_ytt}
  } \quad
  \subfloat[][]
  {  \includegraphics[width=0.45\textwidth]{figures/results/dy_mtt_beta} }
  \caption{Measured \ac{} spectra compared with predictions for SM and
    for a color--octet axigluon with a mass of 300 GeV (red lines) and
    7000 GeV (blue lines) 
    as functions of \mtt{} (a), \pttt{} (b) and \ytt{} (c). The
    asymmetry as a function of \mtt{} with $\betatt{}>0.6$ is also
    shown (d). The uncertainties include statistical and systematic
    components.}
  \label{fig:unfac_diff}
\end{figure}
The asymmetry values corresponding to the spectra in
Fig.~\ref{fig:unfac_diff} are detailed in
Table~\ref{tab:results7tev}.
\begin{table}[!htp]\centering
\begin{tabular}{l c c c c c }
  \toprule
  &\multicolumn{5}{c}{$\mtt{}~[\rm{GeV}]$}    \\
  \ac{} & $0$--$420$ & $420$--$500$ & $500$--$600$ & $600$--$750$ & $>750$ \\
  \midrule
  Data & $0.036\pm0.055$ & $0.003\pm0.044$ & $-0.039\pm0.047$ & $0.044\pm0.054$ & $0.011\pm0.054$ \\
  Theory & $0.0103^{+0.0003}_{-0.0004}$ & $0.0123^{+0.0006}_{-0.0003}$ & $0.0125\pm0.0002$ & $0.0156^{+0.0007}_{-0.0009}$ & $0.0276^{+0.0004}_{-0.0008}$ \\
  \bottomrule
\end{tabular}
\caption{ }
\label{tab:results_mtt}
\end{table}
%
\begin{table}[!htp]\centering
\caption{ }
\begin{tabular}{l c c c }
  \toprule
  &\multicolumn{3}{c}{$\pttt{}~[\rm{GeV}]$}    \\
  \ac{} &     $0$--$25$           &        $25$--$60$         &  $>60$            \\
  \midrule
  Data  &    $-0.032\pm0.052$       &     $0.067\pm0.057$       &      $-0.034\pm0.034$     \\
  Theory & $0.0160^{+0.0007}_{-0.0009}$ & $-0.0058^{+0.0004}_{-0.0004}$ & $-0.0032^{+0.0002}_{-0.0002}$  \\
  \bottomrule
\end{tabular}
\label{tab:results_pttt}
\end{table}
%
\begin{table}[!htp]\centering
\caption{ }
\begin{tabular}{l c c c }
  \toprule
  &\multicolumn{3}{c}{$\ytt{}$}    \\
  \ac{}  &     $0$--$0.3$          &        $0.3$--$0.7$     &  $>0.7$            \\
  \midrule
  Data  &    $-0.010\pm0.043$       &     $0.006\pm0.031$     &      $0.015\pm0.025$     \\
  Theory & $0.0026^{+0.0008}_{-0.0001}$ & $0.0066^{+0.0001}_{-0.0003}$ & $0.0202^{+0.0006}_{-0.0007}$ \\
  \bottomrule
\end{tabular}
\label{tab:results_ytt}
\end{table}
%
\begin{table}[!htp]\centering
\caption{ }
\begin{tabular}{l c c c c c }
  \toprule
  &\multicolumn{5}{c}{$\mtt{}~[\rm{GeV}]$ for $\betatt{} > 0.6$}    \\
  \ac{} & $0$--$420$ & $420$--$500$ & $500$--$600$ & $600$--$750$ & $>750$ \\
  \midrule
  Data  & $0.054 \pm  0.079 $ & $0.008 \pm 0.072  $ & $-0.022 \pm 0.075 $ & $-0.019 \pm 0.102    $ & $0.205 \pm 0.135$ \\
  Theory & $0.0145^{+0.0005}_{-0.0003}$ & $0.0213^{+0.0006}_{-0.0005}$ & $0.0240^{+0.0003}_{-0.0009}$ & $0.0280^{+0.0012}_{-0.0007}$ & $0.0607 \pm 0.0002$ \\
  \bottomrule
\end{tabular}
\label{tab:results_mtt_beta}
\end{table}

The correlation of the asymmetry values $\ac{}^i$ across differential
bins is fully described by the posterior probability density
$\conditionalProb{\ac{}^1,...,\ac{}^N}{\Data{}}$, where $N$ is the
number of differential bins. Table~\ref{tab:corr_mtt} shows the
correlation coefficients $\rho_{ij}$ for the differential \ac{}
measurement as a function of \mtt{}.
\begin{table}[!htp]\centering
\begin{tabular}{l c c c c c }
  \toprule
  &\multicolumn{5}{c}{$\mtt{}~[\rm{GeV}]$}    \\
  $\rho_{ij}$ & $0$--$420$ & $420$--$500$ & $500$--$600$ & $600$--$750$ & $>750$ \\
  \midrule
  $0$--$420$      & $1$ & $-0.38$ & $0.13$  & $-0.05$ & $0.01$  \\
  $420$--$500$    &     & $1$     & $-0.53$ & $0.17$  & $-0.03$ \\
  $500$--$600$    &     &         & $1$     & $-0.54$ & $0.14$  \\
  $600$--$750$    &     &         &         & $1$     & $-0.43$ \\
  $>750$          &     &         &         &         & $1$     \\
  \bottomrule
\end{tabular}
\caption{Correlation coefficients for the \ac{} values measured as a
  function of \mtt{} at \seventev{}.} 
\label{tab:corr_mtt}
\end{table}
The corresponding tables for the other differential measurements are
collected in App.~\ref{app:correlations}.

\subsection{Comparison of ATLAS and CMS results}

Analogous measurements of the asymmetry \ac{} at \seventev{} have been
performed by the CMS collaboration~\cite{Chatrchyan:2012cxa}, who
reported an inclusive asymmetry
$\ac{}=0.004\pm0.010(stat.)\pm0.011(syst.)$.
The results are compatible with the SM predictions and with the
measurements presented in this thesis. 
With about 46000 \ttbar{} events, CMS sample has a similar size to the
ATLAS one ($\sim50000$ events); thus the statistical uncertainties are
of similar size. On the other hand systematic uncertainties in the CMS
measurement are significantly larger, due to the impact of the signal
modeling used in the unfolding algorithm (SVD).

The ATLAS and CMS inclusive \ac{} results are combined in order to
obtain a more precise measurement~\cite{ATLAS-CONF-2014-012}. The
combination is performed using the Best Linear Unbiased Estimate
(BLUE)~\cite{Lyons:1988rp,Valassi:2003mu}, where the correlation of
systematic uncertainties is taken into account. In particular, signal
modeling and PDF uncertainties are considered fully correlated between
ATLAS and CMS results, while a 50\% correlation is assigned to the
\wjets{} normalization uncertainty, as data (uncorrelated) and
simulation (correlated) are both used to estimate the this
background. All uncertainties related with the detector description
are instead considered not correlated. Since the ATLAS results is
dominated by the statistical uncertainty (uncorrelated), the combined
results is insensitive to variations of these choices. The combined
result is found to be $0.005\pm0.007(stat.)\pm0.006(syst.)$,
compatible with the SM prediction. The comparison between values of
\ac{} as measured by ATLAS and CMS, and their combination is shown in
Fig.~\ref{fig:combineac}.
\begin{figure}[!htb]\centering
  \includegraphics[width=0.45\textwidth]{figures/results/combineac7tev} 
  \caption{Summary of the single \ac{} measurements and the LHC
    combination compared to the theory prediction (calculated at NLO
    including electroweak corrections). The inner red error bars
    indicate the statistical uncertainty, the blue outer error bars
    indicate the total uncertainty. The grey band illustrates the
    total uncertainty of the combined result.} 
  \label{fig:combineac}
\end{figure}
The combination of the differential measurements is not possible, as a
different binning has been used in each case.

\subsection{Interpretation}

A comparison of \ac{} measurements at the LHC and \afb{} measurements
at the Tevatron with predictions from a broad range of new physics models
(see Sec.~\ref{sec:bsmca}) is shown in Fig.~\ref{fig:summarybsm}. The
\ac{} and \afb{} measurements, inclusive and at high \mtt{} mass,
define a region where the new physics prediction are compatible with
the current measurements from both Tevatron and LHC experiments.
Some models show only a limited range of parameter values which yield
acceptable asymmetries. The wider spread of \ac{} values highlights
the discrimination power of measurements at high \mtt{} and the
important complementarity of \ac{} and \afb{}. 

\begin{figure}[!htb]
  \centering
  \includegraphics[width=0.495\textwidth]{figures/results/bsm}
  \includegraphics[width=0.495\textwidth]{figures/results/bsm_highmtt}
  \caption{Measured forward--backward asymmetries \afb{} at Tevatron
    and charge asymmetries \ac{} at LHC, compared with the SM
    predictions (black box) and values in allowed new physics
    scenarios. The horizontal bands and lines correspond to the ATLAS
    (light green) and CMS (dark green) measurements, while the
    vertical ones correspond to the CDF (orange) and D0 (yellow)
    measurements. The inclusive asymmetry measurements are shown in
    the left plot. The right plot shows a comparison with the \afb{}
    measurement by CDF for $\mtt{} > 450 \GeV{}$ and the ATLAS \ac{}
    measurement for $\mtt{} > 600 \GeV{}$.}
  \label{fig:summarybsm}
\end{figure}


\section{\ac{} measurements at \eighttev{}}
\label{sec:res8tev}

As discussed in Sec.~\ref{sec:wjets}, the measurements at \eighttev{}
are performed using six sub-samples of $\ell{}+\ge4$ jets events,
where, simultaneously, the background normalizations are fitted and
the parton--level \dy{} distribution is
estimated. Figure~\ref{fig:dy8tev} shows the comparison between
prediction and data for the 18 bins used in the inclusive \ac{}
measurement before and after the simultaneous unfolding procedure and
background calibration.
\begin{figure}[!htb]\centering
  \includegraphics[width=0.6\textwidth]{figures/results/datamc_muonele_prefit} 
  \includegraphics[width=0.6\textwidth]{figures/results/datamc_muonele_postfit} 
  \caption{Comparison between prediction and data for the 18 bins used
  in the inclusive \ac{} measurement before (top) and after (bottom)
  the simultaneous unfolding procedure and background calibration. The
  \dy{} distribution in 4 bins is considered for the
  \ttbar{}--enriched event samples with 1 and at least 2 $b$--jets; a
  single bin is considered for the background--enriched sample with 0
  $b$--jets. After the calibration, the background components are
  normalized to the measured values for the nuisance parameters, and
  the prediction for \ttbar{} events in each bin is estimated by
  folding the measured parton--level parameters through the response
  matrix.}
  \label{fig:dy8tev}
\end{figure}
The inclusive \ac{} in \ttbar{} production at \eighttev{} is measured to be
$\ac{}= 0.011\pm0.006(stat.+syst.)$, compatible with the SM prediction,
$\ac{}=0.0111\pm0.0004$. The improvement in precision with respect to
the measurement at \seventev{} is due to the reduction of the
statistical uncertainty that, with a \ttbar{} sample six times larger,
is of $0.005$. 

Since the background estimation is part of the bayesian inference
procedure described in Sec.~\ref{sec:unfolding}, it's not possible to
study the impact of systematic uncertainties by repeating unfolding on data with
varied templates, without using marginalization. Instead, the expected
impact of systematic uncertainties is studied with pseudo-data distributions
corresponding to the sum of the background and signal predictions.
For each source of uncertainty, the $\pm{}1\sigma$ variations of the
predictions are used to build the pseudo--data, and the unfolding
procedure is repeated. The baseline background templates and response
matrices, as in the actual measurements, are used.
Table~\ref{tab:8tevsystematics} shows the average asymmetry variation
$\delta{\ac{}}$ computed, for each source of uncertainty, as
$|\ac{}(+1\sigma)-\ac{}(-1\sigma)|/2$.
As in the measurements at \seventev{}, the precision is dominated by
the statistical uncertainty, and the main sources of systematic
uncertainty are the signal modeling, possibly affected by the limited
MC statistics, and the ones with a large impact on the size of the
\wjets{} background, such as the uncertainty on the jet energy scale
and resolution. 

\begin{table}[!htb]\centering
\begin{tabular}{l c}
\toprule
Source of systematic uncertainty  & $\delta{\ac{}}$ \\
\midrule
Lepton reconstruction/identification    & $<0.0005$\\
Lepton energy scale and resolution      &  $0.0007$\\
Jet reconstruction efficiency                 &  $<0.0005$ \\
Jet energy scale and resolution             &  $0.0017$ \\
Missing transverse momentum and pile--up modeling& $<0.0005$\\
Multi--jets background normalization & $<0.0005$\\
Other backgrounds normalization        & $<0.0005$\\
$b$--tagging/mis--tag efficiency       & $<0.0005$\\
Signal modeling                                    & $0.0032$\\
Parton shower/hadronization                & $0.0008$\\
Monte Carlo statistics                            & $0.0008$\\
PDF                                                        &$0.0005$\\
Unfolding response                               &$<0.0005$\\
\midrule
Statistical uncertainty                           & $0.005$ \\
\midrule
Total uncertainty                                   & $0.006$ \\
\bottomrule
\end{tabular}
\caption{
  Impact of individual sources of uncertainty on the inclusive
  \ac{} at \eighttev{}. Variations below $10\%$ of the statistical
  uncertainty are quoted as ``$<0.0005$'' and considered negligible.}
\label{tab:8tevsystematics}
\end{table}

The asymmetry spectrum as a function of \mtt{} is compared in
Fig.~\ref{fig:8tevacvsmtt} with the theoretical SM predictions.
As for the \seventev{} measurements, the results are compared with new
physics predictions for right--handed color octets with low and high
masses~\cite{Aguilar-Saavedra:2014nja}. The new physics models are
tuned to be compatible with the Tevatron \afb{} measurements and the
\ac{} measurements at \seventev{}.
\begin{figure}[!htb]\centering
  \includegraphics[width=0.6\textwidth]{figures/results/acvsmtt8tev}
  \caption{Measured \ac{} spectrum as a function of \mtt{} at
    \seventev{} compared with predictions for SM (green lines) and for right--handed
    color octets with masses below the \ttbar{} threshold (red lines) and beyond
    the kinematic reach of current LHC searches (blue line).}
  \label{fig:8tevacvsmtt}
\end{figure}
The measured \ac{} values are summarized in
Table~\ref{tab:results_mtt_8tev} and the correlation coefficients are
shown in Table~\ref{tab:corr_mtt8tev}.
\begin{table}[!htp]\centering
\scalebox{0.9}{
\begin{tabular}{l c c c c c c}
  \toprule
  &\multicolumn{6}{c}{$\mtt{}~[\rm{GeV}]$}    \\
  \ac{} & $0$--$420$ & $420$--$500$ & $500$--$600$ & $600$--$750$ & $750$--$900$ & $>900$ \\
  \midrule
  Data & $0.020\pm0.0265$ & $-0.008\pm0.018$ & $0.024\pm0.020$ & $0.008\pm0.025$ & $-0.01\pm0.05$ & $0.08\pm0.04$\\
  Theory & $0.0081^{+0.0003}_{-0.0004}$ & $0.0112\pm0.0005$ & $0.0114^{+0.0003}_{-0.0004}$ & $0.0134^{+0.0003}_{-0.0005}$ & $0.0167^{+0.0005}_{-0.0006}$ & $0.02100^{+0.00032}_{-0.00018}$ \\
  \bottomrule
\end{tabular}
}
\caption{ }
\label{tab:results_mtt_8tev}
\end{table}

%
\begin{table}[!htp]\centering
\begin{tabular}{l c c c c c c}
  \toprule
  &\multicolumn{6}{c}{$\mtt{}~[\rm{GeV}]$}    \\
  $\rho_{ij}$ & $0$--$420$ & $420$--$500$ & $500$--$600$ &
  $600$--$750$ & $750$--$900$ & $>900$ \\
  \midrule
  $0$--$420$      & $1$ & $-0.270$ & $0.117$ & $-0.061$ & $0.008$ & $-0.002$ \\
  $420$--$500$    &     & $1$           & $-0.596$ & $0.217$ &
  $-0.053$ & $0.012$ \\
  $500$--$600$    &     &                   & $1$          & $-0.590$
  & $0.175$ & $-0.036$ \\
  $600$--$750$    &     &         &         & $1$     & $-0.578$ & $0.132$ \\
  $750$--$900$    &     &         &         &            & $1$ & $-0.497$ \\
  $>900$               &     &         &         &         &   &$1$     \\
  \bottomrule
\end{tabular}
\caption{Correlation coefficients for the \ac{} values measured as a
  function of \mtt{} at \eighttev{}.}
\label{tab:corr_mtt8tev}
\end{table}

\subsection{Comparison of ATLAS and CMS results}

Preliminary results at \eighttev{} from the CMS
collaboration~\cite{CMS-PAS-TOP-12-033} include the measurement of
\ac{}, inclusively and as a function of \mtt{}, \pttt{} and \ytt{}.
The reported inclusive asymmetry
\mbox{$\ac{}=0.005\pm0.007(stat.)\pm0.006(syst.)$} is compatible with the SM
predictions and with the measurement presented in this dissertation
with a total uncertainty about twice as large. Both
statistical and systematic uncertainties are larger than the ones
reported in this work. The statistical precision is presumably
degraded by the use of a large number of \dy{} bins in the unfolding
procedure, while the larger impact of all systematic components
points to a less precise background estimation.

%The asymmetry values corresponding to the spectra in
%Fig.~\ref{fig:8tevacvsmtt} are detailed in
%App.~\ref{app:differentialresults}, together with the breakdown of the
%systematic uncertainties (App.~\ref{app:systematics}) and the
%correlation matrix (App.~\ref{app:correlations}).

