\chapter{Results}
\label{sec:results}


\section{\ac{} measurements at \seventev{}}
\label{sec:res7tev}

The asymmetry \ac{} in \ttbar{} production at \seventev{} is measured to be
$\ac{}= 0.006\pm0.010$, compatible with the SM prediction
$\ac{}=0.0123\pm0.0005$.
Table~\ref{tab:results} summarizes the
measurements and predictions for the inclusive asymmetry, at high
invariant mass of the top quark pair ($\mtt{} >600 \GeV{})$, and at
high $z$--component of the velocity of the \ttbar{} system ($\betatt{}>0.6$).
The quoted uncertainty corresponds to the RMS of the marginal posterior
probability density $\conditionalProb{\ac{}}{\Data{}}$ described in
Sec.~\ref{sec:marginalization}; therefore it accounts for both
statistical and systematic components.

\begin{table}[!htb]\centering
\begin{tabular}{c c c}
\toprule
  & Data & Theory\\
\midrule
 \ac{}                                     & $0.006\pm0.010$ & $0.0123\pm0.0005$ \\
 $\ac{}(\mtt{} > 600 \GeV{})$ & $0.018\pm0.022$ & $0.0175^{+0.0005}_{-0.0004}$\\
 $\ac{}(\beta_{z,\ttbar}>0.6)$   & $0.011\pm0.018$ & $0.020^{+0.006}_{-0.007}$\\
\bottomrule
\end{tabular}
\caption{Measured charge asymmetry \ac{} compared with SM
  predictions. The measurements with the \qqbar{}--enhancing
  $\beta_{z,\ttbar} > 0.6$ requirement and at high \ttbar{} invariant
  mass $\mtt{} > 600 \GeV{}$ are also included.}
\label{tab:results}
\end{table}

In order to study the impact on \ac{} of each systematic uncertainty,
the unfolding procedure is performed for each variation without using
marginalization. 
For each source of uncertainty, the background templates and
response matrix corresponding to $\pm1\sigma$ variation are
considered. 
Table~\ref{tab:systematics} shows the average variations in the
asymmetry computed as $|\ac{}(+1\sigma)-\ac{}(-1\sigma)|/2$.
While the precision is dominated by the statistical uncertainty, the
sources of largest systematic uncertainties are the ones with a large
impact on the size of the \wjets{} background, such as the
uncertainty on the energy scale and resolution of lepton and jets. 

\begin{table}[!htb]\centering
\begin{tabular}{l c c c}
\toprule
Source of systematic uncertainty  & \multicolumn{2}{c}{$\delta{\ac{}}$} \\
\midrule
  & Inclusive & $\mtt{} >600\GeV{}$ & $\betatt{}>0.6$ \\
\midrule
Lepton reconstruction/identification    & $<0.001$ & $<0.002$ & $<0.002$\\
Lepton energy scale and resolution      & $0.003$    & $0.003$   &  $0.003$\\
Jet energy scale and resolution             & $0.003$    & $0.003$   &
$0.005$ \\
Missing transverse momentum and pile--up modeling & $0.002$ & $0.002$
& $0.004$\\
Multi--jets background normalisation    & $<0.001$ & $<0.002$ & $<0.002$\\
$b$--tagging/mis--tag efficiency         & $<0.001$  & $<0.002$ & $<0.002$\\
Signal modeling                                       & $<0.001$ &
$<0.002$ & $<0.002$\\
Parton shower/hadronization             & $<0.001$  & $<0.002$ & $<0.002$\\
Monte Carlo statistics                         & $0.002$  & $<0.002$ & $<0.002$\\
PDF                                                      & $0.001$  & $<0.002$  & $<0.002$\\
$W$+jets normalisation and shape     & $0.002$  & $<0.002$ & $<0.002$\\
\midrule
Statistical uncertainty                          & $0.010$ & $0.021$ & $0.017$\\
\bottomrule
\end{tabular}
\caption{Impact of individual sources of uncertainty on the measured
  \ac{} and $\ac{}(\mtt{}>600 \GeV{})$. Variations below $10\%$ of the
  statistical uncertainty are considered negligible.}
\label{tab:systematics}
\end{table}

The asymmetry spectra, measured as functions of
\mtt{}, \pttt{} and \ytt{} are compared in Fig.~\ref{fig:unfac_diff}
with the theoretical SM predictions and found compatible. 
In addition, the results are compared with new physics predictions for
color octets ({\it axigluons}) with masses $M_G=300 \GeV{}$ and $M_G=7
\TeV{}$~\cite{Ferrario:2009bz,Frampton:2009rk}. Both models would not
be observable in direct searches as \ttbar{} resonances: the light
axigluon mass is below the threshold for production of top quark
pairs, while a $7 \TeV$ mass is beyond the kinematic reach of the
current measurements at \seventev{} and \eighttev{}. In both scenarios
the new physics couplings with top and light quarks are tuned to yield a
forward--backward asymmetry compatible with the results reported by
Tevatron experiments.

\begin{figure}[!htb]\centering
  \subfloat[][]
  {  \includegraphics[width=0.45\columnwidth]{figures/results/dy_mtt}
  } \quad
  \subfloat[][]
  {  \includegraphics[width=0.45\columnwidth]{figures/results/dy_pttt} } \\
  \subfloat[][]
  {  \includegraphics[width=0.45\columnwidth]{figures/results/dy_ytt}
  } \quad
  \subfloat[][]
  {  \includegraphics[width=0.45\columnwidth]{figures/results/dy_mtt_beta} }
  \caption{Measured \ac{} spectra compared with SM and bSM predictions
    as functions of \mtt{} (a), \pttt{} (b) and \ytt{} (c). The
    asymmetry as a function of \mtt{} with $\betatt{}>0.6$ is also
    shown (d).}
  \label{fig:unfac_diff}
\end{figure}

The asymmetry values corresponding to the spectra in
Fig.~\ref{fig:unfac_diff} are detailed in Tables~\ref{tab:results_mtt},
\ref{tab:results_pttt} and \ref{tab:results_ytt}.
The correlation of the asymmetry values $\ac{}^i$ across differential bins is
fully described by the posterior probability density
$\conditionalProb{\ac{}^1,...,\ac{}^{N_{diff}}}{\Data{}}$, where
$N_{diff}$ is the number of differential bins, and it is summarized in
Tables~\ref{tab:corr_mtt}, ~\ref{tab:corr_pttt} and 
\ref{tab:corr_ytt} for each measurement.
The \ac{} spectrum as a function of the \ttbar{} invariant mass \mtt{}
with the additional requirement on the $z$--component of high velocity
of the \ttbar{} system, $\betatt{}>0.6$, is also shown in
Fig.~\ref{fig:unfac_diff}, while Table~\ref{tab:results_mtt_beta}
shows the values of \ac{} for each differential bin, and their
correlation is reported in Table~\ref{tab:corr_mtt_beta}.

\begin{table}[!htp]\centering
\begin{tabular}{l c c c c c }
  \toprule
  &\multicolumn{5}{c}{$\mtt{}~[\rm{GeV}]$}    \\
  \ac{} & $0$--$420$ & $420$--$500$ & $500$--$600$ & $600$--$750$ & $>750$ \\
  \midrule
  Data & $0.036\pm0.055$ & $0.003\pm0.044$ & $-0.039\pm0.047$ & $0.044\pm0.054$ & $0.011\pm0.054$ \\
  Theory & $0.0103^{+0.0003}_{-0.0004}$ & $0.0123^{+0.0006}_{-0.0003}$ & $0.0125\pm0.0002$ & $0.0156^{+0.0007}_{-0.0009}$ & $0.0276^{+0.0004}_{-0.0008}$ \\
  \bottomrule
\end{tabular}
\caption{ }
\label{tab:results_mtt}
\end{table}
%
\begin{table}[!htp]\centering
\caption{ }
\begin{tabular}{l c c c }
  \toprule
  &\multicolumn{3}{c}{$\pttt{}~[\rm{GeV}]$}    \\
  \ac{} &     $0$--$25$           &        $25$--$60$         &  $>60$            \\
  \midrule
  Data  &    $-0.032\pm0.052$       &     $0.067\pm0.057$       &      $-0.034\pm0.034$     \\
  Theory & $0.0160^{+0.0007}_{-0.0009}$ & $-0.0058^{+0.0004}_{-0.0004}$ & $-0.0032^{+0.0002}_{-0.0002}$  \\
  \bottomrule
\end{tabular}
\label{tab:results_pttt}
\end{table}
%
\begin{table}[!htp]\centering
\caption{ }
\begin{tabular}{l c c c }
  \toprule
  &\multicolumn{3}{c}{$\ytt{}$}    \\
  \ac{}  &     $0$--$0.3$          &        $0.3$--$0.7$     &  $>0.7$            \\
  \midrule
  Data  &    $-0.010\pm0.043$       &     $0.006\pm0.031$     &      $0.015\pm0.025$     \\
  Theory & $0.0026^{+0.0008}_{-0.0001}$ & $0.0066^{+0.0001}_{-0.0003}$ & $0.0202^{+0.0006}_{-0.0007}$ \\
  \bottomrule
\end{tabular}
\label{tab:results_ytt}
\end{table}
%
\begin{table}[!htp]\centering
\caption{ }
\begin{tabular}{l c c c c c }
  \toprule
  &\multicolumn{5}{c}{$\mtt{}~[\rm{GeV}]$ for $\betatt{} > 0.6$}    \\
  \ac{} & $0$--$420$ & $420$--$500$ & $500$--$600$ & $600$--$750$ & $>750$ \\
  \midrule
  Data  & $0.054 \pm  0.079 $ & $0.008 \pm 0.072  $ & $-0.022 \pm 0.075 $ & $-0.019 \pm 0.102    $ & $0.205 \pm 0.135$ \\
  Theory & $0.0145^{+0.0005}_{-0.0003}$ & $0.0213^{+0.0006}_{-0.0005}$ & $0.0240^{+0.0003}_{-0.0009}$ & $0.0280^{+0.0012}_{-0.0007}$ & $0.0607 \pm 0.0002$ \\
  \bottomrule
\end{tabular}
\label{tab:results_mtt_beta}
\end{table}


\section{\ac{} measurement at \eighttev{}}
\label{sec:res8tev}

