\chapter{Unfolding}
\label{sec:unfolding}

In order to allow a direct comparison between the experimental results
and the theoretical predictions, the parton level asymmetry spectra need
to be estimated.
As discussed in Sec.~\ref{sec:reconstruction}, the asymmetry computed
from the reconstructed kinematics of the \ttbar{} system is affected
by the efficiency of correctly reconstructing the \dy{} sign. The
distortion induced on the \ac{} values depends on the kinematic region
considered and it can dilute the asymmetry up to half of its original
({\it true}) value.
The \ttbar{} MC simulation is used to map the distortions,
bin--dependent {\it efficiencies} and {\it migrations}, caused by
acceptance and resolution effects. The {\it unfolding} procedure
consists in combining this information with the distributions observed
in data in order to estimate the parton level distributions.
In this chapter, the mathematical description of unfolding
is introduced along with a review of the commonly used algorithm.
The description of the novel algorithm developed and used in this
dissertation follows.

\section{Unfolding in particle physics}

The problem of de--convoluting resolution effects in measured
distributions has been studied for a long time in particle physics,
with the first publications on the topic dating back to the
1980's~\cite{Blobel:157405}. An introduction to unfolding with a
review of the most commonly used techniques can be found
in~\cite{Cowan:2002in} and is summarized in this section. 

\subsection{The mathematical problem}

The general problem consists in determining the distribution $f(y)$ of a
stochastic variable using a sample of data $y_1,\cdots{},y_N$.
The main complication arises from the fact that the measured values
$x$ of the observable $y$ are affected by random fluctuations due to
measurement errors. Therefore, each observation is characterized by a
true value $y$ and a measured value $x$, and the distributions of $x$
and $y$ are related by the convolution
\begin{equation}
\label{eq:convolution}
f_{meas}(x) = \int~\mathcal{M}(x|y)~f_{true}(y)~dy,
\end{equation}
where $\mathcal{M}(x|y)$ is the response function, describing the
probability for the true value $y$ to be measured as $x$.

Since no parametrization is available to describe the {\it true}
distribution $f_{true}(y)$, the corresponding {\it true} histogram
$\Truth{}=\{t_1,\cdots{},t_{N_t}\}$ with $N_t$ bins is
considered. Likewise, the {\it measured} distribution $f_{meas}(x)$ is
replaced by the {\it reconstructed} histogram $\Reco{}=\{r_1,\cdots{},r_{N_r}\}$ with
$N_r$ bins, and the response function $\mathcal{M}$ becomes the $N_t\times N_r$ {\it
  response matrix} \TrasfMatrix{}.
The $t_i$ and $r_i$ values represent the expectation values of the number of
entries in bin $i$ of the true and reconstructed histograms,
respectively, for which Eq.~\ref{eq:convolution} becomes the matrix product
\begin{equation}
\label{eq:matrixprod}
\Reco{} = \TrasfMatrix{}\Truth{}.
\end{equation}

The goal of unfolding is to construct estimators for the $N_t$
parameters $\Truth{}=\{t_1,\cdots{},t_{N_t}\}$, given a distribution
\Reco{} of measured values. The response matrix \TrasfMatrix{} is derived from
an arbitrary reference sample, typically derived from simulation.
In order to derive \Truth{} from Eq.~\ref{eq:matrixprod}, the response
matrix \TrasfMatrix{} has to be inverted. However, depending on the
problem at hand, the response matrix might not be
invertible. Unfolding algorithms typically adopt approximations in
order to ensure invertibility. In addition, the estimator obtained from
matrix inversion is unbiased and with maximal variance, as shown in
\cite{Cowan:2002in}.
Unfolding methods implement {\it regularization} procedures
that allow for more precise estimators at expense of introducing a bias.

\subsection{Unfolding algorithms}

Several unfolding techniques have been proposed and used in particle
physics. In this section, the most commonly used algorithms are
presented. In this dissertation, a different method is used, which is
described in detail in Sec.~\ref{sec:fbu}.

\paragraph{Bin--by--bin correction}
A simple method to overcome the invertibility problem consists on
constructing estimators for \Truth{} based on multiplicative
correction factors from MC simulation. The estimator for bin $i$ is
defined as
\begin{equation}
t_i = C_i~r_i,~{\rm with}~C_i=\frac{t_i^{MC}}{r_i^{MC}},
\end{equation}
where $C_i$ is the bin--by--bin correction factor derived from
simulation as the ratio of true over measured number of events in bin $i$.
It has been shown~\cite{Cowan:2002in} that this estimator, neglecting
to model off--diagonal migrations, is strongly biased towards the
simulated model used to derive the correction factors. 
\paragraph{Singular value decomposition}
In this method~\cite{Hocker:1995kb}, the inverted response matrix is
obtained by singular value decomposition. The $N_t\times N_r$ matrix
\TrasfMatrix{} is factorized as 
\begin{equation}
\TrasfMatrix{} = U~S~V^T,
\end{equation}
where $U$ is an $N_t\times N_t$ unitary matrix, $V$ is an $N_r\times
N_r$ unitary matrix, and $S$ is an $N_t\times N_r$ diagonal matrix
with non-negative eigenvalues $s_i$.
The matrices $U$ and $V$ are used to reduce Eq.~\ref{eq:matrixprod}
to a diagonal system of equations
\begin{equation}
\vect{d} = S~\vect{z},
\end{equation}
where $\vect{d}=U^T\Reco{}$ and $\vect{z}=V^T\Truth{}$. Thus, the
problem is reduced to the trivial inversion of a diagonal matrix.
However, the authors of the method have shown that, for very small
eigenvalues $s_i$, statistical fluctuations in \Reco{} are amplified giving
meaningless solutions for \Truth{}. The problem is addressed by
introducing a regularization that stabilizes the result by suppressing
non--physical fluctuations. In particular, the method implements a
Tikhonov regularization~\cite{Tikhonov1943stability}, which suppresses
results with high curvature (see Sec.~\ref{sec:fbuprior}).
\paragraph{Iterative bayesian unfolding}
A different approach to unfolding is taken in this
method~\cite{dagostini1995NIMPA}, where the estimators for \Truth{}
are constructed using Bayes' theorem:
\begin{equation}
t_i = N~p(t_i|\Reco{}) = N~\sum_j~\frac{p(r_j|t_i)~p(t_i)}{p(r_j)},
\end{equation} 
where the conditional probability $p(r_i|t_i)$ corresponds to the
elements of the response matrix \TrasfMatrix{}, $p(r_j)$ is the
fraction of events in bin $j$ of the measured distribution \Reco{},
and $p(t_i)$ is the prior assumption on the true bin content to be
estimated. At each iteration, $p(t_i)$ is updated with the estimated
$t_i$ value. The bias is introduced by the arbitrary choice of the
starting value for $p(t_i)$, while at each iteration the bias is
reduced, at expense of increasing the variance. For 0
iterations the output distribution is identical to the arbitrary
chosen $t_i$, thus maximally biased, while the unbiased matrix
inversion solution is obtained for infinite iterations. Typically, the
number of iteration is set by defining a convergence criterium. While
the methods exploits the bayesian theorem, it is not a bayesian
inference method, as it doesn't estimate the posterior probability
density for \Truth{}.

\section{Fully Bayesian Unfolding}
\label{sec:fbu}

The unfolding methods presented so far try to cope with the difficulty
of solving the inverse problem by introducing a bias. The algorithm
studied and implemented for this thesis provides an exact solution to
the unfolding problem by removing the need for inverting the response
matrix. The author of this dissertation developed and maintains an
open--source implementation of the algorithm~\cite{pyfbu} using the
popular Bayesian sampling toolkit
PyMC~\cite{Patil:Huard:Fonnesbeck:2010:JSSOBK:v35i04}. 

\subsection{Description of the method}

The Fully Bayesian Unfolding (FBU)~\cite{fbu} consists in the strict
application of Bayesian inference to the problem of unfolding. This
application can be stated in the following terms: given an observed
spectrum $\Data\in\Integer^{N_r}$ and a response matrix
$\TrasfMatrix\in\Real{}^{N_r}\times{}\Real{}^{N_t}$, the posterior
probability of the true spectrum $\Truth{}\in{}\Real{}^{N_t}$ follows
the probability density
\begin{equation}
\conditionalProb{\Truth{}}{\Data{},\TrasfMatrix{}}
\propto{}
\conditionalLhood{\Data{}}{\Truth{},\TrasfMatrix{}}
\cdot{}
\pi{}\left(\Truth{}\right)
\end{equation}
where \conditionalLhood{\Data{}}{\Truth{},\TrasfMatrix{}} is the
conditional likelihood for \Data{} given \Truth{} and \TrasfMatrix{},
and $\pi{}$ is the prior probability density for the true spectrum
\Truth{}.
Each of these ingredients is described in the following.

\subsubsection{Likelihood}
\label{sec:fbullhood}
Under the assumption that the data are poissonian counts, the
likelihood \conditionalLhood{\Data{}}{\Truth{},\TrasfMatrix{}} can be
computed from the following two pieces of information, contained in
the response matrix \TrasfMatrix{}:
\begin{itemize}
\item the probability $P(r_i|t_j)$ for an event produced in the true bin
  $t_j$ to be observed in the reconstructed bin $r_i$;
\item the efficiency $\epsilon{}_{t_j}$ for an event produced in the
  true bin $t_j$ to be reconstructed in any bin $r$.
\end{itemize}
The above quantities define the elements
$m_{ij}=\epsilon_{t_j}\cdot{}p(r_i|t_j)$ of \TrasfMatrix{}, and allow
the prediction of the reconstructed spectrum \Reco{} corresponding to
a given true spectrum \Truth{} as given by
\begin{equation}
r_i(\Truth{},\TrasfMatrix{}) = \sum_{j=0}^{N_r}m_{ij}\cdot{}t_j.
\end{equation}
The likelihood is then defined by comparing the observed spectrum
\Data{} with the expected one, which includes the background
prediction \Bckg{}:
\begin{equation}
\label{eq:lhood}
\conditionalLhood{\Data{}}{\Truth{}} =
\prod_{i=1}^{N_r}Poisson(d_i,r_i(\Truth{},\TrasfMatrix{})+b_i).
\end{equation}

\subsubsection{Prior}
\label{sec:fbuprior}
While the response matrix can be estimated from the simulated
sample of signal events, the prior probability density \prior{} is to
be chosen according to what we know about \Truth{} before the
measurement is performed.
The simplest choice is an {\it uninformative}
prior that assigns equal probabilities to all \Truth{} spectra within
a wide range $\left[\Tinf{}, \Tsup{}\right]$\footnote{In the analysis,
the range $[\tilde{\Truth{}}/5, 2\tilde{\Truth}]$ is considered, where
$\tilde{\Truth{}}$ is the true spectrum in simulation. It has been
checked that this range is sufficiently wide to not introduce any bias.}:
\begin{equation}
\prior{}
\propto{}
\begin{cases}
1 & \text{if }
T_{t}\in{}\left[\Tinf{}, \Tsup{}\right], \forall{}t\in{}\left[1, N_t\right] \\
0 & \text{otherwise} \\
\end{cases}
\end{equation}
A more general definition for the prior is given by 
\begin{equation}
\prior{}
\propto{}
\begin{cases}
e^{\alpha{}S(\Truth{})} & \text{if }
T_{t}\in{}\left[\Tinf{}, \Tsup{}\right], \forall{}t\in{}\left[1, N_t\right] \\
0 & \text{otherwise} \\
\end{cases}
\end{equation}
where $\alpha{}$ is an arbitrary strength parameter, and
$S(\Truth{})$ is a regularization function.
The choice of $\alpha$ determines the impact of the prior on
\conditionalProb{\Truth{}}{\Data{}}, while $S(\Truth{})$ determines
what additional information is being used to constrain the parameter
space, thus reducing the variance of the \Truth{} parameters by
introducing a small bias.

In the analysis of the dataset at \seventev{}, a curvature prior is
used for the unfolding of the inclusive \ac{} and of the
differential \ac{} spectrum as a function of \pttt{}.
In the curvature prior the regularization function $S(\Truth{})$ is
the difference between the \emph{Thikonov} regularization functions
$C(\Truth)$ (i.e. curvatures) for the given point \Truth{} and for the
truth distribution in simulation. 
The \emph{Thikonov} regularization function is defined as
\begin{equation}
        C(\Truth{}) =
        \sum_{t=2}^{N_t-1}\left[(T_{t+1}-T_{t})-(T_t-T_{t-1})\right]^2.
\end{equation}
The application of this prior disfavors, with strength $\alpha{}$,
\Truth{} candidates with curvature that is very different from
the one of the truth spectrum in simulation.
A regularization strength $\alpha{}=-10^{-8}$ is used, reducing the
uncertainty on the inclusive \ac{} by 50\%, with negligible
bias. Additional details are provided in
App.~\ref{app:unfolding:regularization}.

In the measurements as a function of \mtt{} and \ytt{}, where the
expected curvature varies significantly from bin to bin, choices
other than the uninformative prior result in large bias in some of
the differential bins or do not reduce the variance significantly,
depending on the value of the regularization strength $\alpha{}$.
In the analysis of the sample at \eighttev{}, the uninformative
prior is used for all measurements. In this case, the higher
statistical precision already limits the possible curvatures.

\subsubsection{Sampling}
\label{sec:sampling}
Having chosen the prior, the posterior probability density
\conditionalProb{\Truth{}}{\Data{}} is determined by sampling the
$N_t$--dimensional parameter space, and evaluating for each point
\conditionalLhood{\Data{}}{\Truth{}} and \prior{}, thus performing a
numerical integration. For a large number of parameters, direct
sampling techniques become extremely inefficient; therefore, the
Metropolis--Hastings algorithm~\cite{metropolis}, a Markov
Chain Monte Carlo (MCMC) method~\cite{MR2476411} specialized for
multi--dimensional sampling, is used in FBU.
Combining this set of points with the weight given by
$\conditionalLhood{\Data{}}{\Truth{},\TrasfMatrix{}}\cdot{}\pi{}$, one
can determine not only the posterior probability density distribution
for each bin of the spectrum, but also the posterior probability
density distribution for any quantity that is computed from the
spectrum, such as \ac{}:
\begin{equation}
\conditionalProb{\ac{}}{\Data{}} = 
\int~
\delta(\ac{}-\ac{}(\Truth{}))~
\conditionalProb{\Truth{}}{\Data{}}~d\Truth{}
\end{equation}
As an example, the unfolding input \dy{} distributions in simulated data and
background are shown in Fig.~\ref{fig:unfinput}, as well as the
response matrix.
%
\begin{figure}[!htb]\centering
  \includegraphics[width=0.495\textwidth]{figures/unfolding/input}
  \includegraphics[width=0.495\textwidth]{figures/unfolding/resmat}
  \caption{
    \label{fig:unfinput}
   Unfolding input histograms for simulated data and background (left) and
   response matrix (right) for the inclusive \ac{} measurement at \seventev{}.
   }
\end{figure}
%
The posterior probability densities for the \dy{} distribution and the corresponding
\ac{} are illustrated in Fig.~\ref{fig:posteriorIncl}.
%
\begin{figure}[!htb]\centering
  \includegraphics[width=0.495\textwidth]{figures/unfolding/binposterior}
  \includegraphics[width=0.495\textwidth]{figures/unfolding/asyposterior}
  \caption{
    \label{fig:posteriorIncl}
   Posterior probability densities for the bin content of the \dy{}
   distribution overlaid to the true expected values (left) and for
   the asymmetry \ac{} (right), as obtained from simulated data at
   \eighttev{}.
   }
\end{figure}
%
The mean and RMS of the posterior probability density for \ac{} are
taken as central value and statistical uncertainty respectively. 

\subsubsection{Binning choice and bias}
\label{sec:binandbias}

The choice of the binning for the \dy{} distribution is driven by two
competing factors:
\begin{itemize}
\item The number of parameters to estimate (the yields in each \dy{}
  bin) affects their variance; with fewer bins, the relative statistical
  errors on the bin contents are reduced. Therefore the resulting
  statistical error on \ac{} is smaller with fewer bins. At least two
  bins are necessary to compute \ac{} (positive and negative side of
  the \dy{} distribution).
\item A large number of bins allows an accurate mapping of the
  migrations, yielding unbiased estimates for each bin
  content. However, only migrations that change the \dy{} sign affect
  the computation of \ac{}. Such migrations are more likely for small
  \dy{} values, therefore a fine binning of the central \dy{} region
  ensures an unbiased measurement of \ac{}. 
\end{itemize}
The benchmarks driving the choice of the binning are thus the bias and
the statistical uncertainty on \ac{}.

The bias in the unfolding response is measured by studying the
unfolded asymmetry in pseudo-data samples for which the true
asymmetry is known. The asymmetric samples are built by reweighing the
baseline \ttbar{} simulation to a parton--level \dy{} spectrum
corresponding to inclusive asymmetries of about $\pm2\%$, $\pm4\%$ and
$\pm6\%$.
In order to minimize the effect of statistical fluctuations, the
unfolding procedure for each reference point is repeated in $N_{PE}$
pseudo--experiments. For each reference point with true asymmetry
$\ac{}^{true}$, the mean of the unfolded values $\ac{}^{unf}$, with
uncertainty $\sigma/\sqrt{N_{PE}}$, is then obtained. The set of
($\ac{}^{true}$,$\ac{}^{unf}$) pairs is interpolated with a straight
line parametrized as:
\begin{equation}
\ac{}^{unf}=a\cdot{}\ac{}^{true}+b,
\end{equation}
where $a$ and $b$ are the {\it slope} and {\it offset} parameters.
The unfolding response is considered unbiased when the distance
$|1-a|$ and the offset $b$ are smaller than 10\% of the relative and
absolute statistical error, respectively.
Figure~\ref{fig:pulllinearity} shows the fitted line for
the inclusive \ac{} measurement at \eighttev{}.
For all measurements at \seventev{} and \eighttev{}, four is the
minimum number of \dy{} bins that allows an unbiased
response. The binning choice for each measurement and the
corresponding unfolding response are detailed in
App.~\ref{app:unfolding:binning} and \ref{app:unfolding:linearity}.

The statistical uncertainty on \ac{} is validated by performing unfolding in
an ensemble of pseudo-experiments where statistically--independent
pseudo-data distributions are generated based on Poisson statistics.  
The distribution of $(\ac{}^{unf}-\ac{}^{true})/\sigma^{unf}$
({\it pull}) in the ensemble is considered, where $\ac{}^{unf}$ and
$\sigma^{unf}$ are the unfolded asymmetry value and its uncertainty,
while $\ac{}^{true}$ is the parton--level asymmetry of the sample used
to generate the pseudo-data. As shown in Fig.~\ref{fig:pulllinearity}, the
RMS of the pull distribution is $\approx{}1$, indicating that the
uncertainty is correctly estimated. The pull distributions for each
measurement are collected in App.~\ref{app:unfolding:pulls}

\begin{figure}[!htb]\centering
  \includegraphics[width=0.495\textwidth]{figures/unfolding/linearity}
  \includegraphics[width=0.495\textwidth]{figures/unfolding/pulls}
  \caption{Unfolding response (left) and pull distribution (right) for the
    inclusive \ac{} measurement at \eighttev{}}
  \label{fig:pulllinearity}
\end{figure}

\subsection{Marginalization}
\label{sec:marginalization}

The treatment of systematic uncertainties is naturally included in the
Bayesian inference approach by extending the likelihood
\conditionalLhood{\Data{}}{\Truth{}} with nuisance parameters terms.
The {\it marginal} likelihood is defined as
\begin{equation}
  \conditionalLhood{\Data{}}{\Truth{}}=
  \int
  \conditionalLhood{\Data{}}{\Truth{},\thetavec{}} 
  \cdot{} \pi{}(\thetavec{})
  ~d\thetavec{},
\end{equation}
where \thetavec{} are the nuisance parameters, and
$\pi{}(\thetavec{})$ their prior probability densities, which are
assumed to be Gaussian distributions $G$ with $\mu=0$ and $\sigma=1$.

A nuisance parameter is associated to each of the uncertainty sources
listed in Sec.~\ref{sec:syst_objects}. Two categories of nuisances
are considered: the normalizations of the background processes
($\thetavec{}_b$), and the uncertainties associated to the objects
identification, reconstruction and calibration ($\thetavec{}_s$).
While the first ones only affect the background predictions, the
latter, referred to as object systematic uncertainties, affect both the
reconstructed distribution for \ttbar{} signal and for
the total background prediction, referred to as
$\Reco{}(\Truth{};\thetavec{}_s)$ and
$\Bckg{}(\thetavec{}_s,\thetavec{}_b)$, respectively.
The marginal likelihood becomes then
\begin{equation}
\label{eq:marginalLhood}
  \conditionalLhood{\Data{}}{\Truth{}}=
  \int
  \conditionalLhood{\Data{}}{\Reco{}(\Truth{};\thetavec{}_s),\Bckg{}(\thetavec{}_s,\thetavec{}_b)} 
  \cdot{} G(\thetavec{}_s) ~G(\thetavec{}_b)
  ~d\thetavec{}_s ~d\thetavec{}_b.
\end{equation}
The differential likelihood
$\conditionalLhood{\Data{}}{\Reco{}(\Truth{};\thetavec{}_s),\Bckg{}(\thetavec{}_s,\thetavec{}_b)}$
is defined as in Eq.~\ref{eq:lhood}, with 
\begin{equation}
r_i(\Truth{},\TrasfMatrix{};\thetavec{}_s) =
r_i(\Truth{},\TrasfMatrix{};0) \cdot ( 1 + \sum_k
\theta_s^k\cdot\Delta r_i^k )
\end{equation}
and, for each background process $b$,
\begin{equation}
b_i(\thetavec{}_s,\theta_b) =
b_i(0) \cdot ( 1 + \theta_b\cdot\Delta b) \cdot
(1 + \sum_k \theta_s^k\cdot\Delta b_i^k ),
\end{equation}
where $\Delta b$ is the uncertainty on the background normalization,
$\Delta r_i^k$ ,$\Delta r_i^k$ are the systematic variations for
bin $i$ corresponding to the uncertainty $k$, and the sum runs over
all sources of object systematic uncertainty.
The prior probability densities $G(\thetavec{}_b$ are truncated at
$\theta_{min}=-1/\Delta b$ in order to avoid negative background
normalizations. 

The marginal posterior probability density for \Truth{} is computed by
sampling the $N_t+N_{np}$ parameter space, where $N_{np}$ is the
total number of nuisance parameters, and projecting the sample over
the \Truth{} parameter space. The projections over each nuisance
parameter gives the corresponding posterior probability density,
which matches the Gaussian prior for unconstrained nuisance
parameters, while it has a narrower shape for nuisance parameters
that can be measured in the dataset (see Fig.~\ref{fig:nuispar}).
The posterior probability density for \ac{} is computed as described in
Sec.~\ref{sec:sampling} with the difference that the RMS of the
marginal posterior represents the total uncertainty. Analogously, each
nuisance parameter is estimated by the mean and RMS of the
corresponding projection of the posterior probability density.

\begin{figure}[!htb]\centering
  \includegraphics[width=0.495\textwidth]{figures/unfolding/unconstrainednp}
  \includegraphics[width=0.495\textwidth]{figures/unfolding/constrainednp}
  \caption{Prior and posterior probability densities for nuisance parameters
    corresponding to a the electron momentum scale (left) and a
    component of the b-tagging efficiency (right) in the measurement
    at \eighttev{}.}
  \label{fig:nuispar}
\end{figure}

\subsection{Combination of channels}

As discussed in Sec.~\ref{sec:wjets}, the combination of
orthogonal channels with different background compositions is crucial
to estimate precisely the \wjets{} contamination in the \eighttev{}
data sample.
The marginalization approach provides a natural framework to treat
simultaneously unfolding and background estimation using
multiple data regions. Given the distributions $\Data{}_i$ measured in $N_{ch}$
independent channels, the likelihood defined in Eq.~\ref{eq:marginalLhood}
is extended to the product of likelihoods of each channel, so that
\begin{equation}
  \conditionalLhood{\{\Data{}_1\cdots{}\Data{}_{N_{ch}}\}}{\Truth{}}=
  \int
  \prod_{i=1}^{N_{ch}}\conditionalLhood{\Data{}_i}{\Truth{};\thetavec{}} 
  \cdot{} G(\thetavec{})
  ~d\thetavec{},
\end{equation}
where the nuisance parameters are common to all analysis channels.

The measurements at \eighttev{}  are
performed using a combination of six channels based on
the lepton charge ($Q_l>0$ and $Q_l<0$) and the $b$--jet multiplicity
(0 $b$--jets, 1 $b$--jet, at least 2 $b$--jets). In addition to the
expected number of \ttbar{} events for each bin in \Truth{}, the
\wjets{} calibration factors $K_{\bbbar{}/\ccbar{}}$, $K_c$ and
$K_{light}$ (see Sec.~\ref{sec:wjets}) are free parameters in the likelihood.
The posterior probability density is thus
\begin{equation}
\begin{split}
  \conditionalProb{\Truth{}}{\{\Data{}_1\cdots{}\Data{}_{N_{ch}}\}}=&
  \int
  \prod_{i=1}^{N_{ch}}\conditionalLhood{\Data{}_i}{\Reco{}_i(\Truth{};\thetavec{}_s),\Bckg{}_i(\thetavec{}_s,\thetavec{}_b)}\\
&  ~G(\thetavec{}_s)
  ~G(\thetavec{}_b)
  ~\pi{}(\Truth{})
  ~\pi{}(K_{\bbbar{}/\ccbar{}})
  ~\pi{}(K_c)
  ~\pi{}(K_{light})
  ~d\thetavec{}_s
  ~d\thetavec{}_b,\\
\end{split}
\end{equation}
with
$\Bckg{}=\Bckg{}(K_{\bbbar{}/\ccbar{}},K_c,K_{light};\thetavec_s,\thetavec{}_b)$,
and the $\pi{}$'s are uninformative priors.
Therefore, normalization nuisance parameters are considered
only for the remaining background processes: single top, \zjets{} and
multijet. An independent nuisance parameters is assigned to \zjets{}
and multijet normalizations for each $b$--jet multiplicity, given that
the modeling of $b$--tagging efficiencies is not calibrated
specifically for these processes. Independent nuisance parameters are
considered for the multijet background estimation in \mujets{} and in
\ejets{}. An additional nuisance parameters is
assigned to the simulated single top and \zjets{} backgrounds,
corresponding to uncertainty on the integrated luminosity to which the
MC sample are normalized.

A total of eleven normalization nuisance parameters $\thetavec_b$ is
considered, while the object definition parameters $\thetavec_s$ are 53. 
Figure~\ref{fig:nuispar} summarizes the expected precision and measured
values for all nuisance parameters. The expected precision is obtained
by evaluating the likelihood for a pseudo--data sample corresponding
to the sum of background and signal predictions; thus the central
value for each parameter is measured to be $\theta\sim0$, while the
uncertainty is $\delta\theta\leq1$, depending on the constraining power of
the dataset on each parameter.
\begin{figure}[!htb]\centering
  \includegraphics[width=0.6\textwidth]{figures/unfolding/nuispar}
  \caption{Fitted nuisance parameters to pseudo--data (black) and data
    (red) for the inclusive \ac{} measurement at \eighttev{}. The
    shaded regions highlight the $1\sigma$ (green) and $2\sigma$
    intervals of the prior probability density.}
  \label{fig:nuispar}
\end{figure}
The likelihood evaluated for the actual data gives the nuisance
parameters corresponding to the measurements. 
Most nuisance parameters are measured to be well within $1\sigma$ of
their prior probability density, with little constraining power
($\delta\theta\approx1$). The $b$--jet multiplicity provides
information on the $b$--tag calibration, whose largest uncertainty
component ($b$--efficiency, break--5) is constrained. The nuisance
parameters for the measurement as a function of \mtt{} and a
comparison of \ejets{} and \mujets{} channels is shown in
App.~\ref{app:unfolding:marginalization}. 

\subsection{Additional systematic uncertainties}

The use of unfolding introduces systematic uncertainties
specific of the \ttbar{} simulated sample used to derive the response
matrix \TrasfMatrix{}.
The main potential sources of uncertainty are the statistical fluctuations due
to the limited size of the MC sample and the modeling of the acceptance
and reconstruction effects in the response matrix.

\subsubsection{Monte Carlo statistics}
\label{sec:mcstat}

The impact of statistical fluctuations is studied by repeating the
unfolding procedure $N_{ens}=1000$ times to build an ensemble of
pseudo--experiments. For each pseudo--experiment, the response matrix
is smeared according to the statistical uncertainty of each bin. A
pseudo--data sample corresponding to the sum of background and signal
predictions is considered. The distribution of the unfolded \ac{}
values in the ensemble, shown in Fig.~\ref{fig:mcstat} for the
inclusive \ac{} \eighttev{}, is gaussian--like and centered at the
true \ac{} value. The RMS of the distribution quantifies the impact
of the limited MC statistics. In the \seventev{} analysis, a MC sample
of 4.2M \ttbar{} events is used, corresponding to an uncertainty on the
inclusive \ac{} of about $0.2\%$. The measurements at \eighttev{} are
performed with a 75M events MC sample, bringing the uncertainty down to
$0.08\%$. An analogous estimation is performed for the differential
measurements, and the corresponding uncertainties are summarized in
App.~\ref{app:unfolding:mcstat}.

\begin{figure}[!htb]\centering
  \includegraphics[width=0.6\textwidth]{figures/unfolding/mcstat}
  \caption{Distribution of the \ac{} central values for 1000
    pseudo--experiments where the response matrix is varied
    according to statistical fluctuations of the MC sample.}
  \label{fig:mcstat}
\end{figure}

\subsubsection{Response matrix modeling}

The unfolding procedure relies on the MC simulation of \ttbar{} events
to model acceptance and migration effects in the response matrix
\TrasfMatrix{}. Most of these effects originate from the properties of
the detector and the object reconstruction, whose uncertainties are
accounted for in the marginalization procedure described in
Sec.~\ref{sec:marginalization}. However, the modeling of the
$pp\to\ttbar{}$ process might affect the acceptance efficiencies
$\epsilon_{t_j}$ and the reconstruction performance. While both these
effects are typically large in cross measurements, they are expected
to cancel in the difference and ratio of the asymmetry calculation.

The modeling of the \ttbar{} signal is studied by comparing the
measurements with MC samples corresponding different choices of generator
(\alpgen{} or \mcatnlo{} at \seventev{} and \powheg{} or \mcatnlo{} at
\eighttev{}) and PS (\pythia{} or \herwig{}).
The sensitivity of these comparisons is limited by the MC statistics
uncertainty (see Sec.~\ref{sec:mcstat}) of the samples. 
The various \ttbar{} samples  at \seventev{} have similar size, with a
small MC statistics uncertainty ($0.2\%$ for inclusive
\ac{}), and the comparison shows negligible differences for both
inclusive and differential measurements. 
In the inclusive measurement at \eighttev{}, a $0.32\%$ and $0.08\%$
difference is observed when comparing generators and PS,
respectively. The differences observed in the differential measurement
as a function of \mtt{} is summarized in Fig.~\ref{fig:ttbarmodel}.
These differences are large effects compared with the statistical
precision. However, for the alternative \ttbar{} samples considered,
fewer events are generated with respect to the baseline simulation
using \powheg{} interfaced with \pythia{}. 
The sample generated with \powheg{} interfaced with \pythia{} has 30M
events, corresponding to a $0.13\%$ uncertainty on the inclusive
\ac{}, while the 15M events sample, generated with \mcatnlo{}
interfaced with \herwig{}, yields a $0.19\%$ uncertainty. Therefore
the comparisons of generators and PS are affected by $0.23\%$ and
$0.15\%$ uncertainties, respectively. The differences observed when
comparing PS algorithm are compatible with fluctuations due to MC limited
statistics. A significantly large variation is observed for
$\mtt{}>900\GeV{}$ when comparing generators.
In order to perform a more precise estimation of the uncertainties
associated with \ttbar{} modeling, large (100M events) MC samples are
in production and the differences will be studied at the truth level.

\begin{figure}[!htb]\centering
  \includegraphics[width=0.475\textwidth]{figures/unfolding/acvsmtt_phmcnlo}
  \includegraphics[width=0.475\textwidth]{figures/unfolding/acvsmtt_ppph}
  \caption{Differences in the measured \ac{} spectrum as a function of
    \mtt{} using different generators (left) and different PS
    algorithms (right). The MC statistics uncertainty is also
    shown.}
  \label{fig:ttbarmodel}
\end{figure}
