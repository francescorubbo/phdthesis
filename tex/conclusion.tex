\chapter{Conclusion}
\label{sec:conclusion}

Measurements of the asymmetry in the production of top quark pairs at
the ATLAS experiment have been performed, using a dataset
corresponding to an integrated luminosity of $4.7 \ifb{}$ taken over
the course of 2011 at a centre-of-mass energy \seventev{} and a
dataset corresponding to an integrated luminosity of $20 \ifb{}$
taken over the course of 2012 at a centre-of-mass energy \eighttev{}.

The lepton+jets signature of the semileptonic \ttbar{} decay has been
exploited to select samples enriched in \ttbar{} events by identifying
and reconstructing one isolated lepton with large transverse momentum,
at least four jets and a large missing transverse momentum.
The background contamination of the sample has been studied and
data-driven techniques have been developed to precisely determine its
composition. In particular, the normalization of the different
$W$+heavy--flavor processes has been determined in--situ exploiting
the intrinsic charge asymmetry in $W$ production at LHC. 
Other background processes, such
as single top, Z+jets and diboson production have been estimated with
Monte Carlo simulations, while QCD multijet events have been modeled
with well established data-driven techniques.
The four momenta of the \ttbar{} pair has been reconstructed
event--by--event with a kinematic fit of the measured objects. The
inclusive and differential distribution of the difference of absolute
rapidities of the reconstructed top and antitop quarks, \dy{}, has then been
measured in data and estimated for the backgrounds.
A novel unfolding procedure, based on a bayesian inference approach, has
been performed to estimate the inclusive and differential \dy{}
distribution, and the corresponding asymmetries, at the parton--level,
accounting for the distortions induced by acceptance and
resolution effects.
The procedure has been calibrated with the goal of obtaining the most
precise and accurate achievable asymmetry measurement, by studying the
response in distributions for which the corresponding parton--level
asymmetry is known.

In the \seventev{} dataset, the asymmetry \ac{} has been measured
inclusively, as a function of \mtt{}, \pttt{} and \ytt{}, and has been found
to be compatible with the SM predictions in all cases. The inclusive
measurement and the one as a function of \mtt{} have also been performed in
a $\qqbar{}\to\ttbar{}$ enriched sample ($\betatt{}>0.6$), yielding
SM--compatible results as well. The precision of both inclusive and
differential measurements is limited by the statistical uncertainty,
while the largest systematic uncertainties are the ones with a large
impact on the acceptance for asymmetric backgrounds (\wjets{} and
single top), such as the uncertainty on the energy scale and
resolution of jets and leptons.
In the \eighttev{} dataset, the asymmetry \ac{} has been measured
inclusively and as a function of \mtt{}; in both case the results are
compatible with the SM predictions. The set of measurements will be
completed to match the \seventev{} set, with the differential
measurement as a function of \pttt{} and measurements in the
\qqbar{}--enriched sample with $\betatt{}>0.6$.

At this stage the asymmetry measurements at LHC do not show
significant deviations from the SM predictions. Due to the different
initial state and subsequently different asymmetry definitions, a direct comparison
with the anomaly reported at Tevatron cannot be performed. One
important step would be to simultaneously measure the dependence of
the asymmetries on the \ttbar{} invariant mass \mtt{} and velocity
\betatt{}, in order to extrapolate the 'collider-independent'
asymmetries $A_u$ and $A_d$~\cite{AguilarSaavedra:2012va}. This
measurement is quite demanding from the experimental side, since it
requires a 3--dimensional unfolding in \mtt{}, \betatt{} and
\dy{}. But it offers a unique possibility of testing at LHC the same
quantities that are at the origin of the Tevatron \afb{}. 

Measuring the asymmetry \ac{} as an effective probe for new physics in
future LHC datasets will become more challenging. Proton collisions
will be delivered at higher center-of-mass energy -- $13\TeV{}$ in
2015 -- thus making the fraction of $\qqbar{}\to\ttbar{}$ events even
smaller. Beside enhancing the \qqbar{} fraction by requiring a large
$z$--component of the \ttbar{} velocity, a promising approach is to
study the asymmetry in \ttbar{} events produced in association with
photons~\cite{Aguilar-Saavedra:2014vta}. 
