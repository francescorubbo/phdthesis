\chapter{Conclusion}
\label{sec:conclusion}

Measurements of the asymmetry in the production of top quark pairs at the ATLAS
experiment have been performed, using a dataset corresponding to an
integrated luminosity of 4.7 $fb^{−1}$ taken over the course of 2011
at a centre-of-mass energy \seventev{} and a dataset corresponding
to an integrated luminosity of 20 $fb^{−1}$ taken over the course of 2012
at a centre-of-mass energy \eighttev{}.

The lepton+jets signature of the semileptonic \ttbar{} decay has been
exploited to select samples enriched in \ttbar{} events by
identifying and reconstructing one isolated lepton with large
transverse momentum, at least four jets and a large missing transverse
momentum.
The background contamination of the sample has been studied and data-driven
techniques has been developed to precisely determine its
composition. In particular the normalization of the dominant \wjets{} background
has been constrained exploiting the intrinsic charge asymmetry in $W$
production at LHC.
Other background processes, such as single top, Z+jets and diboson
production have been estimated with Monte Carlo simulations, while QCD
multijet events have been modeled with well established data-driven
techniques.

The four momenta of the \ttbar{} pair has been reconstructed event--by--event with a
kinematic fit of the measured objects. The inclusive and differential
distribution of the difference of absolute rapidities of the
reconstructed top and antitop, \dy{}, has then been measured in data
and estimated for the backgrounds.

An unfolding procedure, based on a bayesian inference approach, has
been performed to estimate the inclusive and differential \dy{}
distribution, and the corresponding asymmetries, at the parton--level,
taking into account the distortions induced by acceptance and
resolution effects.
The procedure has been calibrated with the goal of obtaining the most
precise and accurate achievable asymmetry measurement, by studying the
response for distributions for which the corresponding parton--level
asymmetry is known.

In the \seventev{} dataset, the asymmetry \ac{} has been measured
inclusively, as a function of \mtt{}, \pttt{} and \ytt{}, and has been found
to be compatible with the SM predictions in all cases. The inclusive
measurement and the one as a function of \mtt{} has also been performed in
a $\qqbar{}\to\ttbar{}$ enriched sample ($\betatt{}>0.6$), yielding
SM--compatible results as well.
In the \eighttev{} dataset, the asymmetry \ac{} has been measured
inclusively and as a function of \mtt{}.  