\chapter{Conclusions and Outlook}
\label{sec:conclusion}

Measurements of the charge asymmetry in the production of top quark pairs at
the ATLAS experiment have been performed using $pp$ collision data at 
center--of--mass energies of \seventev{} and \eighttev{}, and corresponding
to integrated luminosities of $4.7 \ifb{}$ and $20 \ifb{}$, respectively.
This represents the full dataset collected by ATLAS over the course of
LHC \mbox{Run 1}.

The lepton+jets signature of the semileptonic \ttbar{} decay has been
exploited to select samples enriched in \ttbar{} events by identifying
one isolated lepton with large transverse momentum,
at least four jets and a large missing transverse momentum.
The background contamination of the sample has been studied and
data--driven techniques have been developed to precisely determine its
composition. In particular, the normalization of the different
$W$+heavy--flavor processes has been determined in--situ exploiting
the intrinsic charge asymmetry in $W$ boson production at LHC. 
Other background processes, such
as single top, Z+jets and diboson production have been estimated with
Monte Carlo simulations, while QCD multijet events have been modeled
with well--established data--driven techniques.
The four--momenta of the \ttbar{} pair have been estimated
event--by--event by performing a kinematic fit under \ttbar{} hypothesis 
using the reconstructed physics objects. 
The distribution of the difference of absolute
rapidities of the reconstructed top and antitop quarks, \dy{}, has then been
measured in data and estimated for the backgrounds.
A novel unfolding procedure, based on a bayesian inference approach, has
been performed to estimate the inclusive and differential \dy{}
distribution, and the corresponding asymmetries, at the parton--level,
accounting for the distortions induced by acceptance and
resolution effects.
The procedure has been calibrated with the goal of obtaining the most
precise and accurate achievable asymmetry measurement, by comparing the
unfolded and parton--level asymmetries in dedicated simulated samples.

In the \seventev{} dataset, the asymmetry \ac{} has been measured
inclusively and differentially as a function of \mtt{}, \pttt{} and $|\ytt{}|$, and has been found
to be compatible with the SM predictions in all cases. 
Measurements of the asymmmetry have also been performed in a sample enriched in 
the $\qqbar{}\to\ttbar{}$ process (via a requirement $\betatt{}>0.6$), both
inclusively and differentially as a function of \mtt{}, yielding
results compatible with the SM predictions as well. 
In the \eighttev{} dataset, the asymmetry \ac{} has been measured
inclusively and differentially as a function of \mtt{}; in both cases the results are
compatible with the SM predictions. The set of measurements will be
completed to match the \seventev{} set, with the differential
measurement as a function of \pttt{} and measurements in the
\qqbar{}--enriched sample with $\betatt{}>0.6$.
Both sets of measurements are the most precise to date at the LHC,
reaching an absolute precision of 0.6\% in the case of the inclusive measurement
at \eighttev{}. In all cases the precision achieved is limited by the statistical 
uncertainty, with the largest systematic uncertainties being those that significantly
impact the acceptance for asymmetric backgrounds (\wjets{} and
single top production), such as the uncertainties on the energy scale and
resolution of jets.
The goal of high experimental precision is justified by the need to achieve
sensitivity to the small deviations expected from new physics effects.
Currently, the asymmetry measurements at LHC do not show
significant deviations from the SM predictions, although they are also
consistent with new physics scenarios that explain the Tevatron
anomaly and \ac{} measurements at the LHC close to the SM prediction.

The precision of these measurements will be critical to significantly
constrain possible models of new physics that would be consistent with
the current experimental measurements at the Tevatron and the LHC. 
A combination of ATLAS and CMS results has been performed at
\seventev{}, while a combination of results at \eighttev{} is also planned.
In both cases the measurements described in this dissertation would play
a significant role, driving the overall precision of the combined result.

Despite the fact that the Tevatron excess is less significant then it used to be, 
there are good reasons to continue to perform precision measurements of the 
asymmetry at the LHC. First and foremost, these measurements are needed in order
to settle the experimental puzzle of the Tevatron measurements. But even 
if the Tevatron anomaly largely disapperars following the availability of higher-order
theoretical predictions, the LHC is probing a new kinematic regime where new physics,
beyond the Tevatron reach, could manifest itself with effects such as asymmetric 
\ttbar{} production

Due to the different initial state and subsequently different
asymmetry definitions, a direct comparison with the anomaly reported
at Tevatron cannot be performed. One important step would be to
simultaneously measure the dependence of the asymmetries on the
\ttbar{} invariant mass \mtt{} and velocity \betatt{}, in order to
extrapolate the 'collider-independent' asymmetries $A_u$ and
$A_d$~\cite{AguilarSaavedra:2012va}, corresponding to the partonic
processes $u\bar{u}\to\ttbar{}$ and $d\bar{d}\to\ttbar{}$, respectively. 
This measurement is quite demanding from the experimental side, since it requires a
three--dimensional unfolding in \mtt{}, \betatt{} and \dy{}. But it offers
a unique possibility of testing at LHC the same quantities that are at
the origin of the Tevatron \afb{}.
The experimental strategy used for this analysis, including the
unfolding technique, appears promising to perform such extension of
the analysis, although detailed studies still need to be performed.

Measuring the asymmetry \ac{} as an effective probe for new physics in
future LHC datasets will become more challenging. Proton collisions
will be delivered at higher center--of--mass energy -- $13\TeV{}$ in
2015 -- thus making the fraction of $\qqbar{}\to\ttbar{}$ events even
smaller. Besides enhancing the \qqbar{} fraction by requiring a large
$z$--component of the \ttbar{} velocity, a promising approach would be to
study the asymmetry in \ttbar{} events produced in association with
photons~\cite{Aguilar-Saavedra:2014vta}.
In fact, the SM--like \ac{} at the LHC could be the result of an accidental
cancellation of asymmetries of different sign for $\uubar\to\ttbar$
and $\ddbar\to\ttbar$, and the $\ttbar{}\gamma$ signature would allow
to break such cancellation, but requires a high integrated luminosity,
as anticipated during the LHC \mbox{Run 2}.

To conclude, beyond the intrinsic interest of the program of precision
measurements of the charge asymmetry in \ttbar{} production at the LHC,
the strategies and techniques developed in this dissertation should prove
valuable in the next round of new physics searches and precision
measurements in the top quark sector at the LHC \mbox{Run 2}.
Whether or not these measurements reveal new physics effects, they
will hopefully contribute in a significant way to advance our understanding on
how Nature operates at its most fundamental level.