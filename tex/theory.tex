\chapter{Theoretical framework}
\label{sec:theory}

\section{The Standard Model}
\label{sec:theSM}

The Standard Model (SM) is a quantum field theory describing the
interactions between the elementary constituents of matter through the
fundamental forces.
The matter particles -- three {\it leptons}, three {\it neutrinos} and six
{\it quarks} -- are fermions categorized in three families. For each particle
there exist an antiparticle with same mass and opposite quantum
numbers for a total of 24 matter particles.
They all carry a weak charge and interact with
the weak force carriers $W$ and $Z$. All matter particles, except for
neutrinos, are electrically charged and interact with the
electromagnetic force, mediated by $\gamma$. Only the quarks have a
color charge and undergo strong force interactions with {\it gluons}.  
Table~\ref{tab:smparticles} shows the elementary particles of the
Standard Model, classified as matter particles and force carriers.

\begin{table}[!htb]\centering
  \begin{tabular}{ l c c c c|c}
    \toprule
    \multicolumn{5}{c}{Matter} & Force carriers \\
    \midrule
                                       & Electric Charge & Family 1 &
                                       Family 2 & Family 3 &  \\
                                       \multirow{2}*{quarks} & $+2/3$
                                       & $u$ & $c$ & $t$ & $\gamma$ \\
                                       & $-1/3$ & $d$ & $s$ & $b$ &
                                       $W^{\pm}$ \\
                                       \multirow{2}*{leptons} & $-1$
                                       & $e$ & $\mu$ & $\tau$ & $Z$ \\
                                       & $0$ & $\nu_e$ & $\nu_{\mu}$ & $\nu_{\tau}$ &
                                       $g$ \\
    \bottomrule
  \end{tabular}
  \caption{Table of particles in the Standard Model}
  \label{tab:smparticles}
\end{table}

The theory describing the particle interactions is structured
according to the gauge group $SU(3)_C\times{}SU(2)_L\times{}U(1)_Y$. The three
terms represent the fundamental symmetries corresponding to the forces describing the
interaction of particles: $SU(3)_C$ is the {\it color} ($C$) symmetry corresponding
to the strong force; $SU(2)_L\times{}U(1)_Y$ is the spontaneously
broken symmetry with respect to the {\it isospin} ($L$)
and {\it hypercharge} ($Y$) gauge groups.
The SM is then the combination of two theories: the {\it Quantum
  Chromodynamics} (QCD) lagrangian, which describes the strong
interaction arising from the $SU(3)$ color symmetry, and the {\it
  Electroweak} lagrangian, which accounts for the electroweak
interactions corresponding to the $SU(2)\times{}U(1)$ isospin and
hypercharge symmetries.
\begin{equation}
\begin{split}
\lagr{}_{SM}  & = \lagr{}_{SU(3)} + \lagr{}_{SU(2)\times{}U(1)} \\
& = \lagr{}_{SU(3)}^{Gauge} 
+ \lagr{}_{SU(2)\times{}U(1)}^{Gauge} 
+ \lagr{}_{SU(3)}^{Matter}
+ \lagr{}_{SU(2)\times{}U(1)}^{Matter}
+ \lagr{}_{SU(2)\times{}U(1)}^{Higgs} +
\lagr{}_{SU(2)\times{}U(1)}^{Yukawa} \\
\end{split}
\phantom{.}
\end{equation}
The $\lagr{}^{Gauge}$ terms describe the dynamics of the gauge fields:
the gluons in QCD, and the $W$, $Z$ and $\gamma$ boson in the
electroweak theory.
The $\lagr{}^{Matter}$ terms describe the
interaction of particles with the gauge fields. The $\lagr{}^{Higgs}$
and $\lagr{}^{Yukawa}$ terms arise from the spontaneous symmetry
breaking of the $SU(2)\times{}U(1)$ gauge theory via the Higgs
mechanism, and they are responsible, respectively, for the interaction
of the Higgs field with the other particles and force carriers, and
for generating their masses.
The Higgs field
%, whose corresponding boson has been discovered at the LHC in
%2012~\cite{Aad:2012tfa,Chatrchyan:2012ufa} 
needs to be introduced in
the $SU(2)\times{}U(1)$ theory in order to account for the
non-vanishing masses of $W$ and $Z$ bosons and of lepton and quarks.
The Higgs boson does not interact with the gluon and the photon;
therefore they are the only two massless particles of the
SM. Table~\ref{tab:smmasses} summarizes the masses of the SM particles.

\begin{table}\centering
  \begin{tabular}{cccc}\toprule
    Spin-$1$ & \multirow{2}{*}{mass} & Spin-0 & \multirow{2}{*}{mass}\\
    gauge bosons & & scalar boson & \\\midrule
    $\gamma$      & 0  & \multirow{4}{*}{$H$} & \multirow{4}{*}{$\sim 125~\GeV$}                        \\
    $g$           & 0  & &                        \\
    $W^{\pm}$ & $ 80.385 \pm 0.015 \GeV$ & &\\
    $Z$       & $ 91.188 \pm 0.002 \GeV$ & &\\\bottomrule
  \end{tabular}
  \begin{tabular}{ccccccc}\toprule
    Spin-$\tfrac{1}{2}$ &  \multicolumn{2}{c}{\multirow{2}{*}{I generation}}
    &  \multicolumn{2}{c}{\multirow{2}{*}{II generation}}
    &  \multicolumn{2}{c}{\multirow{2}{*}{III generation}}\\
    fermions & & & & & \\\midrule
    \multirow{2}{*}{leptons} &
    $\nu_{e}$   & \small{$\sim 0$} &  
    $\nu_{\mu}$ & \small{$\sim 0$} &  
    $\nu_{\tau}$ & \small{$\sim 0$} \\
    &
    e            & \small{$0.511\MeV$}   &  
    $\mu$ & \small{$105.7\MeV$} &  
    $\tau$     & \small{$1.777\GeV$} \\
    \multirow{2}{*}{quarks} &
    u & \small{$1.7-3.1\MeV$}         &  
    c & \small{$1.29^{+0.05}_{-0.11}\GeV$}  &  
    t & \small{$173.3\pm0.8\GeV$}\\
    &
    d & \small{$4.1-5.7\MeV$} &  
    s & \small{$100^{+30}_{-20}\MeV$} &  
    b & \small{$4.19^{+0.18}_{-0.06}\GeV$} \\\bottomrule
  \end{tabular}
  \caption{Mass values for the elementary particles of the Standard
    Model, as measured at experiments.\label{tab:smmasses}}
\end{table}
 
\section{The top quark}
\label{sec:topquark}

The top quark was discovered in 1995 at the Tevatron by the
CDF~\cite{topcdf} and \dzero{}~\cite{topdzero} collaborations. With a
mass of $\approx{}170\GeV{}$, it is the heaviest elementary particle
observed so far. Its proximity to the electroweak scale suggests that
the top quark might play a relevant role in new physics
scenarios. Therefore its production, decay modes and properties have
been studied in detail at the two hadron colliders powerful 
enough to produce it: the Tevatron, where protons and antiprotons were
collided (\ppbar{}), and the Large Hadron Collider (LHC), colliding protons.

\subsection{Pair production at hadron colliders}
\label{sec:topprod}

At hadron colliders, top quarks are mainly produced in pairs
(\ttbar{}) via the strong interaction. The constituent partons of the
colliding hadrons (protons or antiprotons) participate in a hard
scattering process and produce a top quark and an antitop quark. The
leading production modes are: {\it gluon--gluon fusion} and {\it
  quark--antiquark annihilation}. The leading order Feynman diagrams
for these two processes are shown in Fig.~\ref{fig:ttbarprod}. The
following discussion focuses on the \ttbar{} production in proton
collisions; however similar considerations apply to the case of
proton-antiproton collisions.

\begin{figure}[!htb]\centering
  \includegraphics[width=0.95\textwidth]{figures/theory/ttbarprod}
  \caption{Leading order diagrams for QCD top quark pair production.
    Gluon fusion, {\tt a)} and {\tt b)}, is the dominant process at
    LHC energies, while quark--antiquark annihilation, {\tt c)}, is
    the dominant one at Tevatron energies~\cite{Fiorini:2012fe}.} 
  \label{fig:ttbarprod}
\end{figure}

Due to the composite nature of the proton, most of the
collisions involve only soft (i.e. long--distance) interactions of the
constituent quarks and gluons. Such interactions cannot be described
with perturbative QCD because the expansion parameter $\alpha_s$ is
large for small momentum exchange.
In some collisions, however, in addition to the long distance parton
interactions, two quarks or two gluons undergo a hard (i.e. short--distance) scattering where a top quark pair is produced. 
The {\it factorization principle}~\cite{factorprinciple} states that
the perturbative QCD description of this hard process is possible by
factorizing the long--distance effects into functions $f$ describing the
proton structure. Thus the cross section for top quark pair
production in a collision of two protons with momenta $P_1$ and $P_2$
is given by: 
\begin{equation}
  \sigma(p(P_1)+p(P_2)\to\ttbar{}+{\rm X})
  = 
  \sum_{a,b}
  ~\int_0^1{\rm d}x_1
  ~\int_0^1{\rm d}x_2
  ~ f_a(x_a,\factscale{}) f_b(x_b,\factscale)
  \cdot{}\hat{\sigma_{ab\to\ttbar{}+\rm{X}}}(x_1P_1,x_2P_2,\renormscale{},\factscale{})
  \phantom{,}
  \label{eq:ttbarxs}
\end{equation}
where the sum runs over all type of partons, and $x_{1,2}$ are the
fractions of proton momentum carried by the constituents participating
in the hard interaction. 

The {\it parton distribution functions} (PDFs) $f_a(x,\factscale{})$
represent the probability density for a parton of a type $i$ to carry
a fraction of proton momentum $x$. The {\it factorization scale}
\factscale{} is set arbitrarily and defines the distinction between
short-- and long--- distance interactions based on the transferred momentum
$Q^2$. An additional renormalization scale \renormscale{} accounts for higher
order corrections. For calculations and simulations, both scales are
set to the typical transferred momentum of the process studied. In the case of
\ttbar{} production the scale is chosen to be equal to the top mass $m_t$.
  
The PDFs cannot be computed in perturbation theory; instead, they are
measured in deep inelastic scattering experiments and at hadron colliders.
For the studies presented in this thesis, the PDF estimations
provided by the CTEQ~\cite{cteq6}, MRST~\cite{mrst}, MSTW~\cite{mstw}, CT10~\cite{ct10}
and NNPDF~\cite{nnpdf} collaborations are used. Fig.~\ref{fig:pdfs}
shows the PDFs of valence quarks, gluons and sea quarks for two values
of transferred momentum $Q^2$ at which the proton is probed. 

\begin{figure}[!htb]\centering
  \includegraphics[width=0.75\textwidth]{figures/theory/mstw2008nnlo68cl_allpdfs}
  \caption{MSTW parton distribution functions~\cite{ct10} for gluons
    and quarks at $Q^2=10\GeV{}^2$ (left) and $Q^2=10^4\GeV{}^2$ (right).}
  \label{fig:pdfs}
\end{figure}

The typical fraction of momentum $x$ carried by each of the colliding
partons in order to produce a top quark pair is defined by the
relationship
\begin{equation}
\sqrt{x_1x_2s}\geq{}2m_t\phantom{.}
\end{equation}
Therefore, assuming the partons carry a similar fraction of momentum
$x_1\simeq{}x_2\simeq{}x$:
\begin{equation}
x=\frac{2m_t}{\sqrt{s}}
\phantom{.}
\end{equation}
This corresponds to a typical value of $x\approx{}0.05$ at the LHC for
a center-of-mass energy \seventev{} and \eighttev{}. As shown in
Fig.~\ref{fig:pdfs}, the probability of gluon collisions is
significantly larger than for any other parton in the corresponding
range of $x$. Thus the production of top quark pairs at the LHC is
dominated by the gluon fusion process ($\approx{}90\%$). On the
contrary, at the Tevatron, the typical value of $x\approx{}0.2$ makes
the valence quark-antiquark annihilation the prevalent \ttbar{}
production mode.
The total \ttbar{} production cross section at the LHC is calculated
at next-to-next-to leading order in
QCD to be
$177.31^{+10.1}_{-10.8}$ pb at \seventev{} and
$252.89^{+13.30}_{-14.52}$ pb at \eighttev{}, for a top quark mass of
$172.5 \GeV{}$~\cite{ttxs1,ttxs2,ttxs3,ttxs4,ttxs5,ttxs6,ttxs7}.
 
Measurements have been performed by both ATLAS and CMS
collaborations, yielding a combined value of $173.3\pm10.1$ pb at
\seventev{}~\cite{ATLAS-CONF-2012-134,CMS-PAS-TOP-12-003}. Preliminary
measurements at \eighttev{} yield $\sigma_{\ttbar{}}=242\pm{}9$ pb
(ATLAS)~\cite{Aad:2014kva} and $\sigma_{\ttbar{}}=239\pm{}13$ pb
(CMS)~\cite{Chatrchyan:2013faa}. All measurements are in agreement
with the SM predictions above.

\subsection{Decay}
\label{sec:topdecay}

The top quark has an extremely short lifetime of $5\times{}10^{-25}$
s. Therefore decay occurs before hadronization can take place, and the
decay products carry all of the information about \mbox{4--momentum} and spin
of the original particle.
The top quark decays in almost all cases via electroweak
charged current interaction into a $b$ quark and a $W$ boson that,
in turn, decays either leptonically, into a charged lepton and the corresponding
antineutrino, or hadronically into a quark--antiquark pair. Thus the
final states corresponding to a \ttbar{} pair can be classified in
three categories, based on the decays of the two $W$ bosons
originating from the decays of the top and the antitop quarks:
\begin{itemize}
\item Full hadronic final state: both $W$ bosons decay into quarks,
  leading to a \ttbar{} final state with six quarks.
\item Semileptonic final state: one $W$ boson decays into quarks,
  while the other decays leptonically, leading to a final state with
  four quarks, one lepton and one neutrino.
\item Dileptonic final state: both $W$ bosons decay leptonically,
  leading to a final state with two $b$ quarks, two leptons and two
  neutrinos. 
\end{itemize}
Given that the $W$ boson hadronic branching ratio is
$\approx{}2/3$, the full hadronic and semileptonic final states occur
$4$ out of $9$ times each, while the dileptonic decay has the
remaining $1/9$ probability.  

\subsection{Properties}
\label{sec:topprop}

The properties of the top quark have been studied in detail at hadron
collider experiments. The combination of Tevatron and LHC results
brought the precision on the top mass measurement well below $1 \GeV{}$
with $m_t=173.3\pm{}0.8 \GeV{}$~\cite{topmass}.
An exotic electric charge of $4e/3$ for the top quark, for which the
SM predicts a $2e/3$ charge, has been excluded by measurements at both
Tevatron experiments~\cite{Abazov:2006vd,Aaltonen:2010js} and at the
ATLAS experiment~\cite{Aad:2013uza}.
Since the top quark does not form bound hadronic states due to its
short lifetime, it is the only quark whose spin properties, which are usually
concealed by hadronization, can be measured. The final state
particles in the decay carry information about the top quark spin;
therefore, it's possible to measure the top polarization and the spin
correlation of the top quark pair. At the current precision, these
quantities have been found to be compatible with the SM
predictions~\cite{Aad:2013ksa,Aad:2014pwa,Chatrchyan:2013wua}. 

\section{Charge asymmetry in top quark pair production}
\label{sec:topca}

The measurements of the top quark properties listed in
Sec~\ref{sec:topprop} yield results compatible with SM predictions.
A different trend appeared to follow the {\it forward-backward}
(FB) asymmetry in the production of \ttbar{} pairs in \ppbar{}
collisions at the Tevatron, The first precise measurements, using half
of the Tevatron Run II dataset, showed large discrepancies between
data and theory ($\gtrsim 3\sigma$). 
The asymmetry \afb{} is defined in terms of the rapidity $y$ of the top and
antitop quarks in the laboratory frame\footnote{The rapidity of a
  particle at colliders is given by $y=\frac{1}{2}ln(\frac{E+p_z}{E-p_z})$, where
  $E$ is the energy of the particle and $p_z$ the component of its
  momentum along the beam axis $z$.}: 
\begin{equation}
\label{eq:afb}
\afb{}=\frac{N(\Delta{}y>0)-N(\Delta{}y<0)}{N(\Delta{}y>0)+N(\Delta{}y<0)}
\phantom{,}
\end{equation}
with $\Delta{}y=y_t - y_{\bar{t}}$ and $N$ number of events.
At present, when the full Tevatron dataset has been analyzed, the
discrepancies have been reduced with respect to previous results. The
CDF collaboration reports a $1.7\sigma$ excess over the SM
prediction~\cite{Aaltonen:2012it}, whereas the \dzero{} collaboration
finds agreement within $1\sigma$~\cite{Abazov:2014cca}.

At the time when this thesis work began, a discrepancy between
experimental data and the SM prediction over three standard deviations
was reported by the CDF collaboration in the measurement of \afb{} at
high \ttbar{} invariant mass $\mtt{}>450\GeV{}$, using half of the
total dataset. This anomaly triggered a intense activity, both in
developing new physics models which could explain the anomaly in
\ttbar{} production, and in calculating more precise and accurate SM
predictions. The result also motivated studies of the \ttbar{}
production phenomenology aiming at identifying other observables where
anomalies might appear if the Tevatron asymmetry were indeed a
sign of new physics. Among these, the charge asymmetry at the LHC is
one of the best candidates. 

In $pp$ collisions a FB asymmetry with respect to a fixed direction,
such as the one defined in Eq.~\ref{eq:afb}, vanishes due to the
symmetry of the initial state. However a {\it forward-central} charge
asymmetry \ac{} can be defined, 
\begin{equation}
\label{eq:ac}
\ac{}=\frac{N(\Delta{}|y|>0)-N(\Delta{}|y|<0)}{N(\Delta{}|y|>0)+N(\Delta{}|y|<0)}
\phantom{,}
\end{equation}
with $\Delta{}|y|=|y_t| - |y_{\bar{t}}|$.

The need for a different observable becomes more clear by considering
the asymmetry in the distribution of the angle $\theta_t$ between the outgoing
top quark and the incoming quark (Fig.~\ref{fig:cmdiagram}). This angle
would be the fundamental observable to measure the charge asymmetry in
\ttbar{} production, but the direction of the incoming quark is
not directly accessible at experiments. However, at Tevatron \ppbar{}
collisions, the quark is provided by the proton with very high
probability ($\gtrsim{}99\%$ at $\sqrt{s}=1.96\TeV{}$); therefore the
top quark rapidity $y$ with respect to the proton beam direction can
be used in Eq.~\ref{eq:afb} to distinguish between {\it forward} and
{\it backward} production with high efficiency. At LHC $pp$ collisions
the incoming quark has the same probability to be provided by
either proton beam; thus the FB asymmetry of Eq.~\ref{eq:afb} vanishes by
construction. However, valence quarks are more likely to participate
in the interaction than sea quarks and carry, on average, a larger fraction of
momentum than sea anti-quarks. Therefore, in the laboratory frame, the
direction of flight of the \ttbar{} system is likely to be the same as of
the incoming quark. In this scenario, a forward top quark in
the \ttbar{} C.M. has on average a larger absolute rapidity $|y|$, in
the laboratory frame, than the backward antiquark. Thus the asymmetry
in Eq.~\ref{eq:ac} allows probing the charge asymmetry in \ttbar{}
production at the LHC.

\begin{figure}[!htb]\centering
  \includegraphics[width=0.495\textwidth]{figures/theory/cmdiagram}
  \caption{A representation of the $\qqbar{}\to\ttbar{}$ process in the
    C.M. frame. The top quark $t$ is produced at an angle $\theta_t$
    with respect to the direction of the incoming quark $q$. For
    symmetric production the top quark is produced isotropically,
    while for a positive (negative) asymmetry, positive (negative)
    values of $\cos\theta_t$ are favored. } 
  \label{fig:cmdiagram}
\end{figure}


\subsection{Charge asymmetry in the SM}
\label{sec:smca}

The dominant contribution to the SM charge asymmetry originates from
the QCD $\qqbar{}\to\ttbar{}+X$ production. Specifically, it originates from the
interference between the Born amplitude for $\qqbar{}\to\ttbar{}$ and
its one--loop box correction, and from the one between initial and final
state radiation (ISR, FSR) in $\qqbar{}\to{}\ttbar{}g$
(Fig.~\ref{fig:asymdiagrams}). The terms of the angular differential cross section
$d\sigma_{\qqbar{}\to{}\ttbar{}}/d\cos\theta_t$ (Fig.~\ref{fig:cmdiagram})
corresponding to these interferences are antisymmetric for the
exchange of top and antitop quark momenta~\cite{Kuhn:1998kw}. In
particular, considering the definitions in Eq.~\ref{eq:afb}
and~\ref{eq:ac}, Born-box interference generates positive asymmetries,
while the ISR-FSR interference generates negative asymmetries.
The relative size of the two contributions depends on the transverse
momentum of the \ttbar{} system \pttt{}. For
$\pttt{}\lesssim{}25\GeV{}$ the asymmetry is positive, while for
$\pttt{}\gtrsim{}25\GeV{}$ is negative. The overall effect is a
positive asymmetry.
\begin{figure}[!htb]
  \centering
  \subfloat[][]
  {  \includegraphics[width=0.35\textwidth]{figures/theory/born} } \quad
  \subfloat[][]
  {  \includegraphics[width=0.35\textwidth]{figures/theory/box} } \\
  \subfloat[][]
  {  \includegraphics[width=0.35\textwidth]{figures/theory/isr} } \quad
  \subfloat[][]
  {  \includegraphics[width=0.35\textwidth]{figures/theory/fsr} } 
  \caption{Main sources of the QCD charge asymmetry in \ttbar{}
    production: interference of Born (a) and box (b) diagrams and
    interference of initial state (c) and final state (d) gluon
    radiation diagrams}
  \label{fig:asymdiagrams}
\end{figure}

Charge asymmetric contributions also arise from the interference of
the QCD $\qqbar{}\to{}\ttbar{}$ diagrams in
Fig.~\ref{fig:asymdiagrams} with the QED analogous where a gluon is
replaced by a photon. The size of the QED corrections to the QCD
asymmetric cross section depends on the relative importance of \uubar{}
and \ddbar{} annihilations, due to the difference in electric charge.
In \ppbar{} collisions (Tevatron) the QED contribution is estimated to
be $\approx{}18\%$, while at LHC $pp$ collisions, with larger relative
\ddbar{} importance, it accounts for approximately $13\%$ of the
asymmetry~\cite{Kuhn:2011ri}.
Analogous contributions from weak interactions constitute only a
$\approx{}1\%$ correction at Tevatron and are negligible at LHC due to
the smallness of the weak coupling.

The asymmetry definition in Eq.~\ref{eq:afb} and~\ref{eq:ac} can be
generalized as the ratio between the asymmetric contribution
$\sigma_A$ to the cross section and the symmetric one $\sigma_S$:
\begin{equation}
  A_{QCD} = \frac{\sigma_A}{\sigma_S} = 
  \frac{\alpha_S^3\sigma_A^{(1)}+\alpha_S^4\sigma_A^{(2)}+\cdots{}}
  {\alpha_S^2\sigma_S^{(0)}+\alpha_S^3\sigma_S^{(1)}+\cdots{}}
\end{equation}
Because the asymmetry vanishes at the tree level in the SM, a
consistent fixed-order expansion at LO in perturbation theory involves the
numerator at NLO and the denominator at LO:
\begin{equation}
  A_{QCD}=\alpha_SA_{QCD}^{(0)}=\frac{\alpha_S\sigma_A^{(1)}}{\sigma_S^{(0)}}
\end{equation}
Predictions of \afb{} for Tevatron and \ac{} for LHC have thus been
computed including the $\mathcal{O}(\alpha_s^3)$
QCD contribution and the mixed $\mathcal{O}(\alpha_s^2\alpha)$ QCD-QED
and QCD-weak corrections discussed above. The SM forward-backward
asymmetry at \ppbar{} collision, evaluated at $\sqrt{s}=1.96\TeV{}$,
is $\afb{} = 0.088\pm{}0.006$, while the SM predictions for the LHC
asymmetry are $\ac{}=0.0123\pm{}0.0005$ and $\ac{}=0.0111\pm0.0005$,
at \seventev{} and \eighttev{} respectively~\cite{Bernreuther:2012sx}.
 
The SM prediction for \ac{} is one order of magnitude smaller than for
\afb{} because of two effects. First, the symmetric $gg\to\ttbar{}+X$
process is dominant at the LHC, accounting for $80\%$ of the total
cross section, as opposed to the $15\%$ at the Tevatron. This process
contributes exclusively to the denominator in Eq.~\ref{eq:afb}
and~\ref{eq:ac}, diluting the asymmetric $\qqbar{}\to\ttbar{}$
production. In addition, the probability that the anti-quark carries a
larger fraction of momentum than the quark -- in which case the
assumption that the \ttbar{} system is boosted in the direction of the
incoming quark is voided -- is not negligible, and leads to a further
dilution of the asymmetry. The impact of both
dilutions can be reduced by applying a minimum requirement on the
$z$-component of the \ttbar{} system velocity \betatt{}~\cite{AguilarSaavedra:2011cp}. Due to the average momentum imbalance
between quark and antiquark, in \qqbar{} annihilation the \ttbar{}
system is often produced with a large longitudinal
momentum. Fig.~\ref{fig:qqbarfrac} shows how the relative importance
of the \qqbar{} process increases with the minimum requirement on \betatt{}.

\begin{figure}[!htb]
  \centering
  \includegraphics[width=0.495\textwidth]{figures/theory/qqfrac-beta.png} 
  \caption{Relative fraction of $\qqbar{}\to\ttbar{}$ events as a function of the
    minimum \ttbar{} velocity.} 
  \label{fig:qqbarfrac}
\end{figure}

\subsection{Charge asymmetry in beyond--SM scenarios}
\label{sec:bsmca}

An appealing possibility is that the discrepancies between the
experimental results and the SM predictions for \afb{} are a signal of
new physics in \ttbar{} production. In the last few years various
extensions of the SM have been proposed to explain the excess of the
measured \afb{}. The following models have been studied in
detail by theorists with a focus on their impact on the charge
asymmetry in \ttbar{} production~\cite{Aguilar-Saavedra:2014kpa}. 

{\em Color-octet vector $G$}. Exchanged in the $s$ channel via
flavor-diagonal couplings, it gives 
an amplitude that interferes with the SM gluon-exchange diagram. The
corresponding contribution to the charge asymmetry in
$\qqbar{}\to\ttbar{}$ is proportional to the product of axial
couplings with light and top quarks $g_A^{u,d}g_A^t$. Depending on the
relative sign of the couplings and on the mass $M_G$, the contribution
to the asymmetry can be positive or negative. 

{\em Neutral $Z'$ boson}. It contributes to the $\uubar{}\to\ttbar{}$
process in a flavor changing $t$ channel. A negative asymmetry
is generated in the leading order interference; therefore higher order
corrections are required to fit the Tevatron excess. However such
tuning of the model is disfavored by measurements of other observables.

{\em Charged $W'$ boson}. This gauge boson couples to right-handed
quarks and contributes in the $t$ channel to the partonic process
$\ddbar{}\to\ttbar{}$. Larger couplings are needed to compensate the
lower \ddbar{} luminosity. This field also produces negative
contributions to the asymmetry, and it's disfavored by current measurements.

{\em Scalar isodoublet $\phi$}.
Exchanged in the $t$ channel with a flavor-changing coupling, it gives
a positive contribution to the asymmetry via the interference with the
tree--level process. For small masses, an asymmetry consistent with the
\afb{} results can be obtained with relatively small couplings.

{\em Color-triplet scalar $\omega$}.
With charge $4e/3$, it appears only in flavor changing $u$ channel of
$\uubar{}\to\ttbar{}$. The leading order contribution to the asymmetry
is negative, so large couplings are required to have large higher
order corrections. Consequently, cancellations of large leading order
effects must occur to accomodate current measurements.

{\em Color-sextet scalar $\Omega$}.
Similar to the $\omega$ above, it also has charge $4e/3$ and therefore
contributes in the $u$ channel diagram. However, with a positive
asymmetry, it can easily fit the experimental results.

In all the models, the mass of the new particle and the couplings are
free parameters which can be tuned to generate asymmetries of
different size. The range of allowed \ac{} and \afb{} values for each
new physics model are shown in Fig.~\ref{fig:bsmmodels}. Depending on
the features of the model, the asymmetries show different dependence
on the \ttbar{} invariant mass. Therefore experimental input on this
dependence in data is valuable for model discrimination. 

\begin{figure}[!htb]
  \centering
  \includegraphics[width=0.495\textwidth]{figures/theory/bsm} 
  \includegraphics[width=0.495\textwidth]{figures/theory/bsm_highmtt} 
  \caption{Allowed regions for the new physics contributions to the FB
    asymmetry at Tevatron and the inclusive charge asymmetry at LHC
    (left) and for $\mtt{}>600 \GeV$~\cite{AguilarSaavedra:2011hz}.}
  \label{fig:bsmmodels}
\end{figure}


