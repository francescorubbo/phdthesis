In this appendix we illustrate the details of the FBU implementation
the unfolding cross-checks.

\subsection{Regularization}
\label{sec:app:unfolding:regularization}
In this section we briefly summarize the details of the regularization
used for the inclusive unfolding and for the differential one as a
function of \pttt{}.

We use the definition of curvature given in eq.~26 in
Ref.~\cite{Fbu2012arXiv1201.4612C}
%
\begin{equation}
        C(\Truth{}) =
        \sum_{t=2}^{N_b-1}\left[(T_{t+1}-T_{t})-(T_t-T_{t-1})\right]^2
\end{equation}
%
where $N_b$ is the number of bins. We compute the regularization
function as the absolute value of the curvature difference between the
truth spectrum and the $\tilde{\Truth{}}$ spectrum of the simulated
events that are used to fill the transfer matrix:
%
\begin{equation}
 S(\Truth{}) = |C(\Truth) - C(\tilde{\Truth{}})|.
\end{equation}
%
The main idea behind this function is that it decreases the weight of
generated points \Truth{} with curvature that is very different from
the one of the truth spectrum.

For the unfolding of two-dimensional spectra, the curvature is
computed separately in each differential subrange and the
regularization function is the sum of the regularization functions
computed in each subrange.  We choose to compute the curvature
separately within each subrange for two reasons: first, we do not want
to include in this calculation the bins that mark the transition
between one subrange and the next; second, the numerical value of the
curvature can be significantly different in different bins of the
differential variable.  The numerical values of the curvature are
reported in table~\ref{tab:curvature}; the numerical value of the
parameter $\alpha{}$ is chosen so that $\alpha{}\cdot{}S$ is not far
from unity.
%
\begin{table}[htbp]
  \begin{center}
  \begin{tabular}{ l ccccc }
  \toprule
   & \multicolumn{5}{c}{Curvature : $\frac{\text{truth}}{\text{reco}}$} \\
   \midrule
   inclusive & $\frac{6.1e+05}{9.7e+05}$ & & & & \\
   \mtt{} & $\frac{1.4e+09}{1.0e+07}$ & $\frac{1.2e+08}{1.4e+05}$ & $\frac{1.8e+08}{5.5e+05}$ & $\frac{7.4e+07}{3.1e+05}$ & $\frac{1.7e+07}{7.6e+04}$ \\
   \pttt{} & $\frac{1.4e+06}{8.4e+04}$ & $\frac{7.0e+05}{1.8e+05}$ & $\frac{2.6e+06}{2.7e+05}$ & & \\
   \ytt{} & $\frac{1.8e+09}{1.5e+07}$ & $\frac{3.2e+09}{2.3e+07}$ & $\frac{5.2e+07}{2.8e+06}$ & $\frac{4.9e+08}{1.5e+06}$ & \\
   \bottomrule
    \end{tabular}
  \end{center}
  \caption{Curvature numerical values of the spectra being unfolded,
  at the truth level and at the reconstruction level. For the
  differential spectra, the curvature is computed separately in each
  subrange corresponding to one bin of the differential variable.
  }
  \label{tab:curvature}
\end{table}

Fig.~\ref{fig:app:fbutestprior} shows the FBU posterior for the inclusive measurement
corresponding to four different choices of the regularization parameter $\alpha{}$.
While the width of the posterior is sensibly reduced for larger values of $\alpha{}$, the mean value is stable.

\begin{figure}
  \begin{center}
  \includegraphics[width=0.45\textwidth]{figures/appendix/fbu/FBUtestprior}
  \caption{
    \label{fig:app:fbutestprior}
    FBU posterior for the inclusive measurement corresponding to four values of $\alpha{}$.
    }
  \end{center}
\end{figure}


\subsection{Linearity}
\label{sec:app:unfolding:linearity}
The following figures show the results of the tests that we perform to
verify the linearity of the FBU method.  These tests are performed for
the one-dimensional unfolding used in the inclusive measurement
(figure~\ref{fig:app:fbuLin}), and for the two-dimensional unfolding
used in the differential measurements
(figures~\ref{fig:app:fbuUnfACmtt}--\ref{fig:app:fbuLinACpTtt}.  The
linearity is tested by measuring the \ac{} value after unfolding for
several \protos{}~\cite{bib:protos} axigluon samples that have a
non-zero asymmetry. Figure~\ref{fig:app:fbuErrLhoodCut} illustrates
the reduction in the unfolding statistical error that can be obtained
by requiring that the kinematic fit likelihood is greater than $-50$.
\begin{figure}\centering
  \includegraphics[width=0.45\textwidth]{figures/appendix/fbu/linearityAcIncl_bin_flatPrior}
  \includegraphics[width=0.45\textwidth]{figures/appendix/fbu/linearityAcIncl_bin_curvature1e-8Prior}
  \caption{
    \label{fig:app:fbuLin}
    Unfolding with FBU for $\ac{}_{\text{incl}}$: linearity for
    several \dy{} binnings using a flat prior (left) and a regularization prior 
    based on the curvature of the \dy{} distribution (right).
    }
\end{figure}
%
\begin{figure}
  \begin{center}
  \includegraphics[width=0.45\textwidth]{figures/appendix/fbu/unfolded_map_dymass}
  \caption{
    \label{fig:app:fbuUnfACmtt}
    Unfolded \dy{} distribution for axigluon sample with 2\% asymmetry
    in the five \ttbar{} mass bins using a flat prior.
    }
  \end{center}
\end{figure}
%
\begin{figure}
  \begin{center}
  \includegraphics[width=0.45\textwidth]{figures/appendix/fbu/map_dymass_fbuLinearity_asymmetries1__ens}
  \includegraphics[width=0.45\textwidth]{figures/appendix/fbu/map_dymass_fbuLinearity_asymmetries2__ens}
  \includegraphics[width=0.45\textwidth]{figures/appendix/fbu/map_dymass_fbuLinearity_asymmetries3__ens}
  \includegraphics[width=0.45\textwidth]{figures/appendix/fbu/map_dymass_fbuLinearity_asymmetries4__ens}
  \includegraphics[width=0.45\textwidth]{figures/appendix/fbu/map_dymass_fbuLinearity_asymmetries5__ens}
  \caption{
    \label{fig:app:fbuLinACmtt}
    Unfolding with FBU for $\ac{}_{\mtt{}}$: linearity in the five
    \ttbar{} mass bins. Red curves are obtained
    without the cut on the kinematic fit likelihood; black curves are
    obtained applying the cut.
  }
  \end{center}
\end{figure}
%
\begin{figure}
  \begin{center}
  \includegraphics[width=0.45\textwidth]{figures/appendix/fbu/unfolded_map_dyy}
  \caption{
    \label{fig:app:fbuUnfACmtt}
    Unfolded \dy{} distribution for axigluon sample with 2\% asymmetry in the three \ttbar{} rapidity bins
    using a flat prior.
    }
  \end{center}
\end{figure}
%
\begin{figure}
  \begin{center}
  \includegraphics[width=0.45\textwidth]{figures/appendix/fbu/map_dyy_fbuLinearity_asymmetries1__ens}
  \includegraphics[width=0.45\textwidth]{figures/appendix/fbu/map_dyy_fbuLinearity_asymmetries2__ens}
  \includegraphics[width=0.45\textwidth]{figures/appendix/fbu/map_dyy_fbuLinearity_asymmetries3__ens}
  \caption{
    \label{fig:app:fbuLinACytt}
    Unfolding with FBU for $\ac{}_{\ytt{}}$: linearity in the three \ttbar{} rapidity bins. Red curves are obtained
    without the cut on the kinematic fit likelihood; black curves are
    obtained applying the cut.
    }
  \end{center}
\end{figure}
%
\begin{figure}
  \begin{center}
  \includegraphics[width=0.45\textwidth]{figures/appendix/fbu/unfolded_map_dypt}
  \caption{
    \label{fig:app:fbuUnfACmtt}
    Unfolded \dy{} distribution for axigluon sample with 2\% asymmetry
    in the three \ttbar{} transverse momentum bins.
    }
  \end{center}
\end{figure}
%
\begin{figure}
  \begin{center}
  \includegraphics[width=0.45\textwidth]{figures/appendix/fbu/map_dypt_fbuLinearity_asymmetries1__ens}
  \includegraphics[width=0.45\textwidth]{figures/appendix/fbu/map_dypt_fbuLinearity_asymmetries2__ens}
  \includegraphics[width=0.45\textwidth]{figures/appendix/fbu/map_dypt_fbuLinearity_asymmetries3__ens}
  \caption{
    \label{fig:app:fbuLinACpTtt}
    Unfolding with FBU for $\ac{}_{\ytt{}}$: linearity in the
    three \ttbar{} transverse momentum bins. Red curves are obtained
    without the cut on the kinematic fit likelihood; black curves are
    obtained applying the cut.
    }
  \end{center}
\end{figure}
%
\begin{figure}
  \begin{center}
  \includegraphics[width=0.45\textwidth]{figures/appendix/fbu/map_dy_fbuLinearity_asymmetries1_beta06_ens}
  \caption{
    \label{fig:app:fbuLinbeta06}
    Unfolding with FBU for $\ac{}_{\text{incl}}$ with \betatt\textgreater0.6: linearity. Red curves are obtained
    without the cut on the kinematic fit likelihood; black curves are
    obtained applying the cut.
    }
  \end{center}
\end{figure}
%
\begin{figure}
  \begin{center}
  \includegraphics[width=0.45\textwidth]{figures/appendix/fbu/map_dymass_fbuLinearity_asymmetries1_beta06_ens}
  \includegraphics[width=0.45\textwidth]{figures/appendix/fbu/map_dymass_fbuLinearity_asymmetries2_beta06_ens}
  \includegraphics[width=0.45\textwidth]{figures/appendix/fbu/map_dymass_fbuLinearity_asymmetries3_beta06_ens}
  \includegraphics[width=0.45\textwidth]{figures/appendix/fbu/map_dymass_fbuLinearity_asymmetries4_beta06_ens}
  \includegraphics[width=0.45\textwidth]{figures/appendix/fbu/map_dymass_fbuLinearity_asymmetries5_beta06_ens}
  \caption{
    \label{fig:app:fbuLinACmttbeta06}
    Unfolding with FBU for $\ac{}_{\mtt{}}$ with \betatt\textgreater0.6: linearity in the five
    \ttbar{} mass bins. Red curves are obtained
    without the cut on the kinematic fit likelihood; black curves are
    obtained applying the cut.
  }
  \end{center}
\end{figure}
%
\begin{figure}
  \begin{center}
  \includegraphics[width=0.45\textwidth]{figures/appendix/fbu/slopeVsDiffBin}
  \includegraphics[width=0.45\textwidth]{figures/appendix/fbu/offsetVsDiffBin}
  \caption{
    \label{fig:app:fbuLinSlopeOffset}
    Unfolding with FBU: slope (left) and offset (right) from the
    linearity fit for $\ac{}_{\mtt{}}$, $\ac{}_{\pttt{}}$, and $\ac{}_{\ytt{}}$.
    }
  \end{center}
\end{figure}
%
\begin{figure}
  \begin{center}
  \includegraphics[width=0.45\textwidth]{figures/appendix/fbu/acErrVsDiffBin}
  \includegraphics[width=0.45\textwidth]{figures/appendix/fbu/pulls_asymmetries1}
  \caption{
    \label{fig:app:fbuErrLhoodCut}
    Two-dimensional unfolding with FBU: \ac{} statistical error as a
    function of the differential bin (left) and example pull
    distribution (right) for the first \mtt{} bin.
  }
  \end{center}
\end{figure}
\clearpage


\subsection{Pulls}
We checked that the mean and RMS of the \ac{} posteriors are good estimators for the \ac{} interval by
looking at pull distributions produced from ten thousand pseudo-experiments.
Figures \ref{fig:app:pulls_inclu},\ref{fig:app:pulls_vs_mass},\ref{fig:app:pulls_vs_y} and \ref{fig:app:pulls_vs_pt} show
the pull distributions for inclusive and differential measurements.
The RMS close to unity indicates the intervals are correctly estimated.

\begin{figure}
  \begin{center}
  \includegraphics[width=0.45\textwidth]{figures/appendix/fbu/ensemble_map_dy_asymmetries1_noReg}
  \includegraphics[width=0.45\textwidth]{figures/appendix/fbu/ensemble_map_dy_asymmetries1}
  \caption{
    \label{fig:app:pulls_inclu}
    Pull distributions for inclusive \ac{} with flat prior (left) and with regularization (right).
  }
  \end{center}
\end{figure}

\begin{figure}
  \begin{center}
  \includegraphics[width=0.45\textwidth]{figures/appendix/fbu/ensemble_map_dymass_asymmetries1}
  \includegraphics[width=0.45\textwidth]{figures/appendix/fbu/ensemble_map_dymass_asymmetries2}
  \includegraphics[width=0.45\textwidth]{figures/appendix/fbu/ensemble_map_dymass_asymmetries3}
  \includegraphics[width=0.45\textwidth]{figures/appendix/fbu/ensemble_map_dymass_asymmetries4}
  \includegraphics[width=0.45\textwidth]{figures/appendix/fbu/ensemble_map_dymass_asymmetries5}
  \caption{
    \label{fig:app:pulls_vs_mass}
    Pull distributions for \ac{} in the five \mtt{} bins.
  }
  \end{center}
\end{figure}

\begin{figure}
  \begin{center}
  \includegraphics[width=0.45\textwidth]{figures/appendix/fbu/ensemble_map_dypt_asymmetries1}
  \includegraphics[width=0.45\textwidth]{figures/appendix/fbu/ensemble_map_dypt_asymmetries2}
  \includegraphics[width=0.45\textwidth]{figures/appendix/fbu/ensemble_map_dypt_asymmetries3}
  \caption{
    \label{fig:app:pulls_vs_pt}
    Pull distributions for \ac{} in the three \pttt{} bins.
  }
  \end{center}
\end{figure}

\begin{figure}
  \begin{center}
  \includegraphics[width=0.45\textwidth]{figures/appendix/fbu/ensemble_map_dyy_asymmetries1}
  \includegraphics[width=0.45\textwidth]{figures/appendix/fbu/ensemble_map_dyy_asymmetries2}
  \includegraphics[width=0.45\textwidth]{figures/appendix/fbu/ensemble_map_dyy_asymmetries3}
  \caption{
    \label{fig:app:pulls_vs_y}
    Pull distributions for \ac{} in the three \ytt{} bins.
  }
  \end{center}
\end{figure}






\subsection{Acceptance efficiency in asymmetric samples}
Fig.~\ref{fig:app:protosEff} shows that the acceptance in \dy{} bins does not change significantly in the reweighted \alpgen{} samples used for the linearity test.

\begin{figure}
  \begin{center}
  \includegraphics[width=0.45\textwidth]{figures/appendix/fbu/protosEfficiencies}
  \caption{
    \label{fig:app:protosEff}
    Acceptance efficiencies for the six asymmetric \alpgen{} samples obtained by reweighting with truth \dy{} distributions from protos axigluon samples.
  }
  \end{center}
\end{figure}
\clearpage

\subsection{Matrix Inversion}
\label{sec:app:unfolding:matinv}
The FBU unfolding with a flat prior corresponds to the plain matrix inversion.
In this section we compare the unfolded \dy{} distribution and asymmetry obtained
with FBU with the ones obtained with the matrix inversion method.

Fig.~\ref{fig:app:fbuvsmatinvbins} shows the FBU posteriors for \dy{} bins compared 
with ensembles of the unfolded bin content obtained with matrix inversion.
The mean and RMS values are compatible in the two cases.
fig.~\ref{fig:app:fbuvsmatinvasymm} shows the FBU posterior and the matrix inversion 
ensemble distribution for \ac{}. The distributions obtained with the two methods are cmpatible. 
\begin{figure}
  \begin{center}
    \includegraphics[width=0.45\textwidth]{figures/appendix/fbu/FBUvsMI_dim1_map_dy}
    \includegraphics[width=0.45\textwidth]{figures/appendix/fbu/FBUvsMI_dim2_map_dy}
    \includegraphics[width=0.45\textwidth]{figures/appendix/fbu/FBUvsMI_dim3_map_dy}
    \includegraphics[width=0.45\textwidth]{figures/appendix/fbu/FBUvsMI_dim4_map_dy}
    \caption{
      \label{fig:app:fbuvsmatinvbins}
      FBU posterior distributions for the four unfolded \dy{} bins (black points) 
      with overlayed the distributions of corresponding bin content from ensembles
      of unfolded \dy{} distributions obtained with matrix inversion (blue line).
    }
  \end{center}
\end{figure}

\begin{figure}
  \begin{center}
    \includegraphics[width=0.45\textwidth]{figures/appendix/fbu/FBUvsMI_asymmetry_map_dy}
    \caption{
      \label{fig:app:fbuvsmatinvasymm}
      FBU posterior distributions for \ac{} (black points) 
      with overlayed the \ac{} distributions from ensembles
      of unfolded \dy{} distributions obtained with matrix inversion (blue line).
    }
  \end{center}
\end{figure}

