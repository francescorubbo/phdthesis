In this appendix we illustrate the details of the unfolding
configuration and validation.

\section{Regularization}
\label{app:unfolding:regularization}
In this section we briefly summarize the details of the regularization
used for the inclusive \ac{} measurement at \seventev{} and for the
differential one as a function of \pttt{}.

We use the definition of curvature
%
\begin{equation}
        C(\Truth{}) =
        \sum_{t=2}^{N_b-1}\left[(T_{t+1}-T_{t})-(T_t-T_{t-1})\right]^2
\end{equation}
%
where $N_b$ is the number of bins. We compute the regularization
function as the absolute value of the curvature difference between the
truth spectrum and the $\tilde{\Truth{}}$ spectrum of the simulated
events that are used to build the response matrix:
%
\begin{equation}
 S(\Truth{}) = |C(\Truth) - C(\tilde{\Truth{}})|.
\end{equation}
%
The main idea behind this function is that it decreases the weight of
sampled points \Truth{} with curvature that is very different from
the one of the simulated spectrum.
For the differential measurement, the curvature is
computed separately in each differential subrange and the
regularization function is the sum of the regularization functions
computed in each subrange.  The curvature is computed
separately within each subrange in order to skip the transition
between one subrange and the next. 
Fig.~\ref{fig:app:fbutestprior} shows the posterior probability
density for the inclusive \ac{} corresponding to four different
choices of the regularization parameter $\alpha{}$.
While the width of the posterior is sensibly reduced for larger values
of $\alpha{}$, the mean value is stable.

\begin{figure}
  \begin{center}
  \includegraphics[width=0.45\textwidth]{figures/appendix/fbu/FBUtestprior}
  \caption{
    \label{fig:app:fbutestprior}
    Posterior probability density for the inclusive \ac{}
    corresponding to four values of $\alpha{}$.
    }
  \end{center}
\end{figure}

\section{Binning}
\label{app:unfolding:binning}

All measurements are performed using four bins for the \dy{}
distribution.
The bin edges are adjusted in the different kinematic regions in order
to take into account the varying width of the distribution and \dy{}
resolution. The simplest configuration yielding a negligible bias in
the unfolding response is chosen. In the measurements at \seventev{},
the same \dy{} binning, summarized in Table~\ref{tab:unf:binning7tev},
is used for all differential measurements. 
\begin{table}[htbp]
  \begin{center}
    \begin{tabular}{l l r}
      \toprule
      Variable & & Bin edges \\
      \midrule
      \dy{}         &            & $[-5.0, -0.5, 0.0, +0.5, +5.0]$ \\
      \mtt{}        & $[\GeV{}]$ & $[0, 420, 500, 600, 750,
      +\infty{}]$ \\
      \pttt{}       & $[\GeV{}]$ & $[0, 25, 60, +\infty{}]$ \\
      |\ytt{}|     &            & $[0, 0.3, 0.7, +\infty{}]$ \\
      \bottomrule
     \end{tabular}
   \end{center}
  \caption{Bin edges of the \dy{}, \mtt{}, \pttt{}, and |\ytt{}|
  bins in the measurements at \seventev{}.}
  \label{tab:unf:binning7tev}
\end{table}
The binning choices for the measurements at \eighttev{} is detailed in
Table~\ref{tab:unf:binning8tev}.
\begin{table}[htbp]
  \begin{center}
    \begin{tabular}{l r}
      \toprule
      Differential bin & Bin edges \\
      \midrule
      Inclusive & $[-5.0, -0.8, 0.0, +0.8, +5.0]$ \\
      \midrule
      $\mtt{}<420\GeV{}$ & $[-5.0, -0.3, 0.0, +0.3, +5.0]$ \\
      $420\GeV{}\mtt{}<500\GeV{}$ & $[-5.0, -0.6, 0.0, +0.6, +5.0]$ \\
      $500\GeV{}\mtt{}<600\GeV{}$ & $[-5.0, -0.9, 0.0, +0.9, +5.0]$ \\
      $600\GeV{}\mtt{}<750\GeV{}$ & $[-5.0, -1.2, 0.0, +1.2, +5.0]$ \\
      $750\GeV{}\mtt{}<900\GeV{}$ & $[-5.0, -1.4, 0.0, +1.4, +5.0]$ \\
      $\mtt{}>900\GeV{}$ & $[-5.0, -1.4, 0.0, +1.4, +5.0]$ \\
      \bottomrule
     \end{tabular}
   \end{center}
  \caption{Bin edges of the \dy{} bins in each \mtt{} range in the
    measurement at \eighttev{}.}
  \label{tab:unf:binning8tev}
\end{table}

\section{Unfolding response}
\label{app:unfolding:linearity}

In this section the parameters of the unfolding response, measured as
described in Sec.~\ref{sec:binandbias}, are reported in
Table~\ref{tab:unf:response7tev} and \ref{tab:unf:response8tev} for
the measurements at \seventev{} and \eighttev{}, respectively. 

\begin{table}[htbp]
  \begin{center}
    \begin{tabular}{l c c}
      \toprule
      Differential bin & $a$ & $b$ \\
      \midrule
      Inclusive                           & $1.04\pm0.01$ & $0.001\pm0.001$\\
      Incl., $\betatt{}>0.6$        & $1.05\pm0.02$ & $-0.002\pm0.001$\\
      Incl., $\mtt{}>600\GeV{}$ & $0.99\pm0.02$ & $-0.002\pm0.002$\\
      \midrule
      $\mtt{}<420\GeV{}$                & $0.98\pm0.09$  & $0.000\pm0.002$\\
      $420\GeV{}<\mtt{}<500\GeV{}$& $0.99\pm0.05$  & $0.002\pm0.002$  \\
      $500\GeV{}<\mtt{}<600\GeV{}$& $0.97\pm0.05$  & $-0.005\pm0.002$\\
      $600\GeV{}<\mtt{}<750\GeV{}$& $0.97\pm0.05$ & $-0.007\pm0.002$\\
      $\mtt{} >750\GeV{}$& $0.99\pm0.06$ & $-0.003\pm0.003$\\
      \midrule
      $|\ytt{}|<0.3$                & $0.98\pm0.12$  & $-0.003\pm0.002$\\
      $0.3<|\ytt{}|<0.7$         & $0.99\pm0.04$  & $0.004\pm0.001$\\
      $|\ytt{}|>0.7$                & $0.98\pm0.03$  & $0.000\pm0.001$\\
      \midrule
      $\pttt{}<25\GeV{}$                & $1.01\pm0.06$  & $-0.004\pm0.002$\\
      $25\GeV{}<\pttt{}<60\GeV{}$& $1.02\pm0.05$  & $0.008\pm0.003$  \\
      $500\GeV{}<\mtt{}<600\GeV{}$& $1.04\pm0.04$  & $-0.001\pm0.001$\\
      \midrule
      $\betatt{}>0.6, \mtt{}<420\GeV{}$                & $0.96\pm0.11$  & $-0.004\pm0.003$\\
      $\betatt{}>0.6, 420\GeV{}<\mtt{}<500\GeV{}$& $0.99\pm0.07$  & $0.004\pm0.003$  \\
      $\betatt{}>0.6, 500\GeV{}<\mtt{}<600\GeV{}$& $0.97\pm0.07$  & $-0.006\pm0.004$\\
      $\betatt{}>0.6, 600\GeV{}<\mtt{}<750\GeV{}$& $0.94\pm0.07$ & $-0.007\pm0.006$\\
      $\betatt{}>0.6, \mtt{} >750\GeV{}$& $0.88\pm0.08$ & $0.003\pm0.007$\\
      \bottomrule
     \end{tabular}
   \end{center}
  \caption{Coefficient ($a$) and offset ($b$) parameters of the unfolding
    response for measurements at \seventev{}.}
  \label{tab:unf:response7tev}
\end{table}

\begin{table}[htbp]
  \begin{center}
    \begin{tabular}{l c c}
      \toprule
      Differential bin & $a$ & $b$ \\
      \midrule
      Inclusive & $0.9993\pm0.0019$ & $-0.0003\pm0.0001$\\
      \midrule
      $\mtt{}<420\GeV{}$                & $0.988\pm0.009$  & $-0.0004\pm0.0002$\\
      $420\GeV{}<\mtt{}<500\GeV{}$& $0.9968\pm0.0034$  & $-0.0001\pm0.0002$  \\
      $500\GeV{}<\mtt{}<600\GeV{}$& $0.9833\pm0.0029$  & $-0.0002\pm0.0002$\\
      $600\GeV{}<\mtt{}<750\GeV{}$& $0.9581\pm0.0029$ & $0.0014\pm0.0002$\\
      $750\GeV{}<\mtt{}<900\GeV{}$& $0.924\pm0.005$ & $-0.0013\pm0.0004$\\
      $\mtt{}>900\GeV{}$                & $0.958\pm0.005$ & $-0.0003\pm0.0004$\\
      \bottomrule
     \end{tabular}
   \end{center}
  \caption{Coefficient ($a$) and offset ($b$) parameters of the unfolding
    response for measurements at \eighttev{}.}
  \label{tab:unf:response8tev}
\end{table}

\section{Pulls}
\label{app:unfolding:pulls}

In order to check that the mean and RMS of the \ac{} posteriors are
good estimators for the \ac{} interval the pull distributions are
produced from ten thousand pseudo-experiments.
Figures \ref{fig:app:pulls_inclu}, \ref{fig:app:pulls_vs_mass},
\ref{fig:app:pulls_vs_y} and \ref{fig:app:pulls_vs_pt} show the pull
distributions for inclusive and differential measurements at \seventev{}.
The pull distributions for the differential measurement at \eighttev{}
are shown in Fig.~\ref{fig:app:pulls_vs_mass8tev}
The RMS close to unity indicates the intervals are correctly estimated.

\begin{figure}
  \begin{center}
  \includegraphics[width=0.45\textwidth]{figures/appendix/fbu/ensemble_map_dy_asymmetries1_noReg}
  \includegraphics[width=0.45\textwidth]{figures/appendix/fbu/ensemble_map_dy_asymmetries1}
  \caption{
    \label{fig:app:pulls_inclu}
    Pull distributions for inclusive \ac{} at \seventev{} with flat prior (left) and with regularization (right).
  }
  \end{center}
\end{figure}

\begin{figure}
  \begin{center}
  \includegraphics[width=0.45\textwidth]{figures/appendix/fbu/ensemble_map_dymass_asymmetries1}
  \includegraphics[width=0.45\textwidth]{figures/appendix/fbu/ensemble_map_dymass_asymmetries2}
  \includegraphics[width=0.45\textwidth]{figures/appendix/fbu/ensemble_map_dymass_asymmetries3}
  \includegraphics[width=0.45\textwidth]{figures/appendix/fbu/ensemble_map_dymass_asymmetries4}
  \includegraphics[width=0.45\textwidth]{figures/appendix/fbu/ensemble_map_dymass_asymmetries5}
  \caption{
    \label{fig:app:pulls_vs_mass}
    Pull distributions for \ac{} at \seventev{} in the five \mtt{} bins.
  }
  \end{center}
\end{figure}

\begin{figure}
  \begin{center}
  \includegraphics[width=0.45\textwidth]{figures/appendix/fbu/ensemble_map_dypt_asymmetries1}
  \includegraphics[width=0.45\textwidth]{figures/appendix/fbu/ensemble_map_dypt_asymmetries2}
  \includegraphics[width=0.45\textwidth]{figures/appendix/fbu/ensemble_map_dypt_asymmetries3}
  \caption{
    \label{fig:app:pulls_vs_pt}
    Pull distributions for \ac{} at \seventev{} in the three \pttt{} bins.
  }
  \end{center}
\end{figure}

\begin{figure}
  \begin{center}
  \includegraphics[width=0.45\textwidth]{figures/appendix/fbu/ensemble_map_dyy_asymmetries1}
  \includegraphics[width=0.45\textwidth]{figures/appendix/fbu/ensemble_map_dyy_asymmetries2}
  \includegraphics[width=0.45\textwidth]{figures/appendix/fbu/ensemble_map_dyy_asymmetries3}
  \caption{
    \label{fig:app:pulls_vs_y}
    Pull distributions for \ac{} at \seventev{} in the three |\ytt{}| bins.
  }
  \end{center}
\end{figure}

\begin{figure}
  \begin{center}
    \includegraphics[width=0.495\textwidth]{figures/appendix/fbu/pulls_bin0_mtt_6bins_asimov_combined_mtt_Pulls}
    \includegraphics[width=0.495\textwidth]{figures/appendix/fbu/pulls_bin1_mtt_6bins_asimov_combined_mtt_Pulls}
    \includegraphics[width=0.495\textwidth]{figures/appendix/fbu/pulls_bin2_mtt_6bins_asimov_combined_mtt_Pulls}
    \includegraphics[width=0.495\textwidth]{figures/appendix/fbu/pulls_bin3_mtt_6bins_asimov_combined_mtt_Pulls}
    \includegraphics[width=0.495\textwidth]{figures/appendix/fbu/pulls_bin4_mtt_6bins_asimov_combined_mtt_Pulls}
    \includegraphics[width=0.495\textwidth]{figures/appendix/fbu/pulls_bin5_mtt_6bins_asimov_combined_mtt_Pulls}
  \caption{
    \label{fig:app:pulls_vs_mass8tev}
    Pull distributions for \ac{} at \eighttev{} in the six \mtt{} bins at \eighttev{}.
  }
  \end{center}
\end{figure}

\section{Marginalization}
\label{app:unfolding:marginalization}

\subsection{\ac{} as a function of \mtt{}}

The expected precision and measured values for the nuisance parameters
in the measurement as a function of \mtt{} at \eighttev{} is shown in
Fig.~\ref{fig:nuisparmtt}.
\begin{figure}[!htb]\centering
  \includegraphics[width=0.6\textwidth]{figures/appendix/fbu/nuisparmtt}
  \caption{Fitted nuisance parameters to pseudo--data (black) and data
    (red) for \ac{} measurement as a function of \mtt{} at \eighttev{}. The
    shaded regions highlight the $1\sigma$ (green) and $2\sigma$
    intervals of the prior probability density.}
  \label{fig:nuisparmtt}
\end{figure}
The \wjets{} calibration obtained in this measurement is consistent
with the one obtained in the inclusive measurement, as shown in
Fig.~\ref{fig:wnormincluvsmtt}. 
\begin{figure}[!htb]\centering
  \includegraphics[width=0.6\textwidth]{figures/appendix/fbu/wnormincluvsmtt}
  \caption{Fitted \wjets{} scale factors $K$s to data for the inclusive
    measurement (black) and for the measurement as a function of
    \mtt{} (red) at \eighttev{}. The results are consistent.}
  \label{fig:wnormincluvsmtt}
\end{figure}

\subsection{Comparison of \mujets{} and \ejets{}}
\label{app:unf:evsmu}

The measured values for the nuisance parameters
in the inclusive measurement at \eighttev{} are shown in
Fig.~\ref{fig:nuisparmtt} considering only \mujets{} or \ejets{}
events. The nuisance parameters are found to be consistent.
\begin{figure}[!htb]\centering
  \includegraphics[width=0.6\textwidth]{figures/appendix/fbu/nuisparmuvse}
  \caption{Fitted nuisance parameters to \mujets{} data (black) and
    \ejets{} data (red) for the inclusive \ac{} measurement at \eighttev{}. The
    shaded regions highlight the $1\sigma$ (green) and $2\sigma$
    intervals of the prior probability density.}
  \label{fig:nuisparmtt}
\end{figure}
The \wjets{} calibration yields similar scale factor, as shown in
Fig.~\ref{fig:wnormmuvse}.
\begin{figure}[!htb]\centering
  \includegraphics[width=0.6\textwidth]{figures/appendix/fbu/wnormmuvse}
  \caption{Fitted $W+jets$ scale factors $K$s to \mujets{} data (black)
    and \ejets{} data (red) for the inclusive
    measurement at \eighttev{}. The results are consistent.}
  \label{fig:wnormmuvse}
\end{figure}

\section{Limited Monte-Carlo statistics}
\label{app:unfolding:mcstat}

The impact of the limited Monte-Carlo statistics is evaluated with an
ensemble of 1000 response matrices, applying fluctuations according
to the number of Monte-Carlo event.

\begin{figure}[htbp]
\begin{center}
\begin{tabular}{cc}                                    
{(a)  $\mtt{} < 420$ GeV} & {(b) $420 < \mtt{} < 500$ GeV}  \\
 \includegraphics[width=0.495\textwidth]{figures/appendix/fbu/Ac_bin0_mtt_6bins_asimov_asimov_combined_mtt_smeared} &
 \includegraphics[width=0.495\textwidth]{figures/appendix/fbu/Ac_bin1_mtt_6bins_asimov_asimov_combined_mtt_smeared} \\
{(c) $500 < \mtt{} < 600$ GeV} & {(d) $600 < \mtt{} < 750$ GeV}  \\
 \includegraphics[width=0.495\textwidth]{figures/appendix/fbu/Ac_bin2_mtt_6bins_asimov_asimov_combined_mtt_smeared} &
 \includegraphics[width=0.495\textwidth]{figures/appendix/fbu/Ac_bin3_mtt_6bins_asimov_asimov_combined_mtt_smeared} \\
{(e) $750 < \mtt{} < 900$ GeV} & {(f) $\mtt{} > 900$ GeV}  \\
 \includegraphics[width=0.495\textwidth]{figures/appendix/fbu/Ac_bin4_mtt_6bins_asimov_asimov_combined_mtt_smeared} &
 \includegraphics[width=0.495\textwidth]{figures/appendix/fbu/Ac_bin5_mtt_6bins_asimov_asimov_combined_mtt_smeared} \\
\end{tabular}\caption{Distribution of the unfolded \ac{} in the six
  \mtt{} bins for an ensemble of 1000 response matrices. The dotted
  vertical line shows the truth \ac{} value.}
\label{fig:app:unf:limstatmtt}
\end{center}
\end{figure}
