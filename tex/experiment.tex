\chapter{The ATLAS experiment at the LHC}
\label{sec:experiment}

The ATLAS detector is one of the two general-purpose experiments 
at the Large Hadron Collider (LHC) at CERN, Switzerland.
It has been built to pursue a broad particle physics program. 

The LHC is described in Sec.~\ref{sec:LHC}, while the design and operation of the 
ATLAS detector are discussed in Sec.~\ref{sec:ATLAS}.

\section{The Large Hadron Collider}
\label{sec:LHC}

The Large Hadron Collider (LHC)~\cite{LHCreport12, *LHCreport3} 
is a circular proton-proton ($pp$) collider.
It consists of a 27~km ring of superconducting magnets and
accelerating cavities, where two beams of protons circulate in
opposite directions.
The magnets bend and focus the proton beams into a circular trajectory while radio-frequency
cavities boost their energy.
The beams undergo several acceleration steps in the pre-accelerators chain
and in the LHC, which are both part of the CERN accelerator complex, shown in Fig.~\ref{fig:lhc}.
Protons from the ionization of hydrogen atoms are first accelerated to \mbox{$50 \MeV$}
in a linear collider~(LINAC2), then the energy is increased up to
\mbox{$450 \GeV$} through three stages of synchrotrons: the Proton
Synchrotron Booster~(BOOSTER), the Proton Synchrotron~(PS) and the Super Proton Synchrotron~(SPS).
Finally the protons are injected into the LHC.

\begin{figure}[ht]
\begin{center}
\includegraphics[width=0.9\textwidth]{figures/experiment/accelerator_chain}
\caption[CERN Accelerator complex]{
  The CERN accelerator complex. The four main LHC experiments are
  shown at the four interaction points.}
\label{fig:lhc}
\end{center}
\end{figure}
 
In 2011 each beam was accelerated to an energy of \mbox{$3.5 \TeV$}, 
resulting into a center-of-mass energy \mbox{$\sqrt{s} = 7 \TeV$}.
In 2012, the beam energy was increased to \mbox{$4 \TeV$}, 
corresponding to \mbox{$\sqrt{s} = 8 \TeV$}.
The LHC was designed to collide protons up to \mbox{$\sqrt{s} = 14 \TeV$},
which is expected for future operation.

The beams are brought to collision in four interaction points along
the ring, where four experiments are situated: 
ATLAS, described in Sec.~\ref{sec:ATLAS}, CMS~\cite{cms}, ALICE~\cite{alice} and LHCb~\cite{lhcb} .

\section{The ATLAS experiment}
\label{sec:ATLAS}

\begin{figure}[ht]
\begin{center}
\includegraphics[width=0.58\textwidth]{figures/experiment/atlas_overview}
\caption[Drawing of the ATLAS detector]{
  Drawing of the ATLAS detector showing the different detectors and magnet systems.}
\label{fig:ATLAS}
\end{center}
\end{figure}

ATLAS (A Toroidal LHC ApparatuS) is a particle detector
experiment~\cite{detectorpaper} designed to observe highly massive
particles and to measure the properties of known particles produced at
unprecedented energies and high rates. 

As shown in Fig.~\ref{fig:ATLAS}, the interaction point is surrounded
by several layers of sub-detectors, each devoted to the measurement of
different properties for different types of particles.
The sub-detectors are grouped into three main systems:

\begin{itemize}
\item The Inner Detectors, described in Sec.~\ref{sec:ID}, immersed in
  a solenoidal magnetic field, constitute a tracking system used to
  identify and measure the momenta of charged particles, and to identify
  the interaction vertices and displaced vertices. 
\item The Calorimeters are used to identify and measure the energy of
  neutral and charged particles. They are designed to stop most types
  of particles, except for muons and
  neutrinos. Sec.~\ref{sec:calorimeter} illustrates the design and
  operation of the calorimetry system.
\item The Muon system is described in
  Sec~\ref{sec:muonspectrometer}. Because muons minimally interact
  with the other parts of the detector and have long lifetimes, they
  are identified and measured in the outermost detector
  layer, which consists of muon spectrometers immersed in a toroidal magnetic field. 
\end{itemize}

In \ttbar{} events with $\ell{}$+jets final state, electrons
and jets are identified by the tracking and calorimetry systems, muons
are identified by the tracking and muon systems, and the transverse
components of the neutrino momenta are inferred from the imbalance in
the total momenta measured in the detector.

\subsection{Inner detectors}
\label{sec:ID}

\begin{figure}[ht]
\centering
\includegraphics[width=0.4\textwidth]{figures/experiment/in_det}
\includegraphics[width=0.3\textwidth]{figures/experiment/in_det_dist}
\caption[Overview of the Inner Detector]{
  Overview of the Inner Detector: 
  the left figure shows a longitudinal section of the Inner Detector
  with the different sub-detectors.
  The right figure shows a transverse section and illustrates the
  distances of the different detector layers from the beam line.}
\label{fig:ID}
\end{figure}

The ATLAS Inner Detector (ID) consists of three sub-detector systems:
the Pixel detector and the SemiConductor Tracker~(SCT), which use silicon semiconductor
technology, and the Transition Radiation Tracker~(TRT), 
which exploits the transition radiation produced in a gas mixture of Xe, CO$_2$ and O$_2$.
Fig.~\ref{fig:ID} shows a longitudinal and a transverse section of the ID.
The whole ID is immersed in a \mbox{$2 \T$} solenoidal field (Sec.~\ref{sec:magnets}).

With three concentric cylinders~(barrel part), and three endcap disks
perpendicular to the beam axis, the Pixel detector covers a
range\footnote{ATLAS uses a right-handed coordinate system
  with its origin at the nominal interaction point (IP) in the centre
  of the detector and the $z$-axis along the beam pipe. 
  The $x$-axis points from the IP to the centre of the LHC ring, and the $y$-axis points
  upwards. Cylindrical coordinates ($r$, $\phi$) are used in the
  transverse plane, where $\phi$ is the azimuthal angle around the beam pipe. The
  pseudorapidity is defined in terms of the polar angle $\theta$ as 
  \mbox{$\eta = - \ln \left[ \tan \left( \frac{\theta}{2} \right)
    \right]$}.} of \mbox{$|\eta| < 2.5$}.
Each of the 1744 sensors consists of a segmented silicon wafer with
pixels of minimum area \mbox{$50 \times 400 \mum^2$} and 46080 readout channels.
The innermost pixel layer, called $b$-layer, is as close to the beam
line as \mbox{$50.5 \mm$} and allows for a precise extrapolation of tracks
to the vertices.
This is crucial for the identification of displaced secondary vertices.

The SCT consists of four layers in the barrel and nine endcap disks.
It covers the range \mbox{$|\eta| < 2.5$}.
In the barrel, silicon strips are arranged parallel to the beam line, while in the disks, the strips are oriented radially.
Modules are arranged back-to-back with a small stereo angle of \mbox{$40 \mrad$} to allow for a measurement of the azimuth angle in each layer.
A typical track yields three space-points in the Pixel detector and eight in the SCT.
Together, the silicon trackers ensure the measurement of the track momenta and the identification of primary and secondary vertices.

In the barrel part of the TRT, there are 73 planes of straw tubes
filled with gas mixture, which are arranged parallel to the beam axis.
In the endcap, there are 160 straw planes, oriented radially.
The TRT covers a range of \mbox{$|\eta| < 2.0$}, 
in which the separation of electrons from charged pions
is improved by exploiting transition radiation.

The total amount of material of the ID is as large as roughly 0.5 electromagnetic radiation
lengths\footnote{The radiation length is defined as the typical amount of material traversed by an electron after which it has lost
$\frac{1}{e}$ of its original energy by bremsstrahlung.}
$X_0$ for \mbox{$|\eta| < 0.6$}.
For \mbox{$0.6 < |\eta| < 1.37$} and \mbox{$1.52 < |\eta| < 2.5$}, the amount of material reaches up to \mbox{$1.5 X_0$}.
In the barrel-to-endcap transition region at \mbox{$1.37 < |\eta| < 1.52$}, the amount of material is even larger.
Electrons in this region were not taken into account in this analysis.

\subsection{Calorimeters}
\label{sec:calorimeter}

Fig.~\ref{fig:calorimeters} shows an overview of the different electromagnetic and hadronic calorimeters of the ATLAS detector.

The ATLAS electromagnetic and hadronic calorimeters are sampling
calorimeters consisting of alternating layers of dense absorber
material and active material, where only the active
material is used for the energy measurement.

The hadronic calorimeter in the barrel~(Tile) uses steel as absorber
and scintillators as active material.
The electromagnetic and the forward hadronic calorimeters use Liquid
Argon~(LAr) technology with different types of absorbers:
lead in the ElectroMagnetic Barrel~(EMB) and the ElectroMagnetic
Endcap Calorimeter (EMEC), copper in the Hadronic Endcap
Calorimeter~(HEC) and the electromagnetic part of the Forward Calorimeter~(FCal),
and tungsten in the hadronic part of the FCal.
The LAr calorimeters are placed in three cryostats: one for the barrel and one for each endcap.

\begin{figure}[ht]
\begin{center}
\includegraphics[width=0.4\textwidth]{figures/experiment/calorimeters}
\caption[Overview of the calorimeter system]{
  Overview of the calorimeter system:
  the different subdetectors of the electromagnetic and hadronic calorimeter are shown.}
\label{fig:calorimeters}
\end{center}
\end{figure}

The materials have been chosen to provide fast readout, 
radiation hardness, and high containment of electromagnetic and
hadronic showers to ensure
a precise measurement of their energies.

All calorimeters are finely granulated and also segmented
longitudinally to allow for a precise determination of the position of the showers and to
distinguish different shower types by the use of shower shapes.
This is particularly important for the central region, 
which is devoted to precision measurements of electrons and photons:
the EMB (\mbox{$|\eta| < 1.475$}) is segmented into three longitudinal
layers, where the first layer $\eta$ granularity is 0.0031.
To ensure continuous coverage in azimuth and to enable fast readout, 
the lead absorbers are folded into an accordion shaped structure.

A similar design as for the EMB has been used for the EMEC, 
which is divided into two wheels covering the ranges \mbox{$1.375 < |\eta| < 2.5$}
and \mbox{$2.5 < |\eta| < 3.2$}.
The inner wheel has a coarser granularity in $\eta$ and $\phi$, 
limiting the region devoted to precision physics to \mbox{$|\eta| < 2.5$}.
A thin LAr layer (pre-sampler) is placed in front of the EMB and the
EMEC at \mbox{$|\eta| < 1.8$} to correct for energy lost in front of the calorimeter.

The Tile calorimeter is located behind the EMB and the EMEC and is divided into three longitudinal layers.
It consists of a central barrel (\mbox{$|\eta| < 1.0$}) and an extended-barrel part (\mbox{$0.8 < |\eta| < 1.7$}).

The HEC is a traditional LAr sampling calorimeter covering the region (\mbox{$1.5 < |\eta| < 3.2$}), which is placed behind the EMEC in the same
cryostat.
It consists of two independent wheels, each of which is divided longitudinally into two parts.

Altogether, the calorimeters cover the range \mbox{$|\eta| < 4.9$},
thus providing enough hermeticity to ensure a precise measurement
of the imbalance of the transverse momentum.
Over the whole range in $\eta$, the total thickness of the calorimeter
system ensures a good containment of
electromagnetic and hadronic showers and limits punch-through effects to the muon spectrometer.

\subsection{Muon detectors}
\label{sec:muonspectrometer}

The ATLAS muon system covers the range \mbox{$|\eta| < 2.7$}
and is designed to measure the momenta of muons with energy 
above \mbox{$\sim 3 \GeV$}.
The tracks of the muons are bent by the toroidal magnetic field,
whose orientation is such that muon tracks in both barrel and endcaps 
are mostly orthogonal to the field lines.

The muon system, shown in Fig.~\ref{fig:muons}, consists of high-precision tracking chambers as well as trigger systems.
In the barrel part, Monitored Drift Tubes~(MDTs) are used for tracking and Resistive Plate Chambers~(RPCs) for triggering.
In the endcaps, tracking information is provided by Cathode Strip Chambers~(CSCs) and Thin Gap Chambers~(TGCs) are used for triggering.
In the barrel as well as in the endcaps, muons typically cross three longitudinal layers of the muon spectrometer.
The muon system is divided into eight octants with overlaps in $\phi$ to avoid gaps in the detector coverage.

\begin{figure}[ht]
\begin{center}
\includegraphics[width=0.4\textwidth]{figures/experiment/muonchambers}
\caption[Overview of the muon system]{
  Overview of the muon system:
  the different types of tracking and trigger chambers are shown.}
\label{fig:muons}
\end{center}
\end{figure}

The technologies for the tracking systems have been chosen according to
level of particle flux.
The MDTs in the barrel part follow a robust and reliable detector design.
Since each tube contains only one sense wire, the simple geometry
allows for the prediction of deformations as well as for a precise reconstruction.
As the particle flux is increasing with $|\eta|$, 
the CSCs are more suited for the endcap region:
the higher granularity of the multi-wire proportional chambers 
facilitates to cope with the increasing rates.

The choice of the technologies for the trigger chambers was driven 
by the requirement for fast and highly efficient trigger capabilities
given the different conditions present in the barrel and endcap regions during data taking.
Additionally, an adequate resolution of the transverse momentum of the tracks was required.
In the barrel, RPCs provide good spatial and time resolution.
However, in the region \mbox{$1.05 < |\eta| < 2.4$}  the particle flux
is higher and TGCs, with higher granularity, are used.
With RPCs and TGCs, a time resolution of \mbox{$15 - 25 \ns$} can be
achieved, 
which is sufficient for fast trigger decisions and a good association of
tracks to bunch crossings.

\subsection{Magnet system}
\label{sec:magnets}

The ATLAS magnets system consists of four superconducting
magnets: a central solenoid and three toroid magnets in the barrel and the
two endcaps (Fig.~\ref{fig:magnets}).

The central solenoid provides an axial field with a strength of \mbox{$2 \T$}.
The solenoid was designed to be particularly lightweight and to
minimise the amount of material in front of the calorimeter system.

\begin{figure}[ht]
\begin{center}
\includegraphics[width=0.4\textwidth]{figures/experiment/magnets}
\caption[Overview of the magnet system]{
  Overview of the magnet system:
  the solenoid and the toroids are shown.}
\label{fig:magnets}
\end{center}
\end{figure}

The toroid systems provide magnetic fields with a bending power
of \mbox{$1.5 - 5.5 \Tm$} in the barrel and \mbox{$1 - 7.5 \Tm$} in the endcap regions.
Each system consists of eight coils placed in aluminium housings.
The toroidal fields contain non-uniformities which need to be known to
high precision to allow for an accurate measurement of muon momenta.
Hence, 1800 Hall sensors were installed in the muon spectrometer 
volume to enable the monitoring of the magnetic field.


\subsection{Trigger and data acquisition}
\label{sec:triggerDAQ}

With a nominal bunch spacing of \mbox{$50 \ns$} and
\mbox{$\approx{}1~MB$} event size, ATLAS generates more
information than can be recorded to disk. In 20 million 
interactions per second, approximately one \ttbar{} event 
is produced. It is therefore crucial to have an efficient trigger
system for selecting events of interest.

The ATLAS trigger system, shown schematically in
Fig.~\ref{fig:trigger}, has a three tier structure with increasing
levels of information used in reconstruction, and hence refinement of
the selection criteria, at each stage.

\begin{figure}[ht]
\begin{center}
\includegraphics[width=0.4\textwidth]{figures/experiment/trigger}
\caption[ATLAS trigger system]{
  Schematic diagram of the ATLAS trigger system~\cite{trigger}. }
\label{fig:trigger}
\end{center}
\end{figure}

At the first stage, Level 1 (L1), hardware triggers use coarse
calorimeter and muon information for the trigger decision. At this
level the event accept rate is reduced to a maximum of 75 kHz with a
latency on decision of $\approx{}25~\mus{}$. In the case where the
trigger is passed, the raw event data is sent to the readout stream
for the next trigger level. The L1 trigger defines one or more
\emph{regions-of-interest} (RoIs) in $\eta{}$ and $\phi{}$ where the
L1 trigger has identified interesting features. 

The Level 2 (L2) trigger is software based. At this level the full detector
granularity is used for the trigger decision but only within the RoIs
identified at L1. This allows to reduce the throughput to
$\approx{}2~kHz$ within the allowed latency of $\approx{}40~ms$.

The final trigger level is the Event Filter (EF). At this stage the RoIs
information is treated using the same algorithms as the off-line
reconstruction. The EF reduces the output rate to $\approx{}200~Hz$, 
with a latency of $\approx{}4~s$. Once accepted by an EF trigger an
event is written to mass storage.

During 2011 the RoI seeded approach at EF was replaced by a
$full-scan$ strategy. It was found that sufficient time was available
to read the entire ATLAS calorimeter information, which allows to
improve the on-line selection efficiency.

\subsubsection{The trigger menu}



%Assuming a bunch spacing of \mbox{$25 \ns$} and approximately 20 inelastic interactions per bunch crossing\footnote{As mentioned in Sec.~\ref{sec:LHC}, in 2011 most of the data was taken with a bunch-spacing of \mbox{$50 \ns$}.}, the event rate at the ATLAS detector is
%of the order of 1~GHz.
%A three-level trigger system was set up to reduce this rate to about 200~Hz.
%In 2011 data taking, the real trigger rate was indeed of the order of 300~Hz~\cite{fournier}.
%The triggers need to suppress minimum bias events very strongly while efficiently selecting rare physics events.
%The data acquisition system~(DAQ) gathers the data from the different detector subsystems and buffers them until a trigger decision is received.
%When the event is not rejected by one of the trigger levels, the data are recorded permanently.
%Fig.~\ref{fig:triggerdaq} shows a sketch of the ATLAS trigger chain indicating the order of magnitude of the trigger rates at the different trigger
%levels.
%
%The first trigger level~(L1) is a hardware-based trigger, which reduces the event rate to approximately 75~kHz.
%Muons, electrons, photons, jets and hadronically decaying $\tau$-leptons with high transverse momenta are searched for as well as a large momentum
%imbalance in the transverse plane and a large total transverse energy.
%The muon trigger chambers are used as well as the calorimeter system with reduced granularity.
%Within less than \mbox{$2.5 \mus$}, Regions-of-Interest~(RoI) are identified in $\eta$-$\phi$-space, which serve as seeds for the decision at the second
%trigger level~(L2).
%
%The high level trigger is composed of the L2 and the Event Filter (EF), both of which are software-based trigger systems.
%At L2, the energy and direction of the RoIs are further investigated and also the types of the trigger objects are analysed.
%Within \mbox{$40 \ms$} a decision is made, and the event rate is reduced to below 3.5~kHz.
%The EF further decreases the rate down to roughly 200~Hz.
%Events passing the EF are stored permanently.
%The full event information is available at the EF level and, hence, energies and directions of the trigger objects are estimated with higher precision
%than at L1 and L2.
%In particular, the discrimination between the different particle types is enhanced by the use of the ID tracking system and calorimeter
%shower shapes.
%
%Selections of different trigger signatures are collected in so-called trigger menus.
%For triggers with very high rates, only a fraction of the triggered events can be selected on a random basis
%in order to perform cross-checks and studies of less rare physics processes.
%The trigger menus are adjusted to the data taking conditions, in particular to the instantaneous luminosity, in order to make optimal use of the
%band width available for storage.

